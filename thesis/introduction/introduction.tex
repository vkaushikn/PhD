\chapter{Introduction}
\label{chap:introduction}
In today's highly competitive market, it is important that
the process industries integrate their manufacturing
processes with the downstream supply chain to maximize economic
benefits. For example, BASF was able to generate $\$10$ million/year
savings in operating costs by performing a corporate network
optimization \citep{grossmann:2005}. In recent years, Enterprise Wide
Optimization (EWO) has become an important research area for both
academia and industry. \citet{grossmann:2005} defines EWO as 
\begin{quotation}
An area that lies at the interface of chemical engineering (process systems
engineering) and operations research. It involves optimizing the
operations of supply, manufacturing (batch or continuous) and
distribution in a company. The major operational activities include
planning, scheduling, real-time optimization and inventory control.
\end{quotation}

In process control technologies like Model Predictive Control (MPC),
feedback from the process, in terms of measurements of the current state
of the plant, is used to improve the control performance. It has been
recognized that using feedback control can be significant for supply
chain optimization. \citet{backx:bosgra:marquardt:2000} highlight the
importance of considering the dynamics and feedback in process
integration. They say

\begin{quotation}
Future process innovation must aim at a high degree of adaptability of
manufacturing to the increasingly transient nature of the marketplace
to meet the challenges of global competition. Adaptation to changing
environmental conditions requires feedback control mechanisms, which
manipulate the quality performance and transition flexibility
characteristics of the manufacturing processes on the basis of measured
production performance indicators derived from observations of
critical process variables. This feedback can be achieved by means of
two qualitatively completely different approaches residing on two
different time scales. The first, shorter time scale focused approach
aims at the adaptation of process operations by modified planning,
scheduling and control strategies and algorithms assuming fixed
installations. The second approach attempts to achieve performance
improvements by reengineering the plant, including process and
equipment, as well as instrumentation and operation support system design.
\end{quotation}


Model predictive control is a multi-variable control
algorithm which deals with operating constraints and multi-variable interactions. MPC's ability to handle constraints along
with the online optimization of the control problem has made it a very
popular control algorithm in the process industries
\citep{qin:badgwell:2003,morari:lee:1997}. At the heart of MPC is a
dynamic process model that is used to predict the influence of inputs
(manipulated variables) on the process. Based on the prediction, an
optimization problem is solved online, to find the optimal control
action.  


In this thesis, we propose  model predictive control as a
general purpose tool to aid in enterprise wide
optimization. Traditionally, decision making in the process industries
follows a hierarchical structure. At the top, the planning module uses a
simplified model of the facility along with some knowledge of the
supply chain dynamics to predict production targets and material
flows. This problem is called the Planning problem. In the scheduling
layer, the solution of the planning problem is used to find a detailed
schedule for the plant. This problem is called the Scheduling problem.
The Real Time Optimizer (RTO) uses the solution of the scheduling
problem to find  optimal set-points for the plant.  Finally, the
advanced controller regulates the plant to the predicted set-points. The main contribution of this
thesis is to formulate parts of the short term planning problem (interaction of
the production facility with the supply chain) and the production scheduling
problem as dynamic models (also see hybrid modeling for rolling
horizon approaches \citep[Sec 4.4]{maravelias:sung:2009}), that can be ``controlled'' using MPC. Thus,
in conjunction with economic MPC \citep{amrit:rawlings:angeli:2011} that integrates the advanced control
layer with the RTO; the tools developed in this thesis allows us to
study the entire decision making  hierarchy in the enterprise from a predictive control
point of view.  
 
We focus   on two important aspects of the supply
chain. First, from an operations
research standpoint, we use MPC to coordinate 
orders and shipments in the supply chain to minimize (maximize) costs
(profits). Second, from a process systems engineering standpoint, we
develop tools to formulate the short term production scheduling problem as a dynamic control problem.

\section*{Overview of the thesis}

\paragraph{Chapter 2 -- Model predictive control:} In this chapter, we summarize the fundamental theory for linear MPC. We state
stability theorems for centralized, suboptimal and cooperative MPC. We
then propose a cooperative MPC algorithm that is applicable to all
centralized stabilizable systems and an algorithm for robust
cooperative MPC using tube-based MPC \citep[Chapter 3]{rawlings:mayne:2009}.

\paragraph{Chapter 3 -- A state-space model for chemical production
  scheduling:} In this chapter we derive a state space model for the
production scheduling problem and highlight how different scheduling
disturbances can be modeled. We use ideas from MPC like the terminal
region to show how the state space model can be used in iterative scheduling.

\paragraph{Chapter 4 -- Distributed MPC for supply chain
  optimization:} In this chapter, we show how to model a supply chain, and use the theory outlined in
Chapter 2 to design centralized, distributed and, robust MPC for supply
chains.

\paragraph{Chapter 5 -- Economic MPC for supply chains:} In this
chapter, we briefly review economic MPC theory and show how it can be
tailored for supply chains. Instead of optimizing a tracking
objective, we show how to design economic and multiobjective
optimization problems for supply chains. We conclude this chapter with
an example of an integrated production scheduling--supply chain
problem solved in a rolling horizon framework.

\paragraph{Chapter 6 -- Conclusions and future work:} We end with a
summary of the contributions and recommendations for future work.








