A supply chain is a network of facilities and distribution options
that performs the functions of procuring raw materials, transforming
them to products and distributing the finished products to the
customers. The modern supply chain is a highly interconnected network
of facilities that are spread over multiple locations and handle
multiple products. In a highly competitive global
environment, optimal day-to-day operations of supply chains is
essential.

To facilitate optimal operations in supply chains, we propose the use of
Model Predictive Control (MPC) for supply chains. We develop:  

\begin{itemize}
\item A new cooperative MPC algorithm that can stabilize any
  centralized stabilizable system
\item A new algorithm for robust cooperative MPC
\item A state space model for the chemical production scheduling problem
\end{itemize}

We use the new tools and algorithms to design model predictive
controllers for supply chain models. We demonstrate:

\begin{itemize}
\item {\textbf{Cooperative control for supply chains}}: In cooperative MPC, each node makes its
  decisions by considering the effects of their decisions on the
  entire supply chain. We show that the cooperative controller can
  perform better than the noncooperative and decentralized  controller
  and can reduce the  bullwhip effect in the supply chain. 

\item {\textbf{Centralized economic control}}: We propose a new multiobjective
  stage cost that captures both the economics and risk at a node,
  using a weighted sum of an economic stage cost and a tracking stage
  cost. We use Economic MPC theory \citep{amrit:rawlings:angeli:2011} to design
  closed-loop stable controllers for the supply chain. 

\item {\textbf{Integrated supply chain}}: We show an example of integrating
  inventory control with production scheduling using the tools
  developed in this thesis. We develop simple terminal conditions to
  show recursive feasibility of such integrated control schemes.
\end{itemize}

