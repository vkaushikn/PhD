\chapter{Conclusions and Future work}
\label{chap:conclusions}
We conclude with a summary of contributions and suggest possible
directions for further research.

\section*{Contributions}

\paragraph{Cooperative MPC for linear systems:} In Chapter
\ref{chap:mpc} we provided an overview of cooperative MPC for linear
systems. The main contribution in Chapter \ref{chap:mpc} were (i) The
extension of the class of systems for which cooperative MPC is
applicable to all centralized  stabilizable systems and, (ii) Tube based robust
cooperative MPC to avoid centralized restarts if the warm start fails.

\paragraph{State-space models for scheduling:} In Chapter
\ref{chap:scheduling}, we provided a state-space model for
scheduling. We expressed the scheduling problem as a dynamic problem
for iterative scheduling. We also modeled a variety of scheduling
disturances so that rescheduling occurs ``naturally'' in iterative
scheduling. Finally, we used tools from MPC to demonstrate design of closed-loop scheduling
problems with guaranteed recursive feasibility.

\paragraph{MPC for supply chains:} A goal of this thesis was to use
MPC as a general purpose tool for enterprise wide optimization. We
demonstrated MPC design for dynamic supply chain models. The main
contribution of this thesis is to complement the research in 
MPC/ Rolling horizon optimization frameworks for supply chain
management by (i) showing the desirable properties of algorithms that
guarantee closed-loop stability and, (ii) demonstrating the design of
such control policies for supply chains.  The main message of the thesis  is
that future researchers should appreciate the importance of
considering the closed-loop dynamics of the supply chain as a result
of the input actions taken. 
\begin{enumerate}
\item In order to appeal to the distributed nature of decision making in supply
chains, we demonstrated cooperative MPC for supply chains in which
each node makes its local decisions but with a global vision in
Chapter \ref{chap:sc}. We
proposed a new cooperative MPC iteration scheme which closely
resembles the current decision making hierarchy in linear supply
chains.
\item Since supply chains directly optimize the economics, we
  demonstrated design of Economic MPC for supply chains in Chapter
  \ref{chap:esc}. We proposed a multiobjective stage cost, that not
  only accounts for supply chain costs, but also for supply chain
  risks. The supply chain is  stabilized at a steady-state that
  reflects the managers' choice, i.e., risk seeking or risk averse.
\end{enumerate}

\paragraph{Integration of scheduling and control:} We demonstrated a
supply chain example with an integrated scheduling model for the
manufacturing plant. We showed the integration of the MPC design tools from
Chapter \ref{chap:mpc} and Chapter \ref{chap:esc} along with the
state-space scheduling model from Chapter \ref{chap:scheduling} to
guarantee recursive feasibility for the integrated supply chain
model. We also showed the inherent robustness of the proposed approach
to small deviations from the nominal demand.

\section*{Future work}

\paragraph{Terminal conditions for the scheduling problem:}
In Chapter \ref{chap:scheduling}, we showed recursive feasibility by
using a cyclic schedule as the terminal condition. In many cases, we
might not be able to find any cyclic schedule for the scheduling
problem. In such cases, we have to find other suitable terminal
conditions. One such idea that we are currently exploring is to find
safety constraints on inventory from a scheduling point of view.
Methods of finding terminal conditions for the scheduling problem is
an important area for future research.
\paragraph{Hybrid control theory:}
The scheduling state space model comprises both of continuous
variables (like the inventories, batch sizes, etc.) and discrete variables
(like assignment, changeover etc.). For such systems, we need to study
the stability theory for hybrid systems. There are methods that have
been developed for hybrid dynamic systems consisting of both time and
event driven dynamics \citep{bemporad:morari:1999,
  morari:baotic:borrelli:2003} etc. More recently, Lyapunov stability
theory for hybrid dynamic systems also have been studied
\citep{lazar:heemels:2009,lazar:heemels:teel:2009}. Application and
development  of hybrid theory to prove stability of scheduling models
is a challenging research problem.
\paragraph{Impact of forecast:}
Supply chains are described as ``pull'' systems because the  dynamics
is activated when the customer pulls products from the 
supply chain. As such, the supply chain is sensitive to customer
demands, price signals, etc. MPC theory has been mostly built around
dynamic models trying to ``reject'' external disturbances ( \eg the
nominal case is when there is no disturbance affecting the
system). The impact of demand/ price forecast 
the supply chain steady state; performance, etc. are yet to be studied. 
\paragraph{ Robust terminal conditions:}
In this thesis, we developed  algorithms based on a nominal demand
signal. That is, the stability and convergence guarantees; and
specially, the design of terminal
region/ constraints were based on the nominal demand. In practice, it
is desirable to design the terminal constraint so that we are robust
to some known distribution of demands. Design of such terminal regions
and integration with MPC technology remains an avenue of future work. 
\paragraph{Cooperative game theory:} The cooperative MPC tools have
been developed for process industries to coordinate multiple MPC's in
a single plant. Therefore, it is reasonable to assume that all the
subsystems can share models and objectives with each-other. In a
supply chain, however,the nodes could
be owned by different companies. Hence, we need to study the incentives to cooperate
from a cooperative game theory point of view. 
\paragraph{Implementation:} The ultimate test for any new tool is
practical implementation. An avenue of future research is the
implementation of the tools described in this thesis for a large scale
supply chain with real data. Not only, would such a study help
validate the idea of using MPC for supply chains, it would also help
us uncover new research topics.

