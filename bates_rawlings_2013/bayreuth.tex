\documentclass{beamer}
%<--uncomment this section for a handout 
% \documentclass[handout]{beamer}
% \usepackage{pgfpages}
% \pgfpagesuselayout{4 on 1}[letterpaper,landscape,border shrink=5mm]
%-->

\usetheme{Madison}

%highlighting; to be used mainly when passing through an itemized list
\def\highlight<#1>{%
  \temporal<#1>{\color{black}}{\color{blue}}{\color{black}}}

% Package Lists
\usepackage{colortbl}
\usepackage{url}
\usepackage[authoryear,round,longnamesfirst]{natbib}
% \RequirePackage{underbracket}
% \RequirePackage{overbracket}

% Packages
\usepackage{hyperref} % clickable links
\usepackage{url}
\usepackage{graphicx,color,amsmath,amssymb}
%\usepackage{mathptmx}
%\usepackage{times}
\usepackage{cases}
\usepackage{subfig}
\usepackage{setspace}
\usepackage{booktabs}
\usepackage{balance}
\usepackage{verbatim} % \begin and \end {comment}

% Environments
% \newtheorem{theorem}{Theorem} 
\newtheorem{assumption}{Assumption}
% \newtheorem{definition}{Definition}
% \newtheorem{lemma}{Lemma}
% \newtheorem{proposition}{Proposition}
% \newtheorem{remark}{Remark}
% \newtheorem{corollary}{Corollary}
% \newtheorem{algorithm}{Algorithm}
% \newenvironment{proof}{\noindent {\em Proof}.\ }{\hspace*{\fill}$\Box$\medskip\\}

% Commands
\newcommand{\abs}[1]{\left\lvert #1 \right\rvert}
\newcommand{\norm}[1]{\left\lvert #1 \right\rvert}
\newcommand{\smax}[1]{\left\lVert #1 \right\rVert}
\newcommand{\bbR}{\mathbb{R}}
\newcommand{\bbD}{\mathbb{D}}
\newcommand{\bbX}{\mathbb{X}}
\newcommand{\bbU}{\mathbb{U}}
\newcommand{\bbI}{\mathbb{I}}
\newcommand{\bbZ}{\mathbb{Z}}
\newcommand{\bbP}{\mathbb{P}}
\newcommand{\bbB}{\mathbb{B}}
\newcommand{\useq}{\mathbf{u}}
\newcommand{\bb}{(\cdot)}
\newcommand{\mc}{\mathcal}
\newcommand{\red}[1]{\textcolor{red}{#1}}
\newcommand{\blue}[1]{\textcolor{blue}{#1}}

% \title[Inherent robustness of suboptimal MPC]{On the Inherent
%   Robustness of Suboptimal Model Predictive Control} 

% \author[Rawlings/Pannocchia/Wright/Bates]{James B. Rawlings, Gabriele
%   Pannocchia, Stephen   J. Wright, and Cuyler N. Bates}
% \institute[UW/Pisa]{Department of Chemical and Biological Engineering and\\
% Computer Sciences Department\\[1ex] 
% \pgfuseimage{uw-logo}
% \and
% Department of Civil and Industrial Engineering (DICI),\\
% University of Pisa, Italy}

%\date{2013 SIAM Conference on Control and Its Applications}

\title[Inherent robustness of suboptimal MPC]{On the Inherent
  Robustness of Suboptimal Model Predictive Control} 

\author[Rawlings/Pannocchia/Wright/Bates]{James B. Rawlings, Gabriele
  Pannocchia, Stephen   J. Wright, and Cuyler N. Bates}
\institute[UW/Pisa]{
\begin{minipage}{0.6\textwidth}
\begin{center}
Department of Chemical \& Biological Engineering \\
Computer Sciences Department
\end{center}
\end{minipage}
\begin{minipage}{0.2\textwidth}
\pgfuseimage{uw-logo}
\end{minipage}
\and
Department of Civil and Industrial Engineering, Univ. of Pisa, Italy}

\date{OMPC 2013 SADCO Summer School and Workshop on Optimal and Model
  Predictive Control}
\begin{document}

\begin{frame}
\titlepage
\end{frame}

\section*{Outline}

\begin{frame}{Outline}
\tableofcontents
\end{frame}

%\section{Introduction}

\section{Robustness of stability: overview and literature review} 

\begin{frame}{\parbox[c]{0.8\textwidth}{Overview and objectives} \parbox[c]{0.18\textwidth}{\pgfimage[width=0.15\textwidth]{Overview}}}
\begin{alertblock}{\emph{Nominal} stability of constrained nonlinear systems}
%\emph{Nominal} stability of constrained nonlinear systems:
\begin{equation*}
 x^+ = f(x,u), \qquad x \in \bbX, \quad u \in \bbU
\end{equation*}
in closed-loop with \emph{optimal} MPC, $u = \kappa_N(x)$, can be proved by Lyapunov arguments \citep{mayne:rawlings:rao:scokaert:2000}. 
\end{alertblock}

\begin{exampleblock}{Issues}
\begin{itemize}
\item Optimal control problem must be solved \red{exactly} at each decision time
\item Can \red{perturbations} (e.g. process disturbances or measurement noise) destroy stability?
\end{itemize}
\end{exampleblock}

\begin{block}{Objectives}
\begin{itemize}
\item \blue{General} and \red{implementable} suboptimal MPC
\item \red{Inherent} robustness questions (recursive feasibility and stability)
\end{itemize}
\end{block}

\end{frame}

\begin{frame}{Robustness and MPC: literature review (1/2)}
%\hfill \pgfimage[width=0.08\textwidth]{big-fat-greek-wedding}}

\begin{alertblock}{Robust MPC synthesis}
\begin{itemize}
\item Perturbations are considered \red{explicitly} by design (e.g.
\citep{bemporad:morari:1999}, \citep[Ch. 3]{rawlings:mayne:2009} and refs. therein).

\item Robust MPC formulations (usually numerically \blue{tractable} only for \blue{linear systems})
tend to be \red{conservative} to preserve recursive feasibility
\end{itemize}
\end{alertblock}

\begin{exampleblock}{Inherent robustness of MPC}
\begin{itemize}
\item A \red{difficult} problem that received less attention \citep{denicolao:magni:scattolini:1996b,scokaert:rawlings:meadows:1997,grimm:messina:tuna:teel:2004}

\item \cite{grimm:messina:tuna:teel:2004} showed examples of nonlinear systems for which \red{arbitrarily small} disturbances \red{destroy} asymptotic stability
\end{itemize}
\end{exampleblock}
\end{frame}

\begin{frame}{Robustness and MPC: literature review (2/2)}
\begin{alertblock}{Recursive feasibility and restricted constraints}
\cite{grimm:messina:tuna:teel:2007} presented conditions to ensure \red{recursive feasibility} by adopting a \red{constraint tightening} approach \citep{marruedo:alamo:camacho:2002a}.
\end{alertblock}

\begin{exampleblock}{ISS and suboptimal MPC}
\begin{itemize}
\item Inherent robustness of \red{suboptimal} MPC was first addressed in \citep{lazar:heemels:2009}
\item They showed Input-to-State-Stability (ISS) of the equilibrium provided that a \red{sub-optimality} margin is \red{guaranteed}
\end{itemize}
\end{exampleblock}
\end{frame}



\subsection{Optimal and Suboptimal MPC (nominal)}

\begin{frame}{The basic nonlinear, constrained MPC problem}

\begin{itemize}
\item The (nonlinear) system model is
\begin{equation}\label{eq:nominal}
x^+ = f(x, u)
\end{equation}

\item Only the input is subject to constraints (state constraints are soft)
\begin{equation*}
u(k)\in \bbU \qquad \text{for all } k\in\bbI_{\geq 0}
\end{equation*}

\item Given an integer $N$ (referred to as the finite horizon), and an
  input sequence $\useq$ of length $N$,  
$\useq = \{ u(0), u(1), \ldots, u(N-1) \}$

\item Let $\phi(k;x,\useq)$
denote the solution of \eqref{eq:nominal} 
at time $k$ for a given initial state $x(0) = x$.

\item Terminal state constraint (and penalty)
\begin{equation*}
\phi(N; x, \useq) \in \bbX_f
\end{equation*}

\end{itemize}

\end{frame}

\begin{frame}{Feasible sets}

\begin{itemize}

\item The set of feasible initial states and associated
control sequences
\begin{equation*}
\bbZ_N = \{ (x,\useq) \mid u(k) \in \bbU , %\, \phi(k; x,\useq) \in \bbX \text{ for all } k\in\bbI_
%{0:N-1} ,  \\
\text{ and } \phi(N; x,\useq) \in \bbX_f \}
\end{equation*}

and $\bbX_f$ is the feasible terminal set.

\item The set of feasible initial states is
\begin{equation}
\mc{X}_N = \{ x \in \bbR^n \mid \exists \useq \in \bbU^N \text{ such that } (x,\useq) \in \bbZ_N \}
\label{eq:feasibleset}
\end{equation}

\item For each $x\in\mc{X}_N$, the corresponding set of feasible input
sequences is
\begin{equation*}
 \mc{U}_N (x) = \{ \useq \mid (x,\useq) \in \bbZ_N \} 
\end{equation*}
\end{itemize}
\end{frame}

\begin{frame}{Cost function and control problem}
\begin{itemize}
\item For any state $x \in \bbR^n$ and input sequence $\useq \in \bbU^N$, we define% 

\begin{equation*}
V_N(x,\useq) = \sum_{k=0}^{N-1} \ell (\phi(k;x,\useq),u(k)) + V_f (\phi(N;x,\useq))
\end{equation*}

\item $\ell(x,u)$ is the stage cost; $V_f(x(N))$ is the terminal cost

\item Consider the finite horizon optimal control problem
\begin{equation*}
\bbP_N (x): \qquad \min_{\useq \in \mc{U}_N} V_N(x,\useq) 
\end{equation*}
\end{itemize}
\end{frame}


\begin{frame}{Suboptimal MPC}

\begin{itemize}
\item May not be able to solve online $\bbP_N(x)$ \alert{exactly}, so
  we consider using \alert{any} 
suboptimal algorithm having the following properties.

\item Let $\useq\in\mc{U}_N(x)$ denote the (suboptimal) control sequence for
the initial state $x$, and let $\tilde{\useq}$ denote a \alert{warm
start} for the successor initial state $x^+ = f(x,u(0; x))$, obtained
from $(x,\useq)$ by
\begin{equation}\label{eq:warm}
 \tilde{\useq} := \{u(1; x),u(2; x), \ldots, u(N-1; x), u_+\}
\end{equation}

\item  $u_+ \in\bbU$ is any input that satisfies the
invariance conditions of Assumption~\ref{as:terminalstability} for  
$x = \phi(N;x, \useq) \in \bbX_f$, i.e., $u_+ \in \kappa_f(\phi(N;x, \useq))$. 
\end{itemize}
\end{frame}

\begin{frame}{Suboptimal MPC}

\begin{itemize}

\item The warm start satisfies $\tilde{\useq} \in \mc{U}_N
(x^+)$. 

\item The suboptimal input sequence for any given $x^+ \in
\mc{X}_N$ is defined as \alert{any} $\useq^+ \in  \bbU^N$ that 
satisfies:
\begin{subequations}\label{eq:sub}
\begin{align}
 \useq^+ &\in \mc{U}_N(x^+) & & \label{eq:sub:feas} \\
 V_N(x^+,\useq^+) &\leq V_N(x^+,\tilde{\useq}) & &  \label{eq:sub:cost} \\
% |\useq^+| \leq c |x^+| \text{ for all } x^+ \in r\bbB \label{eq:sub:ball} 
 V_N(x^+,\useq^+) &\leq V_f(x^+) & & \text{when } x^+ \in r\bbB \label{eq:sub:ball}
\end{align}
\end{subequations}
in which $r$ is a positive scalar sufficiently small that $r\bbB
\subseteq \bbX_f$.  

% \item Notice that constraint \eqref{eq:sub:ball} is
% required to hold only if $x^+ \in r\bbB$, and it implies that
% $|\useq^+| \to 0$ as $|x^+|\to 0$.

\item Condition \eqref{eq:sub:cost} ensures that the computed suboptimal
cost is no larger than that of the warm start.
\end{itemize}
\end{frame}

% \begin{frame}{Suboptimal MPC}

% \begin{itemize}
% \item For any $x^+\in\mc{X}_N$, there exists a
%   $\useq^+\in\mc{U}_N(x^+)$ satisfying all conditions \eqref{eq:sub}
%   for all $\tilde{\useq} \in \mc{U}_N (x^+)$. 
% %\end{cor}

% \item We now observe that $\useq^+$ is a set-valued map of the state $x^+$, and so is 
% the associated first component $u(0; x^+)$.

% \item If we, again, denote the latter map as $\kappa_N(\cdot)$, we can write the evolution of the closed-loop 
% system as 
% the following difference inclusion:
% \begin{equation}
% x^+ \in \{ f(x,u) \mid u \in \kappa_N (x) \}
% \label{eq:nonext}
% \end{equation}
% \end{itemize}

% \end{frame}


% \begin{frame}{Control law and closed-loop system}

% \begin{itemize}
% % \item The control law is
% % \begin{equation*}
% % \kappa_N(x) = u^0(0; x)
% % \end{equation*}
% % For suboptimal MPC (or nonunique optimal MPC),  $\kappa_N\bb$ is a point-to-set map

% \item Closed-loop system
% \begin{equation*} 
% x^+ = f(x, \kappa_N(x))\qquad \text{difference equation} 
% \end{equation*}
% \begin{equation*}
% x^+ \in f(x, \kappa_N(x)) \qquad \text{difference inclusion}
% \end{equation*}

% \item Nominal closed-loop stability. 
% Both optimal and suboptimal systems are recursively feasible, and the
% origin is asymptotically stable. 

% \item The  region of attraction is the feasible set $\mc{X}_N$.

% \item The cost function $V_N(x,\useq)$ is a Lyapunov function for
%   the system in the extended state $z=(x,\tilde{\useq})$.

% \end{itemize}
% \end{frame}


\begin{frame}{Extended state}

\begin{itemize}
\item Since the suboptimal algorithm requires a measured state and warm start
pair, we define the extended state by 
\begin{equation*}
z = (x,\tilde{\mathbf{u}})
\end{equation*}

\item The procedure to generate the next warm start is
\begin{equation}
\tilde{\mathbf{u}} = \{u(1),u(2),\dots,u(N-1),\kappa_f(\phi(N;x,\mathbf{u}))\}
\label{eq:pwarm}
\end{equation}

\item The extended system evolves as
\begin{multline}
z^+ \in H(z) = \{(x^+,\tilde{\mathbf{u}}^+) \ | \ x^+ =
f(x,u(0)), \\
\tilde{\mathbf{u}}^+ = \zeta(x,\mathbf{u}), \mathbf{u} \in \mathcal{U}_r(z)\}
%\label{eq:perturbedinc}
\end{multline}
where $\zeta(\cdot)$ is the mapping corresponding to \eqref{eq:pwarm} 
\end{itemize}

\end{frame}

% \begin{frame}{Inherent robustness of the nominal controller}

% \begin{itemize}
% \item Consider a process disturbance $d$, $x^+=f(x,\kappa_N(x)) + d$
% \item A measurement disturbance $x_m= x+e$ 
% \item Nominal controller with disturbance
% \begin{align*}
% x^+ &\in f(x, \kappa_N(x_m)) +d  \\
% x^+ & \in f(x, \kappa_N(x + e )) +d  \\
% x^+ &\in F(x, w) \qquad w=(d,e)
% \end{align*}

% Robust stability; is the system $x^+\in F(x,w)$ input-to-state stable
% considering $w=(d,e)$ as the input.
% \end{itemize}
% \end{frame}

\begin{frame}{Stability definitions}

\begin{definition}[]\mbox{}
\begin{itemize}
\item   A function $\sigma:\bbR_{\geq 0} \rightarrow \bbR_{\geq 0}$ belongs to
  class $\mc{K}$ if it is continuous, zero at zero, and strictly
  increasing; 
% \item $\sigma: \bbR_{\geq 0} \rightarrow \bbR_{\geq 0}$ belongs to
%   class $\mc{K}_\infty$ if it is a class $\mc{K}$ and unbounded
%   ($\sigma(s) \rightarrow \infty$ as $s \rightarrow \infty$).  
\item A function
  $\beta:\bbR_{\geq 0}\times \bbI_{\geq 0} \rightarrow \bbR_{\geq 0}$
  belongs to class $\mc{KL}$ if it is continuous and if, for each
  $t \ge 0$, $\beta(\cdot,t)$ is a class $\mc{K}$ function and for
  each $s\ge 0$, $\beta(s,\cdot)$ is nonincreasing and satisfies
  $\lim_{t \rightarrow \infty}\beta(s,t)=0$.  
\end{itemize}
\end{definition}

\begin{definition}[Asymptotic stability]
\label{def:as}
We say the origin of the difference inclusion $z^+ \in H(z)$ is 
asymptotically stable on the positive invariant set $\mathcal{Z}$ if there exists a function
$\beta \in \mathcal{KL}$ such that for any $z \in \mathcal{Z}$, all solutions
$\psi(k;z)$ satisfy
\begin{equation*}
%\psi(k;z) \in \mathcal{Z} \quad \text{and} \quad \abs{\psi(k;z)} \leq 
\abs{\psi(k;z)} \leq \beta(\abs{z},k) \quad \forall k \in \mathbb{I}_{\geq 0}
\end{equation*}
\end{definition}
\end{frame}

\begin{frame}{Lyapunov function}
\begin{definition}[Lyapunov function]
\label{def:lyap}
V is a Lyapunov function on the positive invariant set $\mathcal{Z}$ for the difference 
inclusion $z^+ \in H(z)$ if there exists functions $\alpha_1,\alpha_2,\alpha_3
\in \mathcal{K}_\infty$ such that for all $z \in \mathcal{Z}$
\begin{align*}
\alpha_1(\abs{z}) &\leq V(z) \leq \alpha_2(\abs{z}) \\
\max_{z^+ \in H(z)} V(z^+) &\leq V(z) - \alpha_3(\abs{z})
\end{align*}
\end{definition}
\end{frame}


\begin{frame}{Basic assumptions for MPC}
\begin{assumption}[1---Continuity of system and stage cost]
\label{as:continuity}
The model $f:\mathbb{R}^n \times \mathbb{R}^m \rightarrow \mathbb{R}^n$, 
stage cost $\ell: \mathbb{R}^n \times \mathbb{R}^m \rightarrow \mathbb{R}_
{\geq 0}$ and terminal cost $V_f:\mathbb{R}^n \rightarrow \mathbb{R}_{\geq 0}$
are continuous. Furthermore, for some steady state $(x_s,u_s)$, we have $\ell(x_s,u_s)=0$
and $V_f(x_s) = 0$.
\end{assumption}
We assume without loss of generality that 
$(x_s,u_s) = (0,0)$.
\begin{assumption}[2---Stage cost bound]
\label{as:pdstagecost}
There exists a function $\alpha_\ell \in \mathcal{K}_\infty$ such that
\begin{equation*}
%\label{eq:pdstagecost}
\alpha_\ell(\abs{(x,u)}) \leq \ell(x,u) \qquad \forall (x,u) \in \mathbb{Z}
\end{equation*}
\end{assumption}
\end{frame}

% \begin{frame}{}
%  we can write the control law 
% \eqref{eq:mpclaw} and closed-loop system \eqref{eq:closed}
% \begin{align}
% u &= \kappa_N(x) := u^0(0;x) \label{eq:mpclaw} \\
% x^+ &= f(x,\kappa_N(x)) \label{eq:closed}
% \end{align}
% \end{frame}


\begin{frame}{Constraint sets, terminal control law, and terminal region}
\begin{assumption}[3---Properties of constraint set]
\label{as:constraints}
The set $\mathbb{U}$ is compact and contains the origin. 
The set $\mathbb{X}_f$ is defined by  $\mathbb{X}_f= \text{lev}_\alpha V_f = \{x \in \mathbb{R}^n \ | \ 
V_f(x) \leq \alpha\}$, with $\alpha > 0$.
\end{assumption}

\begin{assumption}[4---Stability assumption]
\label{as:terminalstability}
There exists a terminal control law $\kappa_f:\mathbb{X}_f \rightarrow
\mathbb{U}$ such that
\begin{align*}
%\label{eq:invariant}
f(x,\kappa_f(x)) &\in \mathbb{X}_f \qquad &\forall x \in \mathbb{X}_f\\
%\label{eq:terminalstability}
V_f(f(x,\kappa_f(x))) &\leq V_f(x)-\ell(x,\kappa_f(x))  
&\forall x \in \mathbb{X}_f
\end{align*}
\end{assumption}

\end{frame}


\begin{frame}{Nominal stability result}
\begin{theorem}[Nominal Asymptotic Stability of Suboptimal MPC]
\label{th:nominalstability}
Under Assumptions \ref{as:continuity} -- \ref{as:terminalstability}, there exists
$\beta(\cdot) \in \mathcal{KL}$ such that 
\begin{equation*}
\abs{\phi(k;z)} \leq \beta(\abs{x},k)
\end{equation*}
for any initial extended state $z = (x,\tilde{\mathbf{u}}) \in \mathcal{Z}_r$.
% There is no state evolution equation in x alone, therefore you can't claim anything is ``asymptotically
% stable'' in terms of x alone because a.s. applies only to evolution equations
\end{theorem}

Outline of proof: Establish that $V_N(x,\tilde{\useq})$ is a Lyapunov
function for the extended system $z^+ \in H(z)$.

\end{frame}

\subsection{Disturbances and Robustness}

\begin{frame}{So far so good; now is the stability robust?}

\begin{itemize}
\item Consider disturbances to the process ($d$) and state measurement
($e$). 
\begin{alignat*}{2}
x^+ &= f(x, \kappa_N(x)) &\qquad &\text{nominal system} \\
x^+ &= f(x, \kappa_N(x+e)) + d &\qquad &\text{nominal controller with disturbances}
\end{alignat*}

\item Study of \textit{inherent} robustness motivated by \cite{teel:2004} who
showed examples for which \alert{arbitrarily small perturbations} can
\alert{destabilize} the nominally stabilizing controller.

\item \cite{kellet:teel:2004a} establish that for $x^+=f(x)$ with $f$
  locally bounded, a compact invariant set is robustly asymptotically
  stable if and only if the system admits a \alert{continuous} global
  Lyapunov function.
\end{itemize}
\end{frame}

\begin{frame}{Effect of disturbances}

\begin{itemize}

\item The closed-loop state and measurement evolutions are
\begin{align*}
x^+ &= f(x,\kappa_N(x+e)) + d \\
x_m^+ &= f(x_m-e, \kappa_N(x_m)) + d + e^+
\end{align*}
where $x_m = x+e$ is the measured state and $d$ is the additive
process disturbance. 

\item Note that the suboptimal control 
law is now calculated based on the measured state $u =
\kappa_N(x_m,\tilde{\mathbf{u}})$. 

\item The results are simpler to state using the measurement system evolution

\item The perturbed extended system then evolves as
\begin{multline}
z_m^+ \in H_{ed}(z_m) = \{(x_m^+,\tilde{\mathbf{u}}^+) \ | \ x_m^+ =
f(x_m-e,u(0))+d+e^+, \\
\tilde{\mathbf{u}}^+ = \zeta(x_m,\mathbf{u}), \mathbf{u} \in \mathcal{U}_r(z_m)\}
\label{eq:perturbedinc}
\end{multline}
where $\zeta(\cdot)$ is the mapping corresponding to \eqref{eq:pwarm} 

\end{itemize}
\end{frame}


\begin{frame}{Desired robustness property}
\begin{definition}[Robust asymptotic stability \cite{teel:2004}]
\label{def:ras}
The origin of the closed-loop system \eqref{eq:perturbedinc} is
robustly asymptotically stable (RAS) on $\mathcal{C}$
if there exists $\delta > 0$, 
$\beta \in \mathcal{KL}$, and $\sigma_d,\sigma_e \in \mathcal{K}$ such that
for each $x_m \in \mathcal{C}$ and for all $\tilde{\mathbf{u}} \in
\tilde{\mathcal{U}}_r(x_m)$,  and for all disturbance sequences
$\mathbf{d}$ and $\mathbf{e}$ 
satisfying $\smax{\mathbf{d}},\smax{\mathbf{e}} \leq \delta$, 
we have that
\begin{equation}
\label{eq:ras}
%\abs{\phi_{ed}(k;x,\tilde{\mathbf{u}},\mathbf{d}_i,\mathbf{e}_i)} \leq \beta(\abs{x},k) + \sigma_d(\norm{
%\mathbf{d}_{i-1}})+ \sigma_e(\norm{\mathbf{e}_i})
\abs{\phi_{ed}(k;x_m,\tilde{\mathbf{u}})} \leq \beta(\abs{x_m},k) + \sigma_d(\smax{
\mathbf{d}_{k-1}})+ \sigma_e(\smax{\mathbf{e}_k})
\end{equation}
for all $k \in \mathbb{I}_{\geq 0}$.
\end{definition}
\end{frame}

\begin{frame}{Behavior with and without disturbances}

\begin{columns}[t]
\onslide<1->
\begin{column}{0.45\textwidth}
\centerline{\resizebox{0.7\textwidth}{!}{\input{nomcl}}}
\begin{block}{Nominal System}
\begin{align*}
x^+ &= f(x, u)\\
u   &= \kappa_N(x)
\end{align*}
\end{block}
\end{column}

\onslide<2->
\begin{column}{0.45\textwidth}
\centerline{\resizebox{0.7\textwidth}{!}{\input{robcl}}}
\begin{block}{System with Disturbance}
\begin{align*}
x^+ &= f(x, u) + d\\
u   &= \kappa_N(x + e)
\end{align*}
$d$ is the process disturbance\\
$e$ is the measurement disturbance
\end{block}
\end{column}
\end{columns}
\end{frame}


\section{New Results on Inherent Robustness}

\begin{frame}{New result}
\begin{theorem}[Robust Asymptotic Stability of Suboptimal MPC]
\label{th:mainiss}
Under Assumptions \ref{as:continuity}--\ref{as:terminalstability}
the origin of the perturbed closed-loop system 
\eqref{eq:perturbedinc} is RAS on any compact subset
of $\mathcal{X}_N$
\end{theorem}
\begin{itemize}
\item This result is an improvement on
  \cite{pannocchia:rawlings:wright:2011}. 
\item The nominal controller is completely unchanged here.
\item For the optimal case, we have not (explicitly) assumed anything
  about continuity of $V_N^0(x)$ here.
\item \cite{yu:reble:chen:allgower:2011} first to point out continuity
  of $V_N^0(x)$ not required.
\item See also \cite{lazar:heemels:2009} for 
  robustness of suboptimal MPC on hybrid systems.
\end{itemize}


\end{frame}

\section{Analysis of a Troublesome Example}

\begin{frame}{A troublesome example}

\begin{align*}
x^+ &= f(x,u) \\
\begin{bmatrix}  x_1 \\ x_2 \end{bmatrix}^+ &= 
\begin{bmatrix}  x_1 \\ x_2 \end{bmatrix} + 
\begin{bmatrix}  u \\ u^3 \end{bmatrix}
\end{align*}

\begin{itemize}
\item Two state, single input example. The origin is the desired
  steady state: $u=0$ at $x=0$.
\item Cannot be stabilized with continuous feedback $u=\kappa(x)$. 

\item Because $(u,u^3)$ have the same sign, must use negative $u$ to
stabilize first quadrant. 
\item Must use positive $u$ to stabilize third
quadrant.  

\item But $u$ cannot pass through zero or that point is a closed-loop
steady state. 
\item Therefore \alert{discontinuous} feedback.
\end{itemize}
\end{frame}

\begin{frame}{And its troubled history}
\begin{itemize}
\item Introduced by \cite{meadows:henson:eaton:rawlings:1995}
  to show MPC control law and optimal cost can be discontinuous. 

\item Based on a CT example by \cite{coron:1990}.

\item \cite{grimm:messina:tuna:teel:2005} established robustness for
  MPC with horizon $N \geq 4$ with a terminal cost and no terminal region constraint.
\end{itemize}
\end{frame}

\begin{frame}{MPC with terminal equality constraint}

\begin{itemize}
\item Because we do \alert{not} know even a \alert{local controller}, we try a terminal
constraint $x(N)=0$ in the MPC controller. 
\item For what initial $x$ is this constraint feasible?
\begin{align*}
(x_1(1), x_2(1)) &= (x_1(0), x_2(0)) + (u_0, u^3_0) \\
(x_1(2), x_2(2)) &= (x_1(1),x_2(1)) + (u_1, u_1^3) \\
(x_1(3), x_2(3)) &= (x_1(2), x_2(2)) + (u_2, u_2^3)
\end{align*}

\item  For $N=1$, the feasible set $\mathcal{X}_1$ is only the line $x_2 = x_1^3$. 
\item For $N=2$, to have real roots $u_0, u_1$, we require 
$−3x_1^4 + 12x_1x_2 \geq 0$ which defines $\mathcal{X}_2$
\item For $N=3$, we have $\mathcal{X}_3$ is all of $\bbR^2$. 
\item So the shortest horizon that can globally stabilize the system
  is $N=3$.
\end{itemize}
\end{frame}


\begin{frame}{Feasibility sets $\mathcal{X}_1$, $\mathcal{X}_2$, and
  $\mathcal{X}_3$}
\centerline{\resizebox{0.8\textwidth}{!}{\input{feasibility_set}}}

\end{frame}

\begin{frame}{Structure of Feasibility Sets}


\centerline{\resizebox{0.6\textwidth}{!}{\input{nested_sets}}}

\begin{block}{}
\begin{itemize}
\item The feasibility sets are nested: $\mathcal{X}_N \supseteq
  \mathcal{X}_{N-1} \supseteq \mathcal{X}_{N-2} \cdots \supseteq
  \mathbb{X}_f$
\item The set $\mathcal{X}_N$ is forward invariant. Important for
  recursive feasibility of controller.
\item The set $\mathcal{X}_{N-1}$ is also forward invariant!
\item The sets $\mathcal{X}_{N-2}, \mathcal{X}_{N-3},\ldots,
  \mathbb{X}_f$ are not necessarily forward invariant.
\end{itemize}
\end{block}
\end{frame}

\begin{frame}{Optimal MPC with $N=3$}

The control constraint set $\mathcal{U}_N(x)$  and optimal
control $\kappa_N(x)$ for $x$ on the unit
circle is given by the following \citep[p. 105]{rawlings:mayne:2009} 

\centerline{\resizebox{0.8\textwidth}{!}{\input{feasible}}}

\end{frame}


\begin{frame}{Optimal cost function with $N=3$}

The discontinuity in the optimal cost for $x$ on  the unit circle

\centerline{\resizebox{0.8\textwidth}{!}{\input{circle_phi}}}

\end{frame}

\begin{frame}{Where is $V_N^0$ discontinuous?}

\begin{itemize}
\item From the theory of cubic equations, we know that changes in the number
of real roots to the equation $au^3+bu^2+cu+d=0$ are determined by the
sign of the discriminant 
\begin{equation*}
\Delta = 18 abcd - 4b^3d + b^2c^2 -4ac^3 -27 a^2 d^2
\end{equation*}

\item For our system, substituting $a=3$, $b=-3x_1$, $c=-3x_1^2$, and
$d=-x_1^3+4x_2$ into the expression for the discriminant and factoring
gives 
\begin{align*}
\Delta &= 432 (-x_1^6 + 10 x_1^3x_2 - 9 x_2^2)\\
       &= -432 (x_1^3 - 9 x_2)(x_1^3 -x_2)
\end{align*}

\item Setting $\Delta=0$ gives two $(x_1, x_2)$ lines at which the number
of real roots changes from one to three. The line that generates a discontinuity in
$V_3^0(x)$ corresponds to setting the first factor to zero giving
\begin{equation*}
x_2 = (1/9) x_1^3
\end{equation*}
\end{itemize}
\end{frame}


\begin{frame}{Where is $V_N^0$ discontinuous? Set $\bbD$.}

%\item The set $\bbD$, points of discontinuity in cost $V_3^0(\cdot)$,
%  and invariant set $\mathcal{X}_2$. 

\centerline{\resizebox{0.75\textwidth}{!}{\input{discontinuities}}}
\begin{itemize}

\item Note that invariant set $\mathcal{X}_2$ and discontinuity set
  $\bbD$   do not intersect (the origin is not an element of
  $\bbD$). But they approach each other at the origin.
\end{itemize}
\end{frame}

\begin{frame}{Robustness result for troublesome example}

\begin{block}{}
\begin{itemize}
\item RGAS follows because the nominal invariant set and set of
  discontinuities of $V_N^0(x)$ do not intersect

\item Outline of proof: For $x$ large, $\mathcal{X}_2$ is far from
  $\bbD$, and (continuous) Lyapunov function argument applies. 

\item For $x$ small, cost \alert{can increase} due to interaction of
  discontinuity in $V_N^0(x)$ and nonzero disturbance. But cost
  increase is small because $x$ and hence $V_N^0(x)$ are  small.

\item These two together give an asymptotic robust invariant set that shrinks to
  zero with the size of disturbances.
\end{itemize}
\end{block}

\end{frame}

\section{Conclusion}
\begin{frame}{Conclusion}

\begin{block}{}
\begin{itemize}
\item \alert{Suboptimal} MPC with a well chosen warm start has the same
  inherent robustness properties as \alert{optimal} MPC. \pause
\item Robust stability of nominal MPC extended to compact subset of
  feasible set ($\mathcal{X}_N$) with \textit{no changes} to the MPC
  controller. \pause 
  Recall:   no state constraints. \pause
\item The control law and optimal cost may be \alert{discontinuous} on
  $\mathcal{X}_N$. \pause 
\item Still no general analysis tools for discontinuous optimal cost
  and terminal \alert{equality constraint}.  But the example shows robustness
  for even this case. \pause Exploits \alert{empty} intersection of invariant set
  $\mathcal{X}_{N-1}$ (not $\mathcal{X}_N$) and discontinuous set. \pause
\item Future work: extend robustness results to \alert{economic} MPC.
\end{itemize}
\end{block}

\end{frame}

\begin{frame}[allowframebreaks]{Further reading}

\renewcommand{\refname}{}
\scriptsize
\bibliographystyle{abbrvnat}
\bibliography{abbreviations,articles,books,proceedings,unpub}
%\bibliography{siam}
\end{frame}

\end{document}
