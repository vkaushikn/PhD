\documentclass{article}
%\documentclass[letterpaper, 10pt, conference]{ieeeconf}
%\IEEEoverridecommandlockouts                          
%\overrideIEEEmargins
%\bibliographystyle{IEEEtran}

% Packages
\usepackage{hyperref} % clickable links
\usepackage{url}
%\usepackage{rxn}
\usepackage{natbib}
\usepackage{graphicx,color,amsmath,amssymb}
%\usepackage{mathptmx}
%\usepackage{times}
\usepackage{cases}
\usepackage{subfig}
\usepackage{setspace}
\usepackage{booktabs}
\usepackage{balance}
\usepackage{verbatim} % \begin and \end {comment}

% Environments
\newtheorem{theorem}{Theorem} 
\newtheorem{assumption}{Assumption}
\newtheorem{definition}{Definition}
\newtheorem{lemma}{Lemma}
\newtheorem{proposition}{Proposition}
\newtheorem{remark}{Remark}
\newtheorem{corollary}{Corollary}
\newtheorem{algorithm}{Algorithm}
\newenvironment{proof}{\noindent {\em Proof}.\ }{\hspace*{\fill}$\Box$\medskip\\}

% Commands
\newcommand{\abs}[1]{\left\lvert #1 \right\rvert}
\newcommand{\norm}[1]{\left\lVert #1 \right\rVert}
\newcommand{\lev}{\text{lev}}
\newcommand{\bbR}{\mathbb{R}}
\newcommand{\bbX}{\mathbb{X}}
\newcommand{\bbD}{\mathbb{D}}
\newcommand{\eqbyd}{:=}
\newcommand{\bbU}{\mathbb{U}}
\newcommand{\useq}{\mathbf{u}}


\title{On the inherent robustness of suboptimal MPC}

\author{Cuyler N. Bates and James B. Rawlings
  \thanks{C. N. Bates and J. B. Rawlings are with Dept. of Chemical and
    Biological Engineering, Univ. of Wisconsin, Madison, WI 53706, USA
    \texttt{\footnotesize (cnbates@wisc.edu,rawlings@engr.wisc.edu)}}} 

\begin{document}

\maketitle

Notation. The symbols $\mathbb{I}_{\geq 0}$ and $\mathbb{R}_{\geq 0}$ denote
the nonnegative integers and reals, respectively. The symbol $\mathbb{I}_{
0:N}$ denotes the set $\{0,1,\dots,N\}$.
The symbol $\abs{\cdot}$ denotes the Euclidean norm and 
$\mathbb{B}$ denotes the closed ball of radius 1 centered at the origin.
%The relation $A \subseteq B$ denotes $A$ is a subset of $B$ while the 
%relation $A \subset B$ denotes $A$ is a proper subset of $B$. 
Bold symbols, e.g., $\mathbf{d}$, denote sequences, $d(k)$ denotes the element of $\mathbf{d}$ at time $k \in \mathbb{I}_{\geq{0}}$ and 
$\mathbf{d}_k$ denotes the truncation of the sequence up to time $k$, i.e.
$\mathbf{d}_k = \{d(0),d(1),\dots,d(k)\}$.
We define $\norm{\mathbf{d}} = \sup_{k \geq 0}\abs{d(k)}$. 

%Given time $k$, initial state $x$ and input sequence $\mathbf{u}$, $\phi(k;x,\mathbf{u})$ denotes 
%the open-loop state solution to the system \eqref{eq:system}.
%Given time $k$, initial state $x$ and initial warm start $\tilde{\mathbf{u}}$, $\psi(k;x,\tilde{\mathbf{u}})$ denotes 
%an arbitrary closed-loop extended state solution
%to the difference inclusion \eqref{eq:nominc}
%and $\phi(k;x,\tilde{\mathbf{u}})$ denotes the resulting state component trajectory. 
%Given time $k$, initial state $x$, initial warm start $\tilde{\mathbf{u}}$ and disturbance and measurement
%error sequences $\mathbf{d}$ and $\mathbf{e}$, $\psi_{ed}(k;x,\tilde{\mathbf{u}})$ denotes 
%an arbitrary closed-loop extended state solution
%to the perturbed difference inclusion \eqref{eq:perturbedinc}
%and $\phi_{ed}(k;x,\tilde{\mathbf{u}})$ denotes the resulting state component trajectory. 

\section{Basic definitions and assumptions}
\subsection{MPC problem}
\label{sec:mpc}
We consider autonomous discrete-time systems without state constraints
\begin{equation}
\label{eq:system}
x^+=f(x,u), \quad u \in \mathbb{U}, \quad x \in \mathbb{R}^n, \quad
(x,u) \in \mathbb{Z} = \mathbb{R}^n \times \mathbb{U} 
\end{equation}
%\begin{equation*}
%u \in \mathbb{U}, \quad x \in \mathbb{R}^n, \quad (x,u) \in \mathbb{Z} = \mathbb{R}^n \times \mathbb{U}
%\end{equation*}
\begin{comment}
The state and input are 
subject to the constraints
\begin{equation*}
%\label{eq:constraints}
(x,u) \in \mathbb{X} \times \mathbb{U}
\end{equation*}
\end{comment}
Given time $k$, initial state $x$ and input sequence $\mathbf{u}$, $\phi(k;x,\mathbf{u})$ denotes 
the open-loop state solution to the system \eqref{eq:system}.
For model predictive control with a horizon of $N$, initial condition $x$ 
and a terminal constraint, we define the set of admissible 
$(x,\mathbf{u})$ pairs \eqref{eq:admissiblepairs}, admissible inputs
\eqref{eq:admissibleinputs}, admissible states
\eqref{eq:admissiblestates} and objective function \eqref{eq:objective} by
\begin{align}
\label{eq:admissiblepairs}
\mathcal{Z}_N &= \{ (x,\mathbf{u}) \mid
x^+=f(x,u), \; (x(k),u(k)) \in \mathbb{Z} \; \forall k \in \mathbb{I}_{0:N-1}, \nonumber \\
              &\phantom{= \{ (x,\mathbf{u}) \mid x} x(N) \in \mathbb{X}_f, \; x(0)=x \} \\
\label{eq:admissibleinputs}
\mathcal{U}_N(x) &= \{\mathbf{u} \mid (x,\mathbf{u}) \in \mathcal{Z}_N \}
\\
\label{eq:admissiblestates}
\mathcal{X}_N &= \{x \mid 
\exists \, \mathbf{u} \text{ such that  } (x,\mathbf{u}) \in \mathcal{Z}_N \}
\\
\label{eq:objective}
 V_N ( x, \mathbf{u} ) &= \sum_{k=0}^{N-1} \ell ( x(k),u(k) ) + V_f(x(N))
\end{align}
\begin{assumption}[Continuity of system and cost]
\label{as:continuity}
The model $f:\mathbb{R}^n \times \mathbb{R}^m \rightarrow \mathbb{R}^n$, 
stage cost $\ell: \mathbb{R}^n \times \mathbb{R}^m \rightarrow \mathbb{R}_
{\geq 0}$ and terminal cost $V_f:\mathbb{R}^n \rightarrow \mathbb{R}_{\geq 0}$
are continuous. Furthermore, we have $f(0,0) = 0$, $\ell(0,0)=0$
and $V_f(0) = 0$.
\end{assumption}
\begin{assumption}[Properties of constraint set]
\label{as:constraints}
The set $\mathbb{U}$ is compact and contains the origin. 
%The sets $\mathbb{X}$ and $\mathbb{X}_f$ are closed, contain the origin in their interiors and $\mathbb{X}_f \subset \mathbb{X}$.
%The set $\mathbb{X}_f$ is closed and contains the origin in its interior.
% We define X_f to be a level set of V_f
The set $\mathbb{X}_f$ is defined by  $\mathbb{X}_f= \lev_\alpha V_f = \{x \in \mathbb{R}^n \ | \ 
V_f(x) \leq \alpha\}$, with $\alpha > 0$.
\end{assumption}
\begin{assumption}[Stability assumption]
\label{as:terminalstability}
There exists a terminal control law $\kappa_f:\mathbb{X}_f \rightarrow
\mathbb{U}$ such that for all $x \in \mathbb{X}_f$, we have that $f(x,\kappa_f(x)) \in \mathbb{X}_f$ and 
\begin{equation*}
V_f(f(x,\kappa_f(x))) \leq V_f(x)-\ell(x,\kappa_f(x))
\end{equation*}
\end{assumption}
\begin{assumption}[Cost function bound]
\label{as:pdstagecost}
There exists $\alpha_\ell \in \mathcal{K}_\infty$ such that
\begin{equation*}
%\label{eq:pdstagecost}
\ell(x,u) \geq \alpha_\ell(\abs{(x,u)}) \quad \quad \forall (x,u) \in \mathbb{Z}
\end{equation*}
\end{assumption}

%From Assumption \ref{as:continuity} and Lemma \ref{lem:overbound}, there exists $\alpha_V,\alpha_f \in \mathcal{K}_\infty$
%such that
%\begin{align}
%\label{eq:vnbound}
%V_N(x,\mathbf{u}) &\leq \alpha_V(\abs{(x,\mathbf{u})}) &\forall (x,\mathbf{u}) \in \mathbb{R}^{n+mN} \\
%\label{eq:vfbound}
%V_f(x) &\leq \alpha_f(\abs{x}) &\forall x \in \mathbb{R}^n
%\end{align}

\subsection{Suboptimal MPC}
Suboptimal MPC refers to the two-step process of solving the MPC problem
approximately given a warm start, and then using the approximate solution to build a warm start 
for the MPC problem at the next time step. Given a state $x$ and warm start 
$\tilde{\mathbf{u}} \in \mathcal{U}_N(x)$, we consider the following three conditions
\begin{align}
\label{eq:feasible}
\mathbf{u} &\in \mathcal{U}_N(x) \\
\label{eq:improved}
V_N(x,\mathbf{u}) &\leq V_N(x,\tilde{\mathbf{u}})  \\
\label{eq:terminal}
V_N(x,\mathbf{u}) &\leq V_f(x) \ \text{if} \ x \in r\mathbb{B}
\end{align}
where $r>0$ is sufficiently small so that $r\mathbb{B} \subset \mathbb{X}_f$. Conditions 
\eqref{eq:feasible} and \eqref{eq:terminal} define the set of feasible warm starts
\begin{equation}
\label{eq:feaswarm}
\tilde{\mathcal{U}}_r(x) = \{ \tilde{\mathbf{u}} \mid \tilde{\mathbf{u}} \in \mathcal{U}_N(x) \ 
\text{and} \ V_N(x,\tilde{\mathbf{u}}) \leq V_f(x) \ \text{if} \ x \in r\mathbb{B} \}
\end{equation}
and all three conditions together define the suboptimal controller's feasible set
\begin{align}
\label{eq:subcontrol}
\mathcal{U}_r(x,\tilde{\mathbf{u}}) 
= \{ \mathbf{u} \mid &\mathbf{u} \in \mathcal{U}_N(x), \
V_N(x,\mathbf{u}) \leq V_N(x,\tilde{\mathbf{u}}), \nonumber \\
& V_N(x,\mathbf{u}) \leq V_f(x) \ \text{if} \ x \in r\mathbb{B} \}
\end{align}
The
suboptimal control law $\kappa_N(x,\tilde{\mathbf{u}})$ is a function of both the state $x$ and the
warm start $\tilde{\mathbf{u}}$, and may select any element of $\mathcal{U}_r(x,\tilde{\mathbf{u}})$.

Given any suboptimal input $\mathbf{u} \in \mathcal{U}_r(x,\tilde{\mathbf{u}})$,
we construct the warm start for the successor state $x^+=f(x,u(0))$ as follows

\begin{align}
\tilde{\mathbf{u}}^+ 
&=
\{u(1),u(2),\dots,u(N-1),\kappa_f(\phi(N;x,\mathbf{u}))\}  \notag  \\
&:= \zeta(x,\mathbf{u}) \label{eq:warm}
\end{align}
We now summarize the suboptimal MPC algorithm. 
\begin{algorithm}[Suboptimal MPC] .\\
\label{alg:suboptimal}
\begin{itemize}
\item Construct $\mathbb{X}_f$ and $V_f(\cdot)$ satisfying Assumption \ref{as:terminalstability}
\item Select $r > 0$ such that $r\mathbb{B} \subset \mathbb{X}_f$
\item Provide initial state $x(0) \in \mathcal{X}_N$ 
and any initial warm start $\tilde{\mathbf{u}}(0) \in \tilde{\mathcal{U}}_r(x(0))$
\item Repeat for $k = 0,1,\dots$
\begin{enumerate}
\item Record current state $x(k)$
\item Compute any input $\mathbf{u} \in \mathcal{U}_r(x(k),\tilde{\mathbf{u}}(k))$
%\item Perform zero or more optimization iterations of the MPC problem \eqref{eq:mpc} subject to 
%\eqref{eq:feasible} - \eqref{eq:terminal} to obtain $\mathbf{u}$
\item Compute the next warm start $\tilde{\mathbf{u}}(k+1)$ according to \eqref{eq:warm}
\item Inject the first element of the input sequence $\mathbf{u}$ and set $k \leftarrow k+1$  
\end{enumerate}
\end{itemize}
\end{algorithm}

\section{Asymptotic stability of suboptimal MPC}
Because the control law $\kappa_N(x,\tilde{\mathbf{u}})$ is a function of the warm start, which is
itself a function of the previous state and warm start, we subsequently analyze the behavior of the
extended state $z = (x,\tilde{\mathbf{u}})$. 
The extended state evolves according to
\begin{equation}
\label{eq:nominc}
z^+ \in H(z) = \{(x^+,\tilde{\mathbf{u}}^+) \ | \ x^+ = f(x,u(0)), \tilde{\mathbf{u}}^+ = \zeta(x,\mathbf{u}), 
\mathbf{u} \in \mathcal{U}_r(z)\}
\end{equation}
where $\zeta(\cdot)$ is defined in \eqref{eq:warm}. 
We denote by $\psi(k;z)$ an arbitrary solution of 
\eqref{eq:nominc} with initial extended state $z$ and denote by $\psi_x(k;z)$ the accompanying state 
component of the trajectory. Since the initial warm start is specified only as a function of $x$, 
$\tilde{\mathbf{u}} \in \tilde{\mathcal{U}}_r(x)$,
we can define a state evolution equation in $x$ by
\begin{equation}
\label{eq:nomincx}
\psi_x(k;z) = \psi_x(k;x,\tilde{\mathbf{u}} \in \tilde{\mathcal{U}}_r(x)) := \phi(k;x)
\end{equation}
%We also define the restriction of $\mathcal{Z}_N$ satisfying \eqref{eq:terminal} by
%\begin{equation*}
%\mathcal{Z}_r = \{(x,\mathbf{u}) \mid (x,\mathbf{u}) \in \mathcal{Z}_N \ \text{and} \
%V_N(x,\mathbf{u}) \leq V_f(x) \ \text{if} \ x \in r\mathbb{B}\}
%\mathcal{Z}_r = \{(x,\mathbf{u}) \mid x \in \mathcal{X}_N \ \text{and} \
%\mathbf{u} \in \tilde{\mathcal{U}}_r(x)\}
%\end{equation*}
%\vspace{-5mm}
\begin{definition}[Asymptotic Stability of Difference Inclusions]
\label{def:as}
We say the origin of the difference inclusion $z^+ \in H(z)$ is 
asymptotically stable on the positive invariant set $\mathcal{Z}$ if there exists a function
$\beta \in \mathcal{KL}$ such that for any $z \in \mathcal{Z}$, all solutions
$\psi(k;z)$ satisfy
\begin{equation*}
%\psi(k;z) \in \mathcal{Z} \quad \text{and} \quad \abs{\psi(k;z)} \leq 
\abs{\psi(k;z)} \leq \beta(\abs{z},k) \quad \forall k \in \mathbb{I}_{\geq 0}
\end{equation*}
\end{definition}
\begin{proposition}[Nominal Asymptotic Extended State Stability]
\label{pp:nominalstability}
Under Assumptions \ref{as:continuity} -- \ref{as:pdstagecost}, the origin of the closed-loop extended state
system \eqref{eq:nominc} is asymptotically stable on any compact invariant
subset of $\mathcal{Z}_N$.
\end{proposition}
\begin{theorem}[Nominal Asymptotic Stability]
\label{th:nominalstability}
Under Assumptions \ref{as:continuity} -- \ref{as:pdstagecost}, the origin of the closed-loop
system \eqref{eq:nomincx} is asymptotically stable on any compact
invariant subset of $\mathcal{X}_N$.
\end{theorem}

\section{Robust asymptotic stability of suboptimal MPC}
\subsection{Disturbances and robust stability definitions}
For robustness analysis, we consider the following modification to \eqref{eq:system}
\begin{align*}
\label{eq:disturbed}
x^+ &= f(x,u) + d \\
x_m &= x + e
\end{align*}
which we choose to rewrite as
\begin{equation}
\label{eq:disturbed}
x_m^+ = f(x_m-e,u) + d + e^+
\end{equation}
The predicted successor state is defined as
\begin{equation*}
\tilde{x}^+ = f(x_m,u)
\end{equation*}
The warm start \eqref{eq:warm}, suboptimal controller \eqref{eq:subcontrol} and Algorithm \ref{alg:suboptimal} now rely
on the measured state $x_m$ instead of the true state $x$, but are otherwise unchanged.
We redefine the extended state by
\begin{equation*}
z_m = (x_m,\tilde{\mathbf{u}})
\end{equation*}
The perturbed extended system then evolves as
\begin{align}
z_m^+ \in H_{ed}(z_m) 
= \{(x_m^+,\tilde{\mathbf{u}}^+) \mid 
&x_m^+ = f(x_m-e,u(0))+d+e^+, \notag \\
&\tilde{\mathbf{u}}^+ = \zeta(x_m,\mathbf{u}), \mathbf{u} \in
\mathcal{U}_r(z_m)\} \label{eq:perturbedinc}  
\end{align}
For given disturbance sequences $\mathbf{d}$ and $\mathbf{e}$
and initial extended state $z_m$ we denote an arbitrary solution of 
\eqref{eq:perturbedinc}
by $\psi_{ed}(k;z_m)$ and the corresponding state trajectory by $\psi_{x,ed}(k;z_m)$. As before,
since the initial warm start is specified only as a function of $x_m$, 
$\tilde{\mathbf{u}} \in \tilde{\mathcal{U}}_r(x_m)$,
we can define a state evolution equation in $x_m$ by
\begin{equation}
\label{eq:perturbedincx}
\psi_{x,ed}(k;z_m) = \psi_{x,ed}(k;x_m,\tilde{\mathbf{u}} \in \tilde{\mathcal{U}}_r(x_m)) := \phi_{ed}(k;x_m)
\end{equation}
\begin{definition}[Robust Asymptotic Stability]
\label{def:ras}
The origin of the closed-loop system $z_m^+ \in H_{ed}(z_m)$ is robustly asymptotically stable
(RAS) on the positive invariant set $\mathcal{C}$
if there exists $\beta \in \mathcal{KL}$, $\sigma_d,\sigma_e \in \mathcal{K}$ and $\delta > 0$ such that
for each $z_m \in \mathcal{C}$
%, for all $\tilde{\mathbf{u}} \in \tilde{\mathcal{U}}_r(x_m)$ 
and for all disturbance sequences $\mathbf{d}$ and $\mathbf{e}$
satisfying $\norm{\mathbf{d}} \leq \delta$ and $\norm{\mathbf{e}} \leq \delta$, we have that
\begin{equation}
\label{eq:ras}
%\abs{\phi_{ed}(k;x,\tilde{\mathbf{u}},\mathbf{d}_i,\mathbf{e}_i)} \leq \beta(\abs{x},k) + \sigma_d(\norm{
%\mathbf{d}_{i-1}})+ \sigma_e(\norm{\mathbf{e}_i})
%\abs{\phi_{ed}(k;x_m,\tilde{\mathbf{u}})} \leq \beta(\abs{x_m},k) + \sigma_d(\norm{
%\mathbf{d}_{k-1}})+ \sigma_e(\norm{\mathbf{e}_k})
\abs{\psi_{ed}(k;z_m)} \leq \beta(\abs{z_m},k) + \sigma_d(\norm{
\mathbf{d}_{k-1}})+ \sigma_e(\norm{\mathbf{e}_k})
\end{equation}
for all $k \in \mathbb{I}_{\geq 0}$.
\end{definition}

\subsection{Main results}
\begin{proposition}[Robust Asymptotic Extended State Stability]
\label{prop:mainiss}
Under Assumptions \ref{as:continuity}--\ref{as:pdstagecost}, 
the origin of the perturbed closed-loop extended state system 
\eqref{eq:perturbedinc} is RAS on any compact invariant subset of $\mathcal{Z}_N$.
\end{proposition}
\begin{theorem}[Robust Asymptotic Stability]
\label{th:mainiss}
Under Assumptions \ref{as:continuity}--\ref{as:pdstagecost}, 
the origin of the perturbed closed-loop system 
\eqref{eq:perturbedincx} is RAS on any compact invariant subset of $\mathcal{X}_N$.
\end{theorem}
\begin{remark}
If the admissible set $\mathcal{X}_N$ is \textit{bounded}, then the
result holds for all $\mathcal{X}_N$ since the admissible set is closed and also an
invariant set. If the admissible set is unbounded,
then the result holds for the intersection of $\mathcal{X}_N$ with any
closed, invariant set, such as the level set of the optimal MPC
controller, $\lev_{\alpha}V_N^0$ for $\alpha > 0$ as large as one pleases.
\end{remark}
\begin{figure}
\scalebox{.6}{\input{feasibility}}
\caption{The terminal control law computed based on the terminal state $\phi(N-1,\tilde{x}^+,\tilde{\mathbf{u}}^+)$ takes the system
to the interior of the terminal set. By uniform continuity, the actual and measured trajectories reach the interior as well.}
\label{fig:feasibility}
\end{figure} 

\begin{remark}[Comparisons with Pannocchia et al., 2011]
The assumptions and robust stability definition are essentially unchanged. 
The primary difference is that robust stability is proven for (essentially)
all of $\mathcal{X}_N$ without changing the terminal penalty, whereas in the preceding paper the terminal penalty was increased and stability 
was proven for state trajectories such that $\phi(N-1,\tilde{x}^+,\tilde{\mathbf{u}}^+)$ was in the interior of $\mathbb{X}_f$. 
The proposed new proof is based on Figure \ref{fig:feasibility};
the successor warm start $\tilde{\mathbf{u}}^+$, constructed from the predicted successor state $\tilde{x}^+$, brings the predicted successor 
terminal state $\phi(N,\tilde{x}^+,\tilde{\mathbf{u}}^+)$ to the interior of $\mathbb{X}_f$, thereby guaranteeing recursive feasibility and
robust stability.
%The key step in the robustness proof is still showing that the predicted 
%successor terminal state $\phi^N$ lies in the interior of $\mathbb{X}_f$. However, 
\end{remark}

\begin{remark}
Discussion for Gabriele and Steve. Recall that these results include
\textit{optimal} MPC as a special case.  So Theorem \ref{th:mainiss}
tells us that optimal MPC is RAS. An interesting issue that has been debated
recently is whether we require \textit{continuity} of $V_N^0(\cdot)$
and/or $\kappa_N(\cdot)$ to obtain robustness.  The standard
robustness arguments tend to break down if $\kappa_N(\cdot)$ is not
continuous. So, what about Assumptions
\ref{as:continuity}--\ref{as:pdstagecost}, coupled with the lack of 
state constraints. Does that imply $\kappa_N(\cdot)$ continuous? 
If so, then we should bring that fact to the reader's attention.
If not, then we have proved robustness for discontinuous
$\kappa_N(\cdot)$, which is worth pointing out in the talk and full
paper.
\end{remark}


\section{Discontinuous MPC Example}

Consider the nonlinear system defined  by
\begin{alignat*}{1}
    x_1^+ &= x_1+u\\
    x_2^+ &= x_2+u^3
\end{alignat*}
with stage cost $\ell(\cdot)$ given by 
\begin{equation*}
  \ell(x,u) \eqbyd |x|^2 + u^2
\end{equation*}
The constraint sets are $\bbX=\bbR^2$, $\bbU=\bbR$,
and $\bbX_f\eqbyd\{0\}$, i.e., there are no state and control
constraints, and the terminal state must satisfy the constraint
$x(N)=0$. 

\begin{figure}
\centering
\resizebox{0.9\textwidth}{!}{\input{feasibility_set}}
\caption{Feasibility sets $\mathcal{X}_1$, $\mathcal{X}_2$, and
  $\mathcal{X}_3$.}
\label{fig:feasibility_set}
\end{figure} 

Next we consider the feasibility sets $\mathcal{X}_N$ for $N \geq 1$. For
$N=1$, the terminal constraint $x(N)=0$ gives
\begin{equation*}
x_1(1) = x_1(0) + u(0) = 0 \qquad
x_2(1) = x_2(0) + u(0)^3 = 0
\end{equation*}
These equations have a solution only for $x_2(0)= x_1(0)^3$, which
defines the feasibility set $\mathcal{X}_1$, depicted in Figure
\ref{fig:feasibility_set}. Next consider $N=2$.  The terminal
constraint can now be expressed as
\begin{equation*}
x_1(2) = x_1(0) + u(0) + u(1)= 0 \qquad
x_2(1) = x_2(0) + u(0)^3 + u(1)^3 = 0
\end{equation*}
Solving the first equation for $u(1)$ and substituting into the second
equation gives
\begin{equation*}
0 = 3x(1)u(0)^2 + 3 x(1)^2 u(0) + x(1)^3 - x(2)
\end{equation*}
For $u(0)$ to be real, we require the discriminant of this quadratic
equation to be nonnegative, which reduces to
\begin{equation*}
-3x(1)^4 + 12 x(1)x(2) \geq 0
\end{equation*}
This inequality defines the feasibility region $\bbX_2$
\begin{equation*}
\bbX_2 = \left\{ (x_1, x_2) \mid 
\begin{cases}
x_2 \geq (1/4)x_1^3, \quad &x_1 > 0 \\
x_2 = \bbR, \quad &x_1 = 0 \\
x_2 \leq (1/4)x_1^3, \quad &x_1 < 0
\end{cases} \right\}
\end{equation*}
which is also depicted as the shaded region in Figure \ref{fig:feasibility_set}.

We next show that $\mathcal{X}_3 = \bbR^2$ so that the shortest
horizon that we can use to stabilize all of $\bbR^2$ is $N=3$. 


Hence, although there are three control actions, $u(0)$, $u(1)$, and
$u(2)$, 
two must be employed to satisfy the terminal constraint, leaving only
one degree of freedom. Choosing $u(0)$ to be the free decision
variable automatically constrains $u(1)$ and $u(2)$ to be functions
of the initial state $x$ and the first control action $u(0)$. Solving
the equation
\begin{alignat*}{2}
  x_1(3) &=  x_1+u(0)+u(1)+u(2) &= 0\\
  x_2(3) &=  x_2+u(0)^3+u(1)^3+u(2)^3 &=0 
\end{alignat*}
for $u(1)$ and $u(2)$ yields
\begin{alignat*}{2}
  u(1) &= -x_1/2-u(0)/2 \pm \sqrt{b}\\
  u(2) &= -x_1/2-u(0)/2 \mp \sqrt{b}
\end{alignat*}
in which
\begin{equation*}
  b =\frac
  {3u(0)^3 - 3u(0)^2x_1 - 3u(0)x_1^2 -x_1^3 + 4x_2}{12(u(0)+x_1)}
\end{equation*}
Clearly a real solution exists only if $b$ is positive, i.e., if both
the numerator and denominator in the expression for $b$ have the same
sign.  The optimal control problem is defined by
\begin{equation*}
  V_3^0(x) = \min_{\useq} \{V_3(x,\useq) \mid \phi(3;x,\useq)=0\}
\end{equation*}


We wish to find the bifurcation points of the nonlinear constraint
$x(3)=0$.  That has been reduced to finding the points where $b$
passes through zero, which is equivalent to the cubic equation
\begin{equation*}
3u(0)^3 - 3u(0)^2x_1 - 3u(0)x_1^2 -x_1^3 + 4x_2 = 0
\end{equation*}
From the theory of cubic equations, we know that changes in the number
of real roots to the equation $au^3+bu^2+cu+d=0$ are determined by the
sign of the discriminant 
\begin{equation*}
\Delta = 18 abcd - 4b^3d + b^2c^2 -4ac^3 -27 a^2 d^2
\end{equation*}
For $\Delta>0$, the equation has three distinct real roots, and when
$\Delta<0$, one root is real and the other two are complex
conjugates.  Substituting $a=3$, $b=-3x_1$, $c=-3x_1^2$, and
$d=-x_1^3+4x_2$ into the expression for the discriminant and factoring
gives 
\begin{align*}
\Delta &= 432 (-x_1^6 + 10 x_1^3x_2 - 9 x_2^2)\\
       &= -432 (x_1^3 - 9 x_2)(x_1^3 -x_2)
\end{align*}
Setting $\Delta=0$ gives two $(x_1, x_2)$ lines at which the number
of real roots changes from one to three. The line that generates a discontinuity in
$V_3^0(x)$ corresponds to setting the first factor to zero giving
\begin{equation*}
x_2 = (1/9) x_1^3
\end{equation*}
Note also that $V_3^0(\cdot)$ is not discontinuous at the origin
because $V_3^0(\cdot)$ on both sides of the line of  discontinuity
go to zero at the origin. Let the set of points of discontinuity be denoted by
\begin{equation*}
\bbD \eqbyd \{ (x_1, x_2) \mid x_2 = (1/9) x_1^3, \quad x_1 \neq 0 \}
\end{equation*}
The set $\bbD$ is shown along with the invariant set $\mathcal{X}_2$
in Figure \ref{fig:discontinuities}.  Note that these two sets do not
intersect, which is the key reason why the closed-loop system is
\textit{robustly} stable.
\begin{figure}
\centering
\resizebox{0.9\textwidth}{!}{\input{discontinuities}}
\caption{The set $\bbD$, points of discontinuity in cost $V_3^0(\cdot)$,
  and invariant set $\mathcal{X}_2$. Note that the two sets
  do not intersect (the origin is not an element of $\bbD$).}
\label{fig:discontinuities}
\end{figure} 



% \bibliographystyle{abbrvnat}
% \bibliography{abbreviations,articles,books,unpub,proceedings}

\end{document}
