\documentclass{article}
%\documentclass[letterpaper, 10pt, conference]{ieeeconf}
%\IEEEoverridecommandlockouts                          
%\overrideIEEEmargins
%\bibliographystyle{IEEEtran}

% Packages
\usepackage{hyperref} % clickable links
\usepackage{url}
%\usepackage{rxn}
\usepackage{natbib}
\usepackage{graphicx,color,amsmath,amssymb}
%\usepackage{mathptmx}
%\usepackage{times}
\usepackage{cases}
\usepackage{subfig}
\usepackage{setspace}
\usepackage{booktabs}
\usepackage{balance}
\usepackage{verbatim} % \begin and \end {comment}

% Environments
\newtheorem{theorem}{Theorem} 
\newtheorem{assumption}{Assumption}
\newtheorem{definition}{Definition}
\newtheorem{lemma}{Lemma}
\newtheorem{proposition}{Proposition}
\newtheorem{remark}{Remark}
\newtheorem{corollary}{Corollary}
\newtheorem{algorithm}{Algorithm}
\newenvironment{proof}{\noindent {\em Proof}.\ }{\hspace*{\fill}$\Box$\medskip\\}

% Commands
\newcommand{\abs}[1]{\left\lvert #1 \right\rvert}
\newcommand{\norm}[1]{\left\lVert #1 \right\rVert}



\title{On the robustness of optimal and suboptimal economic MPC}

\author{Cuyler N. Bates and James B. Rawlings
  \thanks{C. N. Bates and J. B. Rawlings are with Dept. of Chemical and
    Biological Engineering, Univ. of Wisconsin, Madison, WI 53706, USA
    \texttt{\footnotesize (cnbates@wisc.edu,rawlings@engr.wisc.edu)}}} 

\begin{document}

\maketitle

\begin{abstract}
\begin{comment}
Traditionally, the problem of dynamic economic optimization of processes has
been divided into two subproblems. First, a static economic optimization
is performed to select the process steady state with the lowest cost. Then,
a regular is designed to guide the system dynamically to the chosen steady
state without regard for the dynamic cost incurred. Increasingly, however, 
due to demand for improving dynamic economic performance and a commensurate
increase in the quality of available dynamic models and computational tools
, interest in dynamic economic optimization methods is growing rapidly. In 
this paper, we investigate the properties of economic model predictive 
control, which is the use of an economic objective function in place of the
regulatory objective function widely used in model predictive control. \\
When considering the general economic model predictive control problem,
however, we encounter several crucial questions. Firstly, economic model
predictive control problems are almost never convex. Therefore, even with
state of the art solution techniques, we can, in general, hope only to
find a local optimum, not a global optimum as is universally assumed in
previous papers on the subject. Secondly, when operating a control scheme,
there are inevitably disturbances affecting the system. We would 
therefore like to know under what conditions the economic control law 
designed for the nominal system is able to reject those disturbances. \\
The contributions of the paper are as follows. First, we extend the
robustness result of \citep{pannocchia:rawlings:wright:2011} to a more 
general class of cost functions. We then show that these results apply to
suboptimal economic MPC. Finally, we present a proof giving guarantees on
the economic performance of suboptimal economic MPC. The conclusion of the
paper is that not only can the economic performance of optimal economic MPC
be extended to the suboptimal case, but in addition the impressive 
robustness of nominal (noneconomic) MPC can be extended to both optimal and
suboptimal economic MPC as well. 
\end{comment}
\end{abstract}

\section{Introduction}
\begin{remark}[Note to self]
FIXME: import changes from short.pdf
get rid of $\tilde{\phi}$ etc etc
\end{remark}
\begin{comment}
words words words \\ FIMXE
I did a brief literature search and the results here all appear to be novel
but I will need to do a more in depth search FIXME. \\
\\
% FIXME: too similar
\end{comment}
The differences in assumptions between this paper and \cite{pannocchia:rawlings:wright:2011}
are discussed in Remark \ref{rm:assumptions}. Differences in the analysis and theory are discussed
in Remark \ref{rm:analysis}. Open issues are discussed in Remark \ref{rm:issues}. 

Notation. The symbols $\mathbb{I}_{\geq 0}$ and $\mathbb{R}_{\geq 0}$ denote
the nonnegative integers and reals, respectively. The symbol $\mathbb{I}_{
0:N}$ denotes the set $\{0,1,\dots,N\}$. The interior of a set $X$ is
denoted int$X$. Given $V:X\rightarrow\mathbb{R}_{\geq 0}$ and $\alpha > 0$,
we define $\text{lev}_\alpha V = \{x \in X \mid V(x) \leq \alpha\}$.
The symbol $\abs{\cdot}$ denotes the Euclidean norm and 
$\mathbb{B}$ denotes the closed ball of radius 1 centered at the origin.
Set deletion is defined as $A \backslash B = \{x \in A \mid x \notin B\}$. 
The relation $A \subseteq B$ denotes $A$ is a subset of $B$ while the 
relation $A \subset B$ denotes $A$ is a proper subset of $B$. 

Bold symbols, e.g., $\mathbf{d}$, denote sequences, $d(k)$ denotes the element of $\mathbf{d}$ at time 
$k \in \mathbb{I}_{\geq{0}}$ and 
$\mathbf{d}_k$ denotes the truncation of the sequence up to time $k$, i.e.
$\mathbf{d}_k = \{d(0),d(1),\dots,d(k)\}$.
We define $\norm{\mathbf{d}} = \max_{k \geq 0}\abs{d(k)}$. 

Given time $k$, initial state $x$ and input sequence $\mathbf{u}$, $\tilde{\phi}(k;x,\mathbf{u})$ denotes 
the open-loop state solution to the system \eqref{eq:system}.
Given time $k$, initial state $x$ and initial warm start $\tilde{\mathbf{u}}$, $\psi(k;x,\tilde{\mathbf{u}})$ denotes 
an arbitrary closed-loop extended state solution
to the difference inclusion \eqref{eq:nominc}
and $\phi(k;x,\tilde{\mathbf{u}})$ denotes the resulting state component trajectory. 
Given time $k$, initial state $x$, initial warm start $\tilde{\mathbf{u}}$ and disturbance and measurement
error sequences $\mathbf{d}$ and $\mathbf{e}$, $\psi_{ed}(k;x,\tilde{\mathbf{u}})$ denotes 
an arbitrary closed-loop extended state solution
to the perturbed difference inclusion \eqref{eq:perturbedinc}
and $\phi_{ed}(k;x,\tilde{\mathbf{u}})$ denotes the resulting state component trajectory. 

\section{Basic definitions and assumptions}
\subsection{MPC problem}
\label{sec:mpc}
The subject of this paper is autonomous discrete-time systems of the form 
\begin{equation}
\label{eq:system}
x^+=f(x,u) 
\end{equation}
where $x \in \mathbb{R}^n$ is the state, $u \in \mathbb{R}^m$ is the input
and $x^+$ denotes the successor state. 
We consider here the case of input constraints
\begin{equation*}
u \in \mathbb{U}, \quad x \in \mathbb{R}^n, \quad (x,u) \in \mathbb{Z} =   \mathbb{R}^n \times \mathbb{U}
\end{equation*}
\begin{comment}
The state and input are 
subject to the constraints
\begin{equation*}
%\label{eq:constraints}
(x,u) \in \mathbb{X} \times \mathbb{U}
\end{equation*}
\end{comment}
For model predictive control with a horizon of $N$, initial condition $x$ 
and a terminal constraint, we define the set of admissible 
$(x,\mathbf{u})$ pairs \eqref{eq:admissiblepairs}, admissible inputs
\eqref{eq:admissibleinputs}, admissible states
\eqref{eq:admissiblestates} and objective function \eqref{eq:objective} by
\begin{align}
\label{eq:admissiblepairs}
\mathcal{Z}_N &= \{ (x,\mathbf{u}) \mid
x(k+1)=f(x(k),u(k)), (x(k),u(k)) \in \mathbb{Z} \\ \nonumber
&\quad \quad \forall k \in \mathbb{I}_{0:N-1}, x(N) \in \mathbb{X}_f, x(0)=x \} \\
\label{eq:admissibleinputs}
\mathcal{U}_N(x) &= \{\mathbf{u} \mid (x,\mathbf{u}) \in \mathcal{Z}_N \}
\\
\label{eq:admissiblestates}
\mathcal{X}_N &= \{x \mid 
\exists \, \mathbf{u} \text{ such that  } (x,\mathbf{u}) \in \mathcal{Z}_N \}
\\
\label{eq:objective}
 V_N ( x, \mathbf{u} ) &= \sum_{k=0}^{N-1} \ell ( x(k),u(k) ) + V_f(x(N))
\end{align}
The result is the optimization problem %\eqref{eq:mpc}. 
\begin{equation}
\label{eq:mpc}
\mathbb{P}_N(x): V_N^0(x)=V_N(x,\mathbf{u}^0)=
\min_\mathbf{u} \Big \{ V_N(x,\mathbf{u}) \mid \mathbf{u} \in \mathcal{U}_N(x) \Big \}
\end{equation}
The existance of an optimal solution can be proven from Assumptions
\ref{as:continuity} and \ref{as:constraints}
\citep[p. 97-98]{rawlings:mayne:2009}. If the solution to $\mathbb{P}_N(x)$
is unique for all $x \in \mathcal{X}_N$, we can write the control law 
\eqref{eq:mpclaw} and closed-loop system \eqref{eq:closed}
\begin{align}
\label{eq:mpclaw}
 u &= \kappa_N(x) := u^0(0;x)\\
\label{eq:closed}
x^+ &= f(x,\kappa_N(x))
\end{align}
The case where the solution to $\mathbb{P}_N(x)$ is not unique is considered
in the following section.
\begin{assumption}[Continuity of system and cost]
\label{as:continuity}
The model $f:\mathbb{R}^n \times \mathbb{R}^m \rightarrow \mathbb{R}^n$, 
stage cost $\ell: \mathbb{R}^n \times \mathbb{R}^m \rightarrow \mathbb{R}_
{\geq 0}$ and terminal cost $V_f:\mathbb{R}^n \rightarrow \mathbb{R}_{\geq 0}$
are continuous. Furthermore, for some steady state $(x_s,u_s)$, we have $\ell(x_s,u_s)=0$
and $V_f(x_s) = 0$.
\end{assumption}
For the remainder of the paper, we assume without loss of generality that 
$(x_s,u_s) = (0,0)$.
\begin{assumption}[Properties of constraint set]
\label{as:constraints}
The set $\mathbb{U}$ is compact and contains the origin. 
%The sets $\mathbb{X}$ and $\mathbb{X}_f$ are closed, contain the origin in their interiors and $\mathbb{X}_f \subset \mathbb{X}$.
%The set $\mathbb{X}_f$ is closed and contains the origin in its interior.
% We define X_f to be a level set of V_f
The set $\mathbb{X}_f$ is defined by  $\mathbb{X}_f= \text{lev}_\alpha V_f = \{x \in \mathbb{R}^n \ | \ 
V_f(x) \leq \alpha\}$, with $\alpha > 0$.
\end{assumption}
\begin{assumption}[Stability assumption]
\label{as:terminalstability}
There exists a terminal control law $\kappa_f:\mathbb{X}_f \rightarrow
\mathbb{U}$ such that
\begin{align*}
%\label{eq:invariant}
f(x,\kappa_f(x)) &\in \mathbb{X}_f \qquad &\forall x \in \mathbb{X}_f\\
%\label{eq:terminalstability}
V_f(f(x,\kappa_f(x))) &\leq V_f(x)-\ell(x,\kappa_f(x))  
&\forall x \in \mathbb{X}_f
\end{align*}
\end{assumption}
\begin{assumption}[Function bounds]
\label{as:pdstagecost}
There exists a function $\alpha_\ell \in \mathcal{K}_\infty$ such that
\begin{equation*}
%\label{eq:pdstagecost}
\ell(x,u) \geq \alpha_\ell(\abs{(x,u)}) \quad \quad \forall (x,u) \in \mathbb{Z}
\end{equation*}
\end{assumption}
\begin{remark}[Assumptions]
\label{rm:assumptions}
The result in \cite{pannocchia:rawlings:wright:2011} concerning robust stability, Theorem 41,
uses essentially the same assumptions as the corresponding result here, Theorem \ref{th:mainiss}.

The only difference is that since we consider asymptotic
stability rather than exponential stability, we do not require exponential upper bounds for 
$V_N(\cdot)$ and $V_f(\cdot)$ (Assumption 4 there, Assumption \ref{as:pdstagecost} here).
\end{remark}
We require the following lemma on $\mathcal{K}$-functions \citep[Proposition 1]{rawlings:mayne:2011}. % actually, it's slightly different
\begin{lemma}
\label{lem:overbound}
Let a function $V(x)$ be defined and locally bounded on $\mathbb{R}^n$. If 
$V(x)$ is continuous at $x=0$ and $V(0) = 0$, then there exists $\alpha \in \mathcal{K}$ such that
\begin{equation*}
V(x) \leq \alpha(\abs{x})  \quad \forall x \in \mathbb{R}^n
\end{equation*}
$\alpha$ can be chosen to be of class $\mathcal{K}_\infty$.
\end{lemma}
From Assumption \ref{as:continuity} and Lemma \ref{lem:overbound}, we 
conclude that there exist functions $\alpha_V,\alpha_f \in \mathcal{K}_\infty$
such that
\begin{align}
\label{eq:vnbound}
V_N(x,\mathbf{u}) &\leq \alpha_V(\abs{(x,\mathbf{u})}) &\forall (x,\mathbf{u}) \in \mathbb{R}^{n+mN} \\
\label{eq:vfbound}
V_f(x) &\leq \alpha_f(\abs{x}) &\forall x \in \mathbb{R}^n
\end{align}
Given this upper bound on $V_f(x)$, we therefore have that $\mathbb{X}_f$ is compact.
We make several remarks:
\begin{itemize}
\item In Assumption \ref{as:constraints}, we allow the optimal steady state
to be on the boundary of $\mathbb{U}$.
\item For Assumption \ref{as:terminalstability}, the design of the terminal
controller for quadratic $\ell(x,u)$ is covered in \citep[p. 136-138]{rawlings:mayne:2009}. 
% FIXME and for general $\ell(x,u)$ it is covered in  
\end{itemize}

\subsection{Suboptimal MPC}
Suboptimal MPC refers to the two-step process of solving the MPC problem
approximately given a warm start, and then using the approximate solution to build a warm start 
for the MPC problem at the next time step. Given a state $x$ and warm start 
$\tilde{\mathbf{u}} \in \mathcal{U}_N(x)$, we consider the following three conditions
\begin{align}
\label{eq:feasible}
\mathbf{u} &\in \mathcal{U}_N(x) \\
\label{eq:improved}
V_N(x,\mathbf{u}) &\leq V_N(x,\tilde{\mathbf{u}})  \\
\label{eq:terminal}
V_N(x,\mathbf{u}) &\leq V_f(x) \ \text{if} \ x \in r\mathbb{B}
\end{align}
where $r>0$ is sufficiently small so that $r\mathbb{B} \subset \mathbb{X}_f$. Conditions 
\eqref{eq:feasible} and \eqref{eq:terminal} define the set of feasible warm starts
\begin{equation}
\label{eq:feaswarm}
\tilde{\mathcal{U}}_r(x) = \{ \tilde{\mathbf{u}} \mid \tilde{\mathbf{u}} \in \mathcal{U}_N(x) \ 
\text{and} \ V_N(x,\tilde{\mathbf{u}}) \leq V_f(x) \ \text{if} \ x \in r\mathbb{B} \}
\end{equation}
and all three conditions together define the suboptimal controller's feasible set
\begin{align}
\label{eq:subcontrol}
\mathcal{U}_r(x,\tilde{\mathbf{u}}) &= \{ (x,\mathbf{u}) \mid \mathbf{u} \in \mathcal{U}_N(x), \
V_N(x,\mathbf{u}) \leq V_N(x,\tilde{\mathbf{u}}) \nonumber \\
&\qquad \text{and} \ V_N(x,\tilde{\mathbf{u}}) \leq V_f(x) \ \text{if} \ x \in r\mathbb{B} \}
\end{align}
The
suboptimal control law $\kappa_N(x,\tilde{\mathbf{u}})$ is a function of both the state $x$ and the
warm start $\tilde{\mathbf{u}}$, and may select any element of $\mathcal{U}_r(x,\tilde{\mathbf{u}})$.
Note that condition \eqref{eq:terminal} is required so that $\abs{x} \rightarrow 0$ implies 
$\abs{\mathbf{u}} \rightarrow 0$, which is proven in Lemma \ref{lem:utox}.

Given any suboptimal input $\mathbf{u} \in \mathcal{U}_r(x,\tilde{\mathbf{u}})$,
%\eqref{eq:feasible} -- \eqref{eq:terminal},
we construct the warm start for the successor state $x^+=f(x,u(0))$ as follows

\begin{equation}
\label{eq:warm}
\tilde{\mathbf{u}}^+ =
%\begin{cases}
\{u(1),u(2),\dots,u(N-1),\kappa_f(\tilde{\phi}(N;x,\mathbf{u}))\} 
%& \text{if} \ x^+ \notin r\mathbb{B} \\
%\{\kappa_f(x^+),\kappa_f(f(x^+,\kappa_f(x^+))),\dots\} & \text{otherwise}
%\end{cases}
\end{equation}

We summarize the suboptimal algorithm as follows
\begin{algorithm}[Suboptimal MPC] .\\
\label{alg:suboptimal}
\begin{itemize}
\item Construct $\mathbb{X}_f$ and $V_f(\cdot)$ satisfying Assumption \ref{as:terminalstability}
\item Select $r > 0$ such that $r\mathbb{B} \subset \mathbb{X}_f$
\item Provide initial state $x(0) \in \mathcal{X}_N$ 
and any initial warm start $\tilde{\mathbf{u}}(0) \in \tilde{\mathcal{U}}_r(x(0))$
\item Repeat for $k = 0,1,\dots$
\begin{enumerate}
\item Record current state $x(k)$
\item Compute any input $\mathbf{u} \in \mathcal{U}_r(x(k),\tilde{\mathbf{u}}(k))$
%\item Perform zero or more optimization iterations of the MPC problem \eqref{eq:mpc} subject to 
%\eqref{eq:feasible} - \eqref{eq:terminal} to obtain $\mathbf{u}$
\item Compute the next warm start $\tilde{\mathbf{u}}(k+1)$ according to \eqref{eq:warm}
\item Inject the first element of the input sequence $\mathbf{u}$ and set $k \leftarrow k+1$  
\end{enumerate}
\end{itemize}
\end{algorithm}

\begin{proposition}
\label{pp:opt}
For any $x \in \mathcal{X}_N$ and any $\tilde{\mathbf{u}} \in \tilde{\mathcal{U}}_r(x)$, the set 
$\mathcal{U}_r(x,\tilde{\mathbf{u}})$ is never empty. Furthermore, when applying Algorithm 
\ref{alg:suboptimal}, a member of $\mathcal{U}_r(x,\tilde{\mathbf{u}})$ can always be computed without
any optimization.
\end{proposition}
\begin{proof}
First, the existance of an optimal solution to \eqref{eq:mpc} can be proven from Assumptions
\ref{as:continuity} and \ref{as:constraints}
\citep[p. 97-98]{rawlings:mayne:2009}.
Next, we note that any optimal solution $\mathbf{u}^0(x)$ satisfies \eqref{eq:feasible} and 
\eqref{eq:improved}. For $x \in r\mathbb{B}$, we consider the following alternate warm start
\begin{equation}
\label{eq:altwarm}
\tilde{\mathbf{u}}_f = \{\kappa_f(x^+),\kappa_f(f(x^+,\kappa_f(x^+))),\dots\} 
\end{equation}
Applying Assumption \ref{as:terminalstability} for $k = 0, \dots, N-1$, we have
\begin{equation*}
V_N(x,\tilde{\mathbf{u}}_f) = \sum_{k=0}^{N-1}\ell(x(k),u_f(k)) + V_f(x(N)) \leq V_f(x)
\end{equation*}
Therefore, $\tilde{\mathbf{u}}_f$ satisfies \eqref{eq:terminal}. 
Since $\tilde{\mathbf{u}}_f \in \mathcal{U}_N(x)$ and by optimality $V_N(x,\mathbf{u}^0(x)) \leq 
V_N(x,\tilde{\mathbf{u}}_f)$, we have that $\mathbf{u}^0(x) \in 
\mathcal{U}_r(x,\tilde{\mathbf{u}})$.

During the application of Algorithm \ref{alg:suboptimal}, we note that $\tilde{\mathbf{u}}$ computed
from \eqref{eq:warm} is feasible by construction and trivially satisfies \eqref{eq:improved}.
If $x \in r\mathbb{B}$ and $V_N(x,\tilde{\mathbf{u}}) \geq V_f(x)$, then we note that 
$\tilde{\mathbf{u}}_f$ is feasible and satisfies \eqref{eq:terminal} and therefore \eqref{eq:improved}.
We conclude that $\tilde{\mathbf{u}} \in \mathcal{U}_r(x,\tilde{\mathbf{u}})$ whenever $x \notin r\mathbb{B}$ or
$V_N(x,\tilde{\mathbf{u}}) \leq V_f(x)$
and $\tilde{\mathbf{u}}_f \in \mathcal{U}_r(x,\tilde{\mathbf{u}})$ whenever $x \in r\mathbb{B}$
and $\tilde{\mathbf{u}} \notin \mathcal{U}_r(x,\tilde{\mathbf{u}})$.
\end{proof}

%\begin{proposition}
%\label{pp:opt}
%Any optimal solution $\mathbf{u}^0(x)$ of $\mathbb{P}_N(x)$ satisfies 
%conditions \eqref{eq:feasible}, \eqref{eq:improved} 
%for all $x \in \mathcal{X}_N$ and \eqref{eq:terminal} for all 
%$x \in \mathcal{X}_f$.
%\end{proposition}
%The proof of this proposition is provided in \citet{pannocchia:rawlings:wright:2011} and 
%reproduced in the appendix.

\begin{proposition}
$\kappa_N(0,\tilde{\mathbf{u}}) = \{0\}$.
\end{proposition}
\begin{proof}
From the fact that $V_f(\cdot)$ is non-negative (by Assumption 
\ref{as:continuity}) and Assumption \ref{as:pdstagecost}, we have
\begin{align}
V_N(x,\mathbf{u}) &\geq \sum^{N-1}_{k=0}\alpha(\abs{(x(k),u(k))}) \nonumber \\
&\geq \underline{\alpha}(\abs{\mathbf{x},\mathbf{u}}) \nonumber \\
\label{eq:lowerbound}
&\geq \underline{\alpha}(\abs{x,\mathbf{u}})
\end{align}
From \eqref{eq:improved} and \eqref{eq:terminal} we have $V(0,
\mathbf{u}) \leq V_f(0)=0$. We conclude that $\mathbf{u} = 0$ for all 
suboptimal input sequences under consideration and therefore
$\kappa_N(0,\tilde{\mathbf{u}}) = \{0\}$.
\end{proof}

Because the control law $\kappa_N(x,\tilde{\mathbf{u}})$ is a function of the warm start, which is
itself a function of the previous state and input, we subsequently analyze the behavior of the
extended state $z = (x,\tilde{\mathbf{u}})$. 

\subsection{Asymptotic stability of difference inclusions}
As the system of interest is now a difference inclusion, we present the 
following definitions of asymptotic stability and the associated Lyapunov
functions. Consider the difference inclusion $z^+ \in H(z)$, such that $H(0) =
\{0\}$.
\begin{definition}[Asymptotic Stability]
\label{def:as}
We say the origin of the difference inclusion $z^+ \in H(z)$ is 
asymptotically stable on the positive invariant set $\mathcal{Z}$ if there exists a function
$\beta \in \mathcal{KL}$ such that for any $z \in \mathcal{Z}$, all solutions
$\psi(k;z)$ satisfy
\begin{equation*}
%\psi(k;z) \in \mathcal{Z} \quad \text{and} \quad \abs{\psi(k;z)} \leq 
\abs{\psi(k;z)} \leq \beta(\abs{z},k) \quad \forall k \in \mathbb{I}_{\geq 0}
\end{equation*}
\end{definition}
\begin{definition}[Lyapunov Function]
\label{def:lyap}
V is a Lyapunov function on the positive invariant set $\mathcal{Z}$ for the difference 
inclusion $z^+ \in H(z)$ if there exists functions $\alpha_1,\alpha_2,\alpha_3
\in \mathcal{K}_\infty$ such that for all $z \in \mathcal{Z}$
\begin{align}
\label{eq:lowerlyap}
\alpha_1(\abs{z}) &\leq V(z) \\
\label{eq:upperlyap}
\alpha_2(\abs{z}) &\geq V(z) \\
\label{eq:decreaselyap}
\max_{z^+ \in H(z)} V(z^+) &\leq V(z) - \alpha_3(\abs{z})
\end{align}
\end{definition}
Note that $V(\cdot)$ is not required to be continuous.
\begin{lemma}
\label{lem:stab}
If the set $\mathcal{Z}$ is positively invariant for the difference 
inclusion $z^+ \in H(z)$, $H(0) = \{0\}$, $0 \in \mathcal{Z}$ 
and there exists a Lyapunov function $V$ on $\mathcal{Z}$, then the 
origin is asymptotically stable on $\mathcal{Z}$.
\end{lemma}
The proof is functionally identical to the proof of Theorem 12 in \citep{rawlings:mayne:2011} and is
reproduced in the appendix.

\section{Asymptotic stability of suboptimal MPC}
\subsection{Extended state}
In Algorithm \ref{alg:suboptimal} we begin with a state and warm start pair
and proceed from one such pair to the next at the start of each time step. 
We therefore denote as the extended state $z = (x,\tilde{\mathbf{u}})$.
The extended state evolves according to
\begin{equation}
\label{eq:nominc}
z^+ \in H(z) = \{(x^+,\tilde{\mathbf{u}}^+) \ | \ x^+ = f(x,u(0)), \tilde{\mathbf{u}}^+ = \zeta(x,\mathbf{u}), 
\mathbf{u} \in \mathcal{U}_r(z)\}
\end{equation}
where $\zeta(\cdot)$ is the mapping corresponding to \eqref{eq:warm}. 
We denote by $\psi(k;z)$ an arbitrary solution of 
\eqref{eq:nominc} with initial extended state $z$ and denote by $\phi(k;z)$ the accompanying state 
component of the trajectory.
We also define the restriction of $\mathcal{Z}_N$ satisfying \eqref{eq:terminal} by
\begin{equation*}
%\mathcal{Z}_r = \{(x,\mathbf{u}) \mid (x,\mathbf{u}) \in \mathcal{Z}_N \ \text{and} \
%V_N(x,\mathbf{u}) \leq V_f(x) \ \text{if} \ x \in r\mathbb{B}\}
\mathcal{Z}_r = \{(x,\mathbf{u}) \mid x \in \mathcal{X}_N \ \text{and} \
\mathbf{u} \in \tilde{\mathcal{U}}_r(x)\}
\end{equation*}
\begin{lemma}
\label{lem:utox}
There exists a function $\alpha_r \in \mathcal{K}_\infty$ such that
$\abs{\mathbf{u}} \leq \alpha_r(\abs{x})$ for any $(x,\mathbf{u}) \in 
\mathcal{Z}_r$.
\end{lemma}
\begin{proof}
We first consider the case $x \in r\mathbb{B}$. From 
\eqref{eq:lowerbound}, \eqref{eq:terminal} and \eqref{eq:vfbound} we have that
\begin{equation*}
\underline{\alpha}(\abs{\mathbf{u}}) \leq \underline{\alpha}(\abs{x,\mathbf{u}}) \leq
V_N(x,\mathbf{u}) \leq V_f(x) \leq \alpha_f(\abs{x})
\end{equation*}
Therefore, $\abs{\mathbf{u}} \leq \alpha_{r'}(\abs{x})$ where
$\alpha_{r'}(\abs{x}) = \underline{\alpha}^{-1}(\alpha_f(\abs{x}))$ and $\alpha_{r'} \in \mathcal{K}_\infty$. 

Now we consider the case $x \notin r\mathbb{B}$. Define
$\mu = \max_{\mathbf{u} \in \mathbb{U}^N}\abs{\mathbf{u}}$ and note that $\mu$ is
finite because $\mathbb{U}$ is compact by Assumption \ref{as:constraints}. Next, define 
%$\gamma = \min\{1,\alpha_{r'}(r)\} > 0$. Finally, we define
$\gamma = \alpha_{r'}(r) > 0$. Finally, we define
$\alpha_r(\abs{x}) = (\mu/\gamma)\alpha_{r'}(\abs{x})$, which satisfies 
$\abs{\mathbf{u}} \leq \alpha_r(\abs{x})$ for all $(x,\mathbf{u}) \in \mathcal{Z}_r$.
\end{proof}

\subsection{Main results}
\begin{theorem}[Nominal Asymptotic Stability]
\label{th:nominalstability}
Under Assumptions \ref{as:continuity} -- \ref{as:pdstagecost}, there exists
$\beta(\cdot) \in \mathcal{KL}$ such that 
\begin{equation*}
\abs{\phi(k;z)} \leq \beta(\abs{x},k)
\end{equation*}
for any initial extended state $z = (x,\tilde{\mathbf{u}}) \in \mathcal{Z}_r$.
% There is no state evolution equation in x alone, therefore you can't claim anything is ``asymptotically
% stable'' in terms of x alone because a.s. applies only to evolution equations
\end{theorem}
%\begin{lemma}
%\label{lem:zlyap}
%$V_N(\cdot)$ is a Lyapunov function for the extended system \eqref{eq:nominc} in $\mathcal{Z}_r$.
%\end{lemma}
\begin{proof}
We proceed by proving $V_N(z)$ is a Lyapunov function for \eqref{eq:nominc} in $\mathcal{Z}_N$.
From \eqref{eq:lowerbound} we have $\underline{\alpha}(\abs{z})
\leq V_N(z)$ for all $z \in \mathcal{Z}_N$. Therefore the lower bound
\eqref{eq:lowerlyap} of Definition \ref{def:lyap} is satisfied.

From \eqref{eq:vnbound}, we conclude that the upper bound condition 
\eqref{eq:upperlyap} of Definition \ref{def:lyap} is satisfied.

By \eqref{eq:feasible} and construction of the warm start \eqref{eq:warm} we have that
$z^+ \in \mathcal{Z}_N$, so that $\mathcal{Z}_N$ is positive invariant for Algorithm \ref{alg:suboptimal}.

As is standard in MPC analysis, we have for all $z \in \mathcal{Z}_N$ that 
\begin{equation*}
%\label{eq:standard}
V_N(z^+) \leq V_N(x,\mathbf{u}) - \ell(x,u(0))
\end{equation*} 
due to equations \eqref{eq:warm}, \eqref{eq:improved} and Assumption
\ref{as:terminalstability}.
From Assumption \ref{as:pdstagecost} we can write 
\begin{equation*}
V_N(z^+) \leq V_N(x,\mathbf{u}) - \alpha_\ell(\abs{x,u(0)})
\end{equation*}
Suppose $x \notin r\mathbb{B}$. Then we have $\tilde{\mathbf{u}} \in \tilde{\mathcal{U}}_r(x)$ and from Lemma 
\ref{lem:utox} we have that 
\begin{equation*}
\abs{(x,\tilde{\mathbf{u}})} \leq \abs{x} + \abs{\tilde{\mathbf{u}}} \leq \abs{x} + \alpha_r(\abs{x}) := 
\alpha_{r*}(\abs{x}) \leq \alpha_{r*}(\abs{(x,u(0))})
\end{equation*}
Consequently,
\begin{equation*}
\alpha_\ell(\alpha_{r*}^{-1}(\abs{(x,\tilde{\mathbf{u}})})) \leq  \alpha_\ell(\abs{(x,u(0))})
\end{equation*}
Defining $\alpha_3(\cdot) = \alpha_\ell(\alpha_{r*}^{-1}(\cdot))$ and using \eqref{eq:improved}, we have
that
\begin{equation*}
V_N(z^+) \leq V_N(x,\mathbf{u}) - \alpha_3(\abs{z}) \leq V_N(z) - \alpha_3(\abs{z})
\end{equation*}
Now suppose $x \in r\mathbb{B}$. Consider the case $(1/2)\underline{\alpha}
(\abs{(x,\tilde{\mathbf{u}})}) \leq V_f(x)$, with $\underline{\alpha}(\cdot)$ from \eqref{eq:lowerbound}.
In this case, we have that
\begin{equation*}
\abs{(x,\tilde{\mathbf{u}})} \leq \underline{\alpha}^{-1}(2V_f(x)) \leq 
\underline{\alpha}^{-1}(2\alpha_f(\abs{x}))) \leq \underline{\alpha}^{-1}(2\alpha_f(\abs{(x,u(0))}))
\end{equation*}
We conclude that 
\begin{equation*}
V_N(z^+) \leq V_N(z) - \alpha_4(\abs{z})
\end{equation*}
where $\alpha_4(\cdot) = \alpha_\ell(\underline{\alpha}^{-1}(2\alpha_f(\cdot))$.

Now consider the case that $(1/2)\underline{\alpha}
(\abs{(x,\tilde{\mathbf{u}})}) > V_f(x)$. In this case, we have that
\begin{equation*}
V_N(x,\mathbf{u}) \leq V_f(x) < (1/2)\underline{\alpha}(\abs{(x,\tilde{\mathbf{u}})})
\leq V_N(x,\tilde{\mathbf{u}}) - (1/2)\underline{\alpha}(\abs{(x,\tilde{\mathbf{u}})})
\end{equation*}
Therefore
\begin{equation*}
V_N(z^+) \leq V_N(x,\mathbf{u}) \leq V_N(z) - \alpha_5(\abs{z})
\end{equation*}
with $\alpha_5(\cdot) = (1/2)\underline{\alpha}(\cdot)$.

%From \eqref{eq:terminal} and \eqref{eq:vfbound}, we have
%that $V_N(x,\mathbf{u}) \leq V_f(x) \leq \alpha_f(\abs{x})$. 
Defining $\alpha(\cdot) = \min\{\alpha_3(\cdot),\alpha_4(\cdot),\alpha_5(\cdot)\}$, we conclude 
that
\begin{equation*}
V_N(z^+) \leq V_N(z) - \alpha(\abs{z})
\end{equation*}
for all $z \in \mathcal{Z}_N$ and $z^+ \in H(z)$. 
%Finally, note that the set $\mathcal{Z}_r$ is
%positive invariant for the system \eqref{eq:nominc} because of \eqref{eq:warm} and \eqref{eq:feasible} --
%\eqref{eq:terminal}. 
We conclude that $V_N(z)$ is a Lyapunov function for 
\eqref{eq:nominc} in $\mathcal{Z}_N$. Asymptotic stability in $z$ follows directly from Lemma \ref{lem:stab}.
Finally, because for the initial warm start \eqref{eq:terminal} is enforced, we have that 
for all $k \in \mathbb{I}_{\geq 0}$
\begin{equation*}
\abs{\phi(k;z)} \leq \abs{\psi(k;z)} \leq \beta(\abs{z},k) \leq \overline{\beta}(\abs{x},k)
\end{equation*}
where $\overline{\beta}(s,k) := \beta(\alpha_{r*}(s),k)$.
\end{proof}

\section{Robust asymptotic stability of suboptimal MPC}
\subsection{Disturbances and robust stability definitions}
For robustness analysis, we consider the following modification to \eqref{eq:system}
\begin{equation}
\label{eq:disturbed}
x_m^+ = f(x_m-e,u) + d + e^+
\end{equation}
where $x_m = x+e$ is the measured state and $d$ is an additive process disturbance.
Note that the nominal suboptimal control 
law is now calculated based on the measured state $u = \kappa_N(x_m,\tilde{\mathbf{u}})$.

For any $x_m \in \mathcal{X}_N$, we formally replace conditions \eqref{eq:feasible} -- 
\eqref{eq:terminal} with the following 
\begin{align}
\label{eq:pfeasible}
\mathbf{u} &\in \mathcal{U}_N(x_m) \\
\label{eq:pimproved}
V_N(x_m,\mathbf{u}) &\leq V_N(x_m,\tilde{\mathbf{u}}) \\
\label{eq:pterminal}
V_N(x_m,\mathbf{u}) &\leq V_f(x_m) \ \text{if} \ x_m \in r\mathbb{B}
\end{align}
and the controller's resulting feasible set is $\mathcal{U}_r(x_m,\tilde{\mathbf{u}})$. 
As before, the suboptimal control law can select any member of this set, $\kappa_N(x_m,\tilde{\mathbf{u}}) 
\in \mathcal{U}_r(x_m,\tilde{\mathbf{u}})$. 

Since the algorithm now begins with a measured state and warm start pair, we redefine the extended state by
\begin{equation*}
z_m = (x_m,\tilde{\mathbf{u}})
\end{equation*}
The procedure to generate the next warm start is again based on the measured state
\begin{equation}
\label{eq:pwarm}
\tilde{\mathbf{u}} = \{u(1),u(2),\dots,u(N-1),\kappa_f(\tilde{\phi}(N;x_m,\mathbf{u}))\}
%\label{eq:pwarm2}
%\tilde{\mathbf{u}} &= \{\kappa_f(\tilde{x}^+),\kappa_f(f(\tilde{x}^+,\kappa_f(\tilde{x}^+))),\dots\}
\end{equation}
%where $\tilde{x}^+$ is the predicted successor state based on measurement $x_m$, i.e.,
%$\tilde{x}^+ = f(x_m,u(0))$.

The perturbed extended system then evolves as
\begin{equation}
\label{eq:perturbedinc}
z_m^+ \in H_{ed}(z_m) = \{(x_m^+,\tilde{\mathbf{u}}^+) \ | \ x_m^+ = f(x_m-e,u(0))+d+e^+, 
\tilde{\mathbf{u}}^+ = \zeta(x_m,\mathbf{u}), \mathbf{u} \in \mathcal{U}_r(z_m)\}
\end{equation}
where $\zeta(\cdot)$ is the mapping corresponding to \eqref{eq:pwarm} 
%and $ed$ denotes the fact that
%arbitrary elements of the disturbance sequences $\mathbf{d}$ and $\mathbf{e}$ are selected at each point in time
.
Finally, for given disturbance and measurement error sequences $\mathbf{d}$ and $\mathbf{e}$
and initial extended state $z_m = (x_m,\tilde{\mathbf{u}})$ we denote an arbitrary solution of 
\eqref{eq:perturbedinc}
by $\psi_{ed}(k;z_m)$ and the corresponding state by $\phi_{ed}(k;z_m)$.

We consider the following stability definition.
\begin{definition}[Robust Asymptotic Stability]
\label{def:ras}
The origin of the closed-loop system \eqref{eq:perturbedinc} is robustly asymptotically stable
(RAS) on $\mathcal{C}$
if there exists $\beta \in \mathcal{KL}$ and $\sigma_d,\sigma_e \in \mathcal{K}$ such that
for each $x_m \in \mathcal{C}$, for all $\tilde{\mathbf{u}} \in \tilde{\mathcal{U}}_r(x_m)$ and for all disturbance sequences $\mathbf{d}$ and $\mathbf{e}$
satisfying $\norm{\mathbf{d}} \leq \delta$ and $\norm{\mathbf{e}} \leq \delta$, $\delta > 0$, we have that
\begin{equation}
\label{eq:ras}
%\abs{\phi_{ed}(k;x,\tilde{\mathbf{u}},\mathbf{d}_i,\mathbf{e}_i)} \leq \beta(\abs{x},k) + \sigma_d(\norm{
%\mathbf{d}_{i-1}})+ \sigma_e(\norm{\mathbf{e}_i})
\abs{\phi_{ed}(k;x_m,\tilde{\mathbf{u}})} \leq \beta(\abs{x_m},k) + \sigma_d(\norm{
\mathbf{d}_{k-1}})+ \sigma_e(\norm{\mathbf{e}_k})
\end{equation}
for all $k \in \mathbb{I}_{\geq 0}$.
\end{definition}



\begin{remark}[Analysis and Theory]
\label{rm:analysis}
First, we describe the differences in results between the robust stability results of \cite{pannocchia:rawlings:wright:2011}, Theorem 41, and the results here, Theorem \ref{th:mainiss}. In \cite{pannocchia:rawlings:wright:2011}, SRES is proven for the set $\overline{\mathcal{C}}_\rho$ for an MPC formulation with the
terminal penalty multiplied by $\beta$. The set $\overline{\mathcal{C}}_\rho$ is the set of all states such
that when measurement errors are bounded by $\rho$, there exists an input such that $V_N(x_m,\mathbf{u})$ 
is upper bounded by $\overline{V}$ for all measured states. Note that $\beta = \overline{V}/\alpha$, and
that $\overline{\mathcal{C}}_\rho$ is compact. In
the limit as $\overline{V} \rightarrow \infty$ (which implies $\beta \rightarrow \infty$), and in the limit as $\rho \rightarrow 0$, it is stated (Remark 42) that $\overline{\mathcal{C}}_\rho$ approaches 
$\mathcal{X}_N$.

In this paper, Theorem \ref{th:mainiss} states that for MPC, without altering the terminal penalty, the
same robust stability result holds over any compact subset of $\mathcal{X}_N$ (or all of $\mathcal{X}_N$,
if $f(\cdot)$ is $\mathcal{K}$ continuous, although I have not proved this yet).

Next, we discuss the differences in the analysis.
One major difference between the analysis here and the analysis of \cite{pannocchia:rawlings:wright:2011}
is that here we consider the evolution of measured state and warm start inputs $z_m = (x_m,\tilde{\mathbf{u}})$ as
opposed to state and input pairs. This change is due to the fact that the algorithm is initialized with
a measured state and a warm start. This also places more emphasis on how the warm start evolves;
\cite{pannocchia:rawlings:wright:2011} does not discuss any warm start construction that would 
satisfy \eqref{eq:pterminal}, whereas it is presented in \eqref{eq:altwarm} here.

Another major change is that SRES and RES have been replaced with (essentially) input to state stability
in Definition \ref{def:ras}. I believe this simplifies the proofs.

The most significant change, however, is the claim in the robust feasibility section of the proof of 
Theorem \ref{th:mainiss}. Specifically, we argue that the terminal control law brings the terminal state
to the interior of the terminal set at the next time step; for disturbances small enough, the resultant
warm start input sequence still causes the terminal state based on the new measurement to be 
in the interior of the terminal set. This simplifies both the proof of ISS and the set in which 
robust stability holds. 
\end{remark}

We modify the algorithm as follows.
\begin{algorithm}[Suboptimal MPC with disturbances] .\\
\label{alg:psuboptimal}
\begin{itemize}
\item Construct $\mathbb{X}_f$ and $V_f(\cdot)$ satisfying Assumption \ref{as:terminalstability}
%\item Select $r > 0$ and $\overline{r} > 0 such that $r\mathbb{B} \subset \overline{r}\mathbb{B} \subset \mathbb{X}_f$
\item Select $r > 0$ such that $r\mathbb{B} \subset \mathbb{X}_f$
\item Provide initial state $x_m(0) \in \mathcal{X}_N$ 
and any initial warm start $\tilde{\mathbf{u}}(0) \in \tilde{\mathcal{U}}_r(x_m(0))$.
\item Repeat for $k = 0,1,\dots$
\begin{enumerate}
\item Measure current state $x_m(k)$
\item Compute any input $\mathbf{u} \in \mathcal{U}_r(x(k),\tilde{\mathbf{u}}(k))$
\item Compute the next warm start $\tilde{\mathbf{u}}(k+1)$ according to \eqref{eq:pwarm}
\item Inject the first element of the input sequence $\mathbf{u}$ and set $k \leftarrow k+1$  
\end{enumerate}
\end{itemize}
\end{algorithm}

\subsection{Main results}

\begin{remark}[Open Issues]
\label{rm:issues}
The most significant open issue is that of uniform continuity. In the paper \cite{pannocchia:rawlings:wright:2011} there is actually an error in Lemma 30.
The issue comes from statements such as (proof of Lemma 30, slightly paraphrased)

``By continuity of $V_N(\cdot)$, choose $\delta_2 > 0$ such that the condition
\begin{equation*}
V_N(x_m,\mathbf{u}) \leq V_N(x,\mathbf{u}) + \frac{\rho}{3}
\end{equation*}
holds for all $(z_m,e) \in \mathcal{Z}_r \times \delta_2\mathbb{B}$.''

The problem is that this is a statement of uniform continuity, not continuity, because for continuity
$\delta_2$ would be a function of $z_m$. This issue does not appear in either Theorem 31 or Theorem 41,
however, because the definitions of RES and SRES involve only compact sets.

There are two ways around this issue. The first is to restrict robust stability to hold only in an 
arbitrarily large compact subset of $\mathcal{X}_N$. This approach works because it is known 
(``Robust Nonlinear Control Design: State-Space and Lyapunov Techniques'' freeman:kokotovic:2008 
not in .bib database yet) that a continuous function on a compact set is both uniformly continuous 
and $\mathcal{K}$ continuous. 

The second approach is to add an assumption such as the following: either (1) $\mathcal{X}_N$ is 
bounded or (2) $f(\cdot)$ is $\mathcal{K}$ continuous on (an arbitrarily small expansion of) $\mathcal{X}_N$. Note that uniform continuity is
not equivalent to $\mathcal{K}$ continuity on unbounded sets (due to freeman:kokotovic:2008).

%I've kept the compactness requirements in the results for now because 
%I'm not actually sure they can be removed.
%Consider the system $x^+ = f(x+d)$ and nominal Lyapunov function $V(x^+) - V(x) \leq -\alpha(x)$, $\alpha
%\in \mathcal{K}$. 
%As $\abs{x} \rightarrow \infty$, it is true that $\alpha(x) \rightarrow \infty$. But it is also true that
%$\abs{x_{nominal}^+-x_{disturbed}^+} \rightarrow \infty$ (and therefore 
%$\abs{V(x_{nominal}^+)-V(x_{disturbed}^+)} \rightarrow \infty$)
%for a fixed $d$, and I'm not sure how to demonstrate that one dominates the other.

Also, I was incorrect earlier about needing the condition \ref{eq:pterminal} to hold only 
at the initial state;
Lemma \ref{lem:annoying} requires that $V_N$ be a nominal Lyapunov function, which in turn requires 
condition \eqref{eq:pterminal}. Since this lemma is used at every time step, condition \ref{eq:pterminal}
must be enforced at each step.

Finally, it would also shorten things up considerably to define (the equivalent of) an ISS Lyapunov function; 
we could then remove (almost) the entire robust stability section of the proof of 
Theorem \ref{th:mainiss} by using the fact that an ISS Lyapunov function implies ISS.

\end{remark}

\begin{lemma}
\label{lem:annoying}
Let $\mathcal{C}$ be any compact subset of $\mathcal{X}_N$. Then
there exists a $\mathcal{K}_\infty$ 
function $\eta(\cdot)$ and $\mathcal{K}$ functions $\sigma_d(\cdot)$ and $\sigma_e(\cdot)$ such that for all
$z_m \in \mathcal{C} \times \mathcal{U}_N(x_m)$ such that $z_m^+ \in \mathcal{C} \times \mathcal{U}_N(x_m)$ we have
\begin{equation*}
\max_{z_m^+ \in H_{ed}(z_m)} V_N(z_m^+) \leq \max\{V_N(z_m)-\eta(\abs{z_m}),\sigma_d(\abs{d})+
\sigma_e(\abs{(e,e^+)})\}
\end{equation*}
\end{lemma}
\begin{proof}
Let $d,e$ and $e^+$ be arbitrary disturbance and measurement errors.

First, note that from the proof of Theorem \ref{th:nominalstability},
$V_N(\cdot)$ is a Lyapunov function of the nominal system on $\mathcal{Z}_N$.
Therefore
\begin{equation*}
V_N(\tilde{x}^+,\tilde{\mathbf{u}}^+) \leq V_N(z_m) - \alpha_3(\abs{z_m})
\end{equation*}
where $\tilde{x}^+ = f(x_m,u(0))$ is the predicted successor state.

%For all $x_m \in \mathbb{R}^n$, $\mathcal{C} \times \mathcal{U}_N(x_m)$ is compact  
%because it is a subset of the compact set $\mathcal{C} \times \mathbb{U}$ and $\mathcal{U}_N(x_m)$ is closed.
%Let $\rho > 0$ be arbitrary.
By the continuity of $V_N(\cdot)$ and $f(\cdot)$ and compactness of $\mathcal{C} \times \mathbb{U}$,
there exists $\sigma_d'(\cdot)$ and $\sigma_e'(\cdot)$ such that for all $(\tilde{x}^+,\tilde{\mathbf{u}}^+) \in 
\mathcal{C} \times \mathbb{U}$
\begin{equation}
\label{eq:ed1}
V_N(x_m^+,\tilde{\mathbf{u}}^+) \leq V_N(\tilde{x}^+,\tilde{\mathbf{u}}^+) + \sigma_d'(\abs{d}) + \sigma_e'(\abs{
(e,e^+)})
\end{equation}
Therefore, for all $z_m \in \mathcal{C} \times \mathcal{U}_N(x_m)$, we have
\begin{equation}
\label{eq:ed2}
V_N(z_m^+) \leq V_N(z_m) - \alpha_3(\abs{z_m}) + \sigma_d'(\abs{d}) + \sigma_e'(\abs{(e,e^+)})
\end{equation}

%Now, we develop an upper bound of $V_N(z_m) - \alpha_3(\abs{z_m})$ from \eqref{eq:upperlyap} as
%\begin{equation}
%\label{eq:ed3}
%V_N(z_m) - \alpha_3(\abs{z_m}) \leq \alpha_2(\abs{z_m}) - \alpha_3(\abs{z_m})
%\end{equation}

Define $\mathcal{Z}_1 = \{z_m \in \mathcal{Z}_N \mid \alpha_3(\abs{z_m}) > 2\sigma_d'(\abs{d}) + 2\sigma_e'(\abs{(e,e^+)}) \}$ and 
$\mathcal{Z}_2 = \{z_m \in \mathcal{Z}_N \mid \alpha_3(\abs{z_m}) \leq 2\sigma_d'(\abs{d}) + 2\sigma_e'(\abs{(e,e^+)}) \}$. 
%Define also $\sigma_d(\cdot) = 2\sigma_d'(\cdot)$, $\sigma_e(\cdot) = 2\sigma_e'(\cdot)$ 
%and $\eta(\abs{z_m}) = (1/2)\alpha_3(\abs{z_m})$.

Suppose $z_m \in \mathcal{Z}_1$.  We have that
\begin{align*}
(1/2)\alpha_3(\abs{z_m}) - \sigma_d'(\abs{d}) - \sigma_e'(\abs{(e,e^+)}) &> 0\\
\alpha_3(\abs{z_m}) - \sigma_d'(\abs{d}) - \sigma_e'(\abs{(e,e^+)}) &> (1/2)\alpha_3(\abs{z_m})
\end{align*}
Therefore, defining $\eta(\cdot) = (1/2)\alpha_3(\cdot)$, we have
\begin{equation*}
%\label{eq:ed4}
V_N(z_m^+) \leq V_N(z_m) - \eta(\abs{z_m})
\end{equation*}

Now suppose $z_m \in \mathcal{Z}_2$. We have that $\abs{z_m} \leq \alpha_3^{-1}(2\sigma_d'(\abs{d}) + 2\sigma_e'(\abs{(e,e^+)}))$. By \eqref{eq:upperlyap}, we have 
\begin{equation*}
V_N(z_m) \leq \alpha_2(\abs{z_m}) \leq \alpha_2(\alpha_3^{-1}(2\sigma_d'(\abs{d}) + 2\sigma_e'(\abs{(e,e^+)})))
\end{equation*}
Defining $\sigma^*(\cdot) = \alpha_2(\alpha_3^{-1}(2\sigma_d'(\cdot) + 2\sigma_e'(\cdot)))$, we have
\begin{align*}
%\label{eq:ed5}
V_N(z_m) -\alpha_3(\abs{z_m}) + \sigma_d'(\abs{d}) + \sigma_e'(\abs{(e,e^+)}) &\leq 
V_N(z_m) + \sigma_d'(\abs{d}) + \sigma_e'(\abs{(e,e^+)}) \\
&\leq \sigma^*(\abs{d})+\sigma^*(\abs{(e,e^+)}) + \sigma_d'(\abs{d}) + \sigma_e'(\abs{(e,e^+)}) \\
&\leq \sigma_d(\abs{d}) + \sigma_e(\abs{(e,e^+)}) 
\end{align*}
where $\sigma_d(\cdot) := \sigma^*(\cdot)+\sigma_d'(\cdot)$ and 
$\sigma_e(\cdot) := \sigma^*(\cdot)+\sigma_e'(\cdot)$ 
are both of class $\mathcal{K}$.

Since the above arguments hold for all $z \in \mathcal{Z}_N$, we conclude that 
\begin{equation*}
\max_{z_m^+ \in H_{ed}(z_m)} V_N(z_m^+) \leq \max\{V_N(z_m)-\eta(\abs{z_m}),\sigma_d(\abs{d})+
\sigma_e(\abs{(e,e^+)})\}
\end{equation*}
and we note that $\eta(\cdot)$ does not depend on $(d,e,e^+)$.
\end{proof}

The main RAS result of this paper is as follows.

\begin{theorem}[Robust Asymptotic Stability]
\label{th:mainiss}
Under Assumptions \ref{as:continuity} -- \ref{as:pdstagecost}, there exists
$\delta > 0$ such that the origin of the perturbed closed-loop system 
\eqref{eq:perturbedinc} is RAS on any compact subset, $\mathcal{C}$, of ${X}_N$ for 
%any warm start satisfying \eqref{eq:pfeasible} and \eqref{eq:pterminal} and 
any disturbance sequences
satisfying
\begin{equation*}
%\label{eq:fail}
\norm{\mathbf{d}} \leq \delta, \quad \norm{\mathbf{e}} \leq \delta
\end{equation*}
\end{theorem}
\begin{proof}
(\emph{Robust Feasibility}). 
FIXME: WRONG, the set must be positive invariant to start with

First, we expand $\mathcal{C}$ so that
it is positively invariant. We define
\begin{equation*}
\mathcal{C}_\rho = \{x_m \mid \exists \mathbf{u} \ \text{such that} \ V_N(x_m,\mathbf{u}) \leq \rho \}
\end{equation*}
and note that $\mathcal{C}_\rho$ is compact and $\rho > 0$ can be chosen large enough so that 
$\mathcal{C} \subseteq \mathcal{C}_\rho$. 
Since $V_N(\cdot)$ is a Lyapunov function of the nominal system on $\mathcal{Z}_N$, we have
\begin{equation*}
V_N(\tilde{x}^+,\tilde{\mathbf{u}}^+) \leq V_N(z_m) - \alpha_3(\abs{z_m})
\end{equation*}

First, suppose $V_N(z_m) \leq \rho/2$. Then $V_N(\tilde{x}^+,\tilde{\mathbf{u}}^+) \leq V_N(z_m) \leq \rho/2$. 
Because $\mathcal{C}_\rho \times \mathbb{U}^N$ is compact, we can choose $\delta_1 > 0$ such that 
$V_N(z_m^+) \leq V_N(\tilde{x}^+,\tilde{\mathbf{u}}^+) + \rho/2$ for all $(e,d,e^+) \in \delta_1\mathbb{B}$.
Therefore $V_N(z_m^+) \leq \rho$ and $x_m^+ \in \mathcal{C}_\rho$.

Next, suppose $V_N(z_m) > \rho/2$. Then by
\eqref{eq:vnbound}, we have $\abs{z_m^+} > \alpha_V^{-1}(\rho/2)$ and therefore 
$\alpha_3(\abs{z_m^+}) > \alpha_3(\alpha_V^{-1}(\rho/2)) > 0$. We can then choose $\delta_2 > 0$ such that
$V_N(z_m^+) \leq V_N(\tilde{x}^+,\tilde{\mathbf{u}}^+) + \alpha_3(\alpha_V^{-1}(\rho/2))$
for all $(e,d,e^+) \in \delta_2\mathbb{B}$. Therefore $V_N(z_m^+) \leq \rho$ and $x_m^+ \in \mathcal{C}_\rho$.
We conclude that $\mathcal{C}_\rho$ is positive invariant.

Next, we establish that there exists a $\delta_3 > 0$ such that for all 
$x_m \in \mathcal{C}_\rho$, $\tilde{\mathbf{u}} \in \mathcal{U}_N(x_m)$ and $(e,d,e^+) \in  \times \delta\mathbb{B}^3$, 
we have $x_m^+ \in \mathcal{X}_N$,
$x^+ \in \mathcal{X}_N$ and $z_m^+ \in \mathcal{Z}_N$. 

Since $z_m = (x_m,\tilde{\mathbf{u}}) \in \mathcal{Z}_N$, the suboptimal controller is well defined and produces 
an input sequence $\mathbf{u}$ satisfying \eqref{eq:pfeasible} -- \eqref{eq:pterminal}. 
Consider next the warm start $\tilde{\mathbf{u}}^+$ generated from the input sequence $\mathbf{u}$ using 
\eqref{eq:pwarm}.

The key step of the proof is as follows: we argue that the predicted terminal state $\tilde{\phi}^N :=
\tilde{\phi}(N;\tilde{x}^+,\tilde{\mathbf{u}}^+)$ is bounded away from the boundary of $\mathbb{X}_f$. The argument
is depicted graphically in Figure \ref{fig:feasibility}.
\begin{figure}
\scalebox{.8}{\input{feasibility}}
\caption{The terminal control law computed based on the predicted successor state $\tilde{x}^+$ takes the system
to the interior of the terminal set. By uniform continuity, the actual and measured trajectories reach the interior as well.}
\label{fig:feasibility}
\end{figure} 
First, consider the point 
$\tilde{\phi}^{N-1} := \tilde{\phi}(N-1;\tilde{x}^+,\tilde{\mathbf{u}}^+)$, which belongs to $\mathbb{X}_f$ 
because $\mathbf{u}$ satisfied the terminal constraint for $x_m$. From Assumption \ref{as:terminalstability}, 
we have that 
\begin{align*}
V_f(\tilde{\phi}^N) - V_f(\tilde{\phi}^{N-1}) &\leq - 
\ell(\tilde{\phi}^{N-1},\kappa_f(\tilde{\phi}^{N-1})) \\
&\leq - \alpha_\ell(\abs{(\tilde{\phi}^{N-1},\kappa_f(\tilde{\phi}^{N-1}))}) \\
&\leq - \alpha_\ell(\abs{\tilde{\phi}^{N-1}})
%V_f(\phi(N;\tilde{x}^+,\tilde{\mathbf{u}}^+)) - V_f(\phi(N-1;\tilde{x}^+,\tilde{\mathbf{u}}^+)) &\leq - 
%\ell(\phi(N-1;\tilde{x}^+,\tilde{\mathbf{u}}^+),k_f(\phi(N-1;\tilde{x}^+,\tilde{\mathbf{u}}^+)) \\
%&\leq - \alpha_\ell(\abs{(\phi(N-1;\tilde{x}^+,\tilde{\mathbf{u}}^+),k_f(\phi(N-1;\tilde{x}^+,\tilde{\mathbf{u}}^+)))}) \\
%&\leq - \alpha_\ell(\abs{\phi(N-1;\tilde{x}^+,\tilde{\mathbf{u}}^+)})
\end{align*}

Suppose first that $V_f(\tilde{\phi}^{N-1}) > \alpha/2$. By \eqref{eq:vfbound}, we have $\abs{\tilde{\phi}^{N-1}} > 
\alpha_f^{-1}(\alpha/2)$. Therefore, we have that
$V_f(\tilde{\phi}^N) - V_f(\tilde{\phi}^{N-1}) \leq - \alpha_\ell(\alpha_f^{-1}(\alpha/2))$ which implies
$V_f(\tilde{\phi}^N)  \leq \alpha - \alpha_\ell(\alpha_f^{-1}(\alpha/2))$. 

Now suppose $V_f(\tilde{\phi}^{N-1}) \leq \alpha/2$. Since
$V_f(\tilde{\phi}^N) - V_f(\tilde{\phi}^{N-1}) \leq 0$
we have immediately that
$V_f(\tilde{\phi}^{N-1}) \leq \alpha/2$.
We conclude that 
\begin{equation*}
V_f(\tilde{\phi}^N) \leq \alpha - \gamma
\end{equation*}
where $\gamma = \min\{\alpha/2,\alpha_\ell(\alpha_f^{-1}(\alpha/2))\}$ and we note that $\gamma$ is independent of
$(z_m,e,d,e^+)$.

We now consider the actual and measured successor states, $x^+ = f(x,u(0;x_m))+d$ and $x_m^+ = f(x_m-e,u(0))+d+e^+$
, respectively. Because $f(\cdot)$ and $V_f(\cdot)$ are continuous by Assumption \ref{as:continuity} and
$\mathcal{C}_\rho \times \mathbb{U}$ is compact, we have that $f(\cdot)$ and $V_f(\cdot)$ are uniformly continuous on 
$\mathcal{C}_\rho \times \mathbb{U}$ and $\mathcal{C}_\rho$, respectively.
Therefore, for any $\rho > 0$ we can select $\delta_3 > 0$ such that for all 
$x_m \in \mathcal{C}_\rho$, $\tilde{\mathbf{u}} \in \mathcal{U}_N(x_m)$ and $(e,d,e^+) \in  \times \delta_3\mathbb{B}^3$, 
we have 
\begin{align*}
V_f(\tilde{\phi}(N;x^+,\tilde{\mathbf{u}}^+))  &\leq V_f(\tilde{\phi}^N) + \rho \\
V_f(\tilde{\phi}(N;x_m^+,\tilde{\mathbf{u}}^+))  &\leq V_f(\tilde{\phi}^N) + \rho
\end{align*}
By selecting $\rho = \gamma$, we have $\tilde{\phi}(N;x^+,\tilde{\mathbf{u}}^+) \in \mathcal{X}_f$ and 
$\tilde{\phi}(N;x_m^+,\tilde{\mathbf{u}}^+) \in \mathcal{X}_f$. Since $\tilde{\mathbf{u}}^+ \in \mathbb{U}^N$ and there are
no other state constraints, we have that $\tilde{\mathbf{u}}^+ \in \mathcal{U}_N(x^+)$ and 
$\tilde{\mathbf{u}}^+ \in \mathcal{U}_N(x_m^+)$. As a result, $x^+ \in \mathcal{X}_N$, $x_m^+ \in \mathcal{X}_N$,
and $z_m^+ \in \mathcal{Z}_N$.

% below is too similar
(\emph{Robust Stability}) 
%We define $\mathcal{C}_\xi = \{$
From Lemma \ref{lem:annoying},
for any $z_m^+ \in H_{ed}(z_m)$ we have that
\begin{equation*}
V_N(z_m^+) \leq \max\{V_N(z_m)-\eta(\abs{z_m}),\sigma_d(\abs{d})+
\sigma_e(\abs{(e,e^+)})\}
\end{equation*}
Using steps identical to the proof of Lemma \ref{lem:stab}, we have 
\begin{equation*}
V_N(z_m^+) \leq \max\{\sigma(V_N(z_m)),\sigma_d(\abs{d})+\sigma_e(\abs{(e,e^+)})\}
\end{equation*}
for $\sigma(s) = (1/2)(s+\left(\max_{s' \in [0,s]}s'-\eta \circ \alpha_2^{-1}(s')\right))$, where 
$0 < \sigma(s) < s$ for $s > 0$ and $\sigma \in \mathcal{K}_\infty$. Since 
$\mathcal{C}_\rho$ is positive invariant, we can apply this inequality $k-1$ times to obtain
\begin{equation*}
V_N(\psi(k;z_m)) \leq \max\{\sigma^k(V_N(z_m)),\sigma_d(\norm{\mathbf{d}_{k-1}})+\sigma_e(\norm{\mathbf{e}_k})\}
\end{equation*}
From \eqref{eq:lowerlyap}
\begin{align*}
\abs{\psi(k;z_m)} &\leq \max\{\alpha_1^{-1}(\sigma^k(V_N(z_m))),\alpha_1^{-1}(\sigma_d(\norm{\mathbf{d}_{k-1}})+\sigma_e(\norm{\mathbf{e}_k}))\} \\
&\leq \max\{\beta(\abs{z_m},k),\alpha_1^{-1}(\sigma_d(\norm{\mathbf{d}_{k-1}})+\sigma_e(\norm{\mathbf{e}_k}))\}
\end{align*}
Since the initial extended state satisfies condition \eqref{eq:pterminal} by assumption, we invoke
Lemma \ref{lem:utox} to conclude that $\tilde{\mathbf{u}} \leq \alpha_r(x_m)$. Therefore
\begin{align*}
\abs{\phi(k;z_m)} \leq \abs{\psi(k;z_m)} &\leq \max\{\beta(\abs{z_m},k),\alpha_1^{-1}(\sigma_d(\norm{\mathbf{d}_{k-1}})+\sigma_e(\norm{\mathbf{e}_k}))\} \\
&\leq \max\{\overline{\beta}(\abs{x_m},k),\alpha_1^{-1}(\sigma_d(\norm{\mathbf{d}_{k-1}})+\sigma_e(\norm{\mathbf{e}_k}))\}
\end{align*}
where $\overline{\beta}(s,k) := \beta(s+\alpha_r(s),k)$. Defining $\sigma_{de}(\cdot) :=
\alpha_1^{-1}(\sigma_d(\cdot)+\sigma_e(\cdot)) \in \mathcal{K}$, we have
\begin{align*}
\abs{\phi(k;z_m)} &\leq \max\{\overline{\beta}(\abs{x_m},k),\sigma_{de}(\norm{\mathbf{d}_{k-1}})+\sigma_{de}(\norm{\mathbf{e}_{k}})\} \\
&\leq \overline{\beta}(\abs{x_m},k)+\sigma_{de}(\norm{\mathbf{d}_{k-1}})+\sigma_{de}(\norm{\mathbf{e}_{k}})
\end{align*}
We conclude the proof by selecting $\delta = \min \{\delta_1,\delta_2,\delta_3\}$ and noting that
$\delta$ is independent of $z_m$.

\end{proof}
Before continuing, we note that because conditions \eqref{eq:pfeasible} -- \eqref{eq:pterminal} are based on the
measured state, the results of Proposition \ref{pp:opt} apply, that is, we can compute $\mathbf{u} \in \mathcal{U}_r(z_m)$ without any optimization.
We also note that these results apply immediately to optimal MPC, including the
case when the optimal solution of $\mathbb{P}_N(x)$ is not unique.

\begin{comment}

%FIXME: is $V_\infty$ a storage function (when it exists?)
%FIXME: if so, can the terminal penalty to go be used anyway for removing the
%state constraints

\section{Robust stability in the absence of state constraints}
\label{sec:nostateconstraints}
% FIXME: similar
One method to increase the region of attraction of the origin under process
disturbances is to replace the hard state constraints $x(k) \in \mathbb{X}$
with penalties, thus allowing small state constraint violations. In this 
section we also replace the terminal constraint by adjusting the
terminal penalty by weighting it, $\beta V_f(x(N))$. Unlike the case of 
general state constraints,  however, this does not expand the region of
attraction of the origin because the terminal constraint must still be
exactly satisfied to use the local controller $\kappa_f(x)$ at the end of 
the horizon. When considering the case of no terminal constraints we must
therefore restrict attention to states under which the MPC solution
a priori satisfies the the terminal constraint.

\subsection{Revised assumptions and nominal stability results}
We consider the case of no (hard) state constraints and therefore replace
Assumption \ref{as:constraints} with the following
\begin{assumption}
\label{as:nostateconstraints}
The set $\mathbb{U}$ is compact and contains the origin. The sets 
$\mathbb{X}$ and $\mathbb{X_f}$ are defined by $\mathbb{X} = \mathbb{R}^n$
and $\mathbb{X_f} = \text{lev}_\alpha V_f = \{x \in \mathbb{R}^n | V_f(x) \leq
\alpha\}$, with $\alpha > 0$.
\end{assumption}
Assumption \ref{as:vn0continuous} is no longer necessary because there are
no state constraints to violate, and therefore no feasibility restoration
step is required. We still make use of Assumptions \ref{as:continuity}
, \ref{as:terminalstability} and \ref{as:pdstagecost} with the modified
cost function
\begin{equation}
\label{eq:vnbeta}
V_N^\beta(x,\mathbf{u}) = \sum_{k=0}^{N-1}\ell(\phi(k;x,\mathbf{u})+\beta
V_f(\phi(N;x,\mathbf{u}))
\end{equation}
where $\beta \geq 1$ is a parameter which determines the set of initial
conditions for which the terminal constrain is satisfied a priori. Note that
since $\beta \geq 1$ the pair $(\beta V_f(x),\mathbb{X}_f)$ satisfies 
Assumption \ref{as:terminalstability} whenever $(V_f(x),\mathbb{X}_f)$
does. Given
the same warm start $\tilde{\mathbf{u}}$ as defined in \eqref{eq:warm} and
the successor state $x^+ = f(x,u(0;x))$ we replace the requirement for
suboptimal MPC in \eqref{eq:feasible} -- 
\eqref{eq:terminal} by the conditions
\begin{align}
\label{eq:newfeasible}
&\mathbf{u}^+ \in \mathbb{U}^N \\
\label{eq:newimproved}
&V^\beta_N(x^+,\mathbf{u}^+) \leq V^\beta_N(x^+,\tilde{\mathbf{u}}) \\
\label{eq:newterminal}
&V^\beta_N(x^+,\mathbf{u}^+) \leq \beta V_f(x^+) \quad \text{when} x^+
\in r\mathbb{B}
\end{align}
We still refer to \eqref{eq:inclusion} and \eqref{eq:nominc} for the
evolution of the state and extended state, although for \eqref{eq:nominc}, 
the definition of $G(z)$ must be modified according to 
\eqref{eq:newfeasible} -- \eqref{eq:newterminal}. For a chosen maximal cost
$\overline{V} > 0$, we define the compact sets
\begin{align}
\overline{\mathcal{Z}}_r &= \{(x,\mathbf{u}) \in \mathbb{R}^n \times 
\mathbb{U}^N | V_N^\beta(x,\mathbf{u}) \leq \overline{V}, \text{and}
V_N^\beta(x,\mathbf{u}) \leq \beta V_f(x) \text{if} x \in r\mathbb{B}\} \\
\mathbb{X}_0 &= \{x \in \mathbb{R}^n | \exists \mathbf{u} \in \mathbb{U}^N
\text{such that} (x,\mathbf{u} \in \overline{\mathcal{Z}}_r\}
\end{align}
For the remainder of the paper, we choose
\begin{equation*}
\overline{V} \geq \alpha, \beta \geq \overline{B} := \overline{V}/\alpha
\geq 1
\end{equation*}
where $\alpha > 0$ defines the terminal region from Assumption 
\ref{as:nostateconstraints}.
\begin{remark}
% FIXME: similar
The choice $\beta \geq \overline{\beta}$ implies no trajectories in 
$\overline{\mathcal{Z}}_r$ terminate on the boundary of $\mathbb{X}_f$.
\end{remark}
\begin{proof}
For any trajectory terminating on the boundary of $\mathbb{X}_f$, we have
$V_f(x(N)) = \alpha$ and therefore $\sum^{N-1}_{k=0}\ell(x(k),u(k)) \leq 0$.
By Assumption \ref{as:pdstagecost}, this implies $(x(k),u(k)) = (0,0)$ for
$k \in \mathbb{I}_{0:N-1}$, which combined with Assumption 
\ref{as:terminalstability} implies $(x(N),u(N)) = (0,0)$. Thus, $V_f(x(N))
= 0$, a contradiction.
\end{proof}
We have the following nominal stability results.
\begin{lemma}
\label{lem:betalyap}
$V^\beta_N(z)$ is a Lyapunov function for the extended closed-loop system
\eqref{eq:nominc} on $\overline{\mathcal{Z}}_r$. 
\end{lemma}
\begin{proof}
We can replicate the proof of Lemma \ref{lem:utox} for the new terminal 
penalty and new set $\overline{\mathcal{Z}}_r$ to obtain $\alpha_r \in 
\mathcal{K}_\infty$ such that $\abs{\mathbf{u}} \leq \alpha_r(\abs{x})$
for all $(x,\mathbf{u}) \in \overline{\mathcal{Z}}_r$. Consider the 
difference inclusion $z^+ \in H(z)$ with $H(z)$ defined by \eqref{eq:nominc}
and $G(z)$ appropriately modified for \eqref{eq:newfeasible} -- 
\eqref{eq:newterminal}. We can replicate the proof of Theorem \ref{th:nominalstability}
%Lemma \ref{lem:zlyap}
to obtain
\begin{equation*}
alpha_1(\abs{z}) \leq V^\beta_N(z) \leq \alpha_2(\abs{z}), \max_{z^+ \in H(z)}
V^\beta_N(z^+) \leq V_N^\beta(z) - \alpha_3(\abs{z})
\end{equation*}
for all $z \in \overline{\mathcal{Z}}_r$ for some $\alpha_1,\alpha_2,
\alpha_3 \in \mathcal{K}_\infty$. Hence, $V_N^\beta(z)$ is a Lyapunov function
on $\overline{\mathcal{Z}}_r$ for the closed-loop system \eqref{eq:nominc}.
\end{proof}
\begin{theorem}
Under Assumptions \ref{as:continuity}, \ref{as:terminalstability}, 
\ref{as:pdstagecost} and \ref{as:nostateconstraints}, the origin of the
closed-loop system \eqref{eq:inclusion} is asymptotically stable on 
$\mathcal{X}_0$. 
\end{theorem}
\begin{proof}
First, we show that $\overline{\mathcal{Z}}_r$ is positive invariant
for $z^+ \in H(z)$. Let $z \in \overline{\mathcal{Z}}_r$ and consider
any $z^+ \in H(z)$. From \eqref{eq:newimproved},
\begin{equation*}
V^\beta_N(z^+) \leq V^\beta_N(x^+,\tilde{\mathbf{u}}) \leq V^\beta_N(z) \leq 
\overline{V}
\end{equation*}
Since $\beta \geq \overline{V}/\alpha \geq 1$ and $\ell(cdot)$ is 
nonnegative, we have that $V_f(\phi(N;x^+,\mathbf{u}^+)) \leq \alpha$ and
therefore $\phi(N;x^+,\mathbf{u}^+) \in \mathbb{X}_f$. Along with 
\eqref{eq:newfeasible} and \eqref{eq:newterminal}, it follows that $z^+
\in \overline{\mathcal{Z}}_r$. From Lemmas \ref{lem:betalyap} and 
\ref{lem:stab}, we have that the origin is asymptotically stable, i.e.
for some $\sigma in \mathcal{KL}$, $\abs{\psi(k;z)} \leq \sigma(\abs{z},k)$
for all $z \in \overline{\mathcal{Z}}_r$ and $k \in \mathbb{I}_{\geq 0}$.
Let $\phi(k;x)$ denote the state component of $\psi(k;z)$ for any $x \in 
\mathbb{X}_0$ and any input sequence $\mathbf{u}$ such that $(x,\mathbf{u})
\in \overline{\mathcal{Z}}_r$. Using the fact that $\abs{z} \leq \abs{x} +
\abs{u} \leq \abs{x} + \alpha_r{x} := \alpha_{r*}{x}$, we have
\begin{align*}
\abs{\phi(k;x)} &\leq \abs{\psi(k;z)} \\
&\leq \sigma(\abs{z},k) \\
&\leq \sigma(\alpha_{r*}{x},k) \\
&:= \overline{\sigma}(\abs{x},k)
\end{align*}
for any $x \in \mathbb{X}_0$ and $k \in \mathbb{I}_{\geq 0}$.
\end{proof}

Next, we characterize the admissible set $\mathbb{X}_0$ and its limit
for large $\overline{V}$. We define the set of feasible initial states
that can be taken to the interior of $\mathbb{X}_f$ by
\begin{equation}
\overline{\mathbb{X}}_N = \{x \in \mathbb{R}^n | \exists \mathbf{u} \in 
\mathbb{U}^N \text{such that} \phi(N;x,\mathbf{u}) \in \text{int}
(\mathbb{X}_f)
\end{equation}
% FIXME: this is all too similar
Note that the interior of $\mathbb{X}_f$ is not empty because $\alpha > 0$.
\begin{proposition}
\label{pp:admissibleset}
The admissible set $\mathbb{X}_0$ and restricted feasible set 
$\overline{\mathbb{X}}_N$ satisfy
\begin{equation}
\mathbb{X}_0(\overline{V}) \subset \overline{\mathbb{X}}_N \text{for all}
\overline{V} \geq 0, \text{and} \overline{\mathbb{X}}_N \subset \bigcup_{
\overline{V} \geq 0} \mathbb{X}_0(\overline{V})
\end{equation}
\end{proposition}
\begin{proof}
FIXME: the proof is identical to the other paper and can be omitted\\
The first statement follows directly from the definitions of $\mathbb{X}_0
(\overline{V})$ and $\overline{\mathbb{X}}_N$; $\overline{\mathbb{X}}_N$ is
the set of all states that can be brought to the interior of $\mathbb{X}_f$
with a feasible input, while $\mathbb{X}_0(\overline{V})$ is
the set of all states that can be brought to the interior of $\mathbb{X}_f$
with a feasible input and a cost not exceeding $\overline{V}$. \\
To establish the second statement, we first argue that the sets 
$\mathbb{X}_0(\overline{V})$ are nested: $\overline{V}_2 \geq 
\overline{V}_1$ implies $\mathbb{X}_0(\overline{V_2}) \supseteq 
\mathbb{X}_0(\overline{V_1})$. Assume an arbitrary $x \in \mathbb{X}_0(
\overline{V}_1)$ and correstponding $(x,\mathbf{u}) \in \overline{\mathbb{
Z}}_r(\overline{V}_1)$. We show that $x \in \mathbb{X}_0(\overline{V}_2)$.
Let $\beta_1 = \overline{V}_1/\alpha$, $\beta_2 = \overline{V}_2/\alpha$ and
$x(N) = \phi(N;x,\mathbf{u})$. We have that
\begin{equation*}
V^{\beta_2}_N(x,\mathbf{u}) = V^{\beta_1}_N(x,\mathbf{u}) +(\beta_2-\beta_1)
V_f(x(N))
\end{equation*}
First, notice that if $x \in r\mathbb{B}$, $V_N^{\beta_1}(x,\mathbf{u}) \leq
\beta_1V_f(x)$ and this implies that $V_N^{\beta_2}(x,\mathbf{u}) \leq
\beta_2V_f(x)$ so the inequality is established in $r\mathbb{B}$. Then 
notice that $V_f(x(N)) = \alpha' < \alpha$, which gives
\begin{align*}
V_N^{\beta_2}(x,\mathbf{u}) &\leq \overline{V}_1 + (\overline{V}_2 - 
\overline{V}_1)(\alpha'/\alpha) \\
&= \overline{V}_1(1-\alpha'/\alpha) + \overline{V}_2(\alpha'/\alpha) \\
&\leq \overline{V}_2(1-\alpha'/\alpha) + \overline{V}_2(\alpha'/\alpha) \\
&= \overline{V}_2
\end{align*}
and we conclude $x \in \mathbb{X}_0(\overline{V}_2)$. \\
Next, we establish that for every point $x_0 \in \overline{\mathcal{X}}_N$,
there exists a $\overline{V}_0 > 0$ such that $x_0 \in \mathbb{X}_0(
\overline{V}$ for all $\overline{V} \geq \overline{V}_0$. Take an arbitrary
$x_0 \in \overline{\mathcal{X}}_N$ and a corresponding $\mathbf{u}_0 \in
\mathbb{U}^N$ that satisfies $\phi(N;x_0,\mathbf{u}_0) \in \text{int}
(\mathbb{X}_f)$. If $x_0 \in r\mathbb{B}$, we add the restriction to
$\mathbf{u}_0$ that $V_N(x_0,u_0) \leq V_f(x_0)$. Such a $\mathbf{u}_0$ 
exists because Proposition \ref{pp:opt} establishes that the optimal
input sequence has this property. Since $\beta \geq 1$, it follows that
$V^\beta_N(x_0,\mathbf{u}_0) \leq \beta V_N(x_0,\mathbf{u}_0) \leq \beta V_f
(x_0)$, if $x_0 \in r\mathbb{B}$. Then denote by $\alpha'$ the terminal
cost $\alpha' := V_f(\phi(N;x_0,\mathbf{u}_0))$, and we have that $\alpha'
< \alpha$. Then define $\overline{V}_0 := \frac{1}{1-alpha'/\alpha}\sum_{i=0}^{N-1} \ell(\phi(i;x_0,\mathbf{u}_0),u_0(i))$. A direct computation gives
(FIXME: I don't understand this step) $V^\beta_N(x_0,\mathbf{u}_0) = 
\overline{V}_0$, and, if $x_0 \in r\mathbb{B}$, $V^\beta_N(x_0,\mathbf{u}_0)
\leq \beta V_f(x_0)$. Therefore $x_0 \in \mathbb{X}_0(\overline{V}_0)$, and
by the nesting property $x_0 \in \mathbb{X}_0(\overline{V})$ for all
$\overline{V}$ satisfying $\overline{V} \geq \overline{V}_0$ and the limit
is established.
\end{proof}
This result implies that the limit of $\mathbb{X}_0(\overline{V})$ as
$\overline{V} \rightarrow \infty$ containts all feasible sets corresponding
to an arbitrarily small tightening of the terminal set, i.e. $\mathbb{X}_f
(\alpha') = \text{lev}_{\alpha'}V_f$ with $\alpha' < \alpha$.

\subsection{Robust stability results}

% FIXME: identical
For robustness analysis, we again consider the perturbed closed-loop system
\eqref{eq:disturbed}. As an immediate consequence of removing the 
terminal constraint, we have that the warm start $\tilde{\mathbf{u}}$ is 
feasible for the measured successor state $x_m^+ = x^+ + e^+$ because
$\tilde{\mathbf{u}} \in \mathbb{U}^N$. There is therefore no reason for the
seasibility restoration step \eqref{eq:resto}, making the subsequent
analysis much simpler. We still consider the augmented system of 
\eqref{eq:perturbedinc} but modify the set $G_{ed}(z)$ as follows
\begin{align*}
G_{ed}(z) := &\{\mathbf{u}^+ | \mathbf{u}^+ \in \mathbb{U}^N, V^\beta_N(
x_m^+,\mathbf{u}^+) \leq V^\beta_N(x_m^+,\tilde{\mathbf{u}}), \\
&V^\beta_N(x_m^+,\mathbf{u}^+) \leq \beta V_f(x_m^+) if x^+_m \in r\mathbb{B}\}
\end{align*}
Note that the results of Lemma \ref{lem:annoying} still hold for the 
modified cost $V^\beta_N(\cdot)$ and set $\overline{\mathcal{Z}}_r$.
% FIXME: still identical
We now present a set over which we prove SRAS. Given a scalar $\rho > 0$
and any $z_m \in \overline{\mathcal{Z}}_r$, we define
\begin{align}
\label{eq:restricted}
&\overline{V}^\rho_N(z_m) := \max_{e \in \rho\mathbb{B}} V_N^\beta(z) \text{such
that} z = z_m - (e,0) \\
&\overline{\mathcal{S}}_\rho := \{z_m \in \overline{\mathcal{Z}}_r |
\overline{V}^\rho_N(z_m) \leq \overline{V}\}
\end{align}
where we assume that $\rho$ is small enough so that $\overline{
\mathcal{S}}_\rho$ is nonempty. Finally, the candidate set for SRAS is 
defined as
\begin{equation}
\label{eq:crhonew}
\overline{\mathcal{C}}_\rho := \{ x \in \mathbb{R}^n | x = x_m - e, e \in
\rho\mathbb{B}, \exists \mathbf{u} \text{such that} (x_m,\mathbf{u}) \in
\overline{\mathcal{S}}_\rho\}
\end{equation}
\begin{theorem}
Under Assumptions \ref{as:continuity}, \ref{as:terminalstability},
\ref{as:pdstagecost} and \ref{as:nostateconstraints}, the origin of the
closed-loop system \eqref{eq:disturbed} is SRAS on $\overline{
\mathcal{C}}_\rho$.
\end{theorem}
\begin{proof}
% FIXME: identical
Robust Feasibility. Suppose that $x \in \overline{\mathcal{C}}_\rho$ and
let $z = (x,\mathbf{u})$ be the correstponding augmented state where 
$\mathbf{u}$ is any suboptimal sequence computed for the measured state 
$x_m = x + e, e \in \rho\mathbb{B}$. From \eqref{eq:restricted}, it follows
that $V_N^\beta \leq \overline{V}$ and that $z_m \in \overline{
\mathcal{S}}_\rho \subseteq \overline{\mathcal{Z}}_r$. We define $\tilde{
z}^+ = (\tilde{x}^+,\tilde{\mathbf{u}})$ where $\tilde{x}^+ = f(x_m,u(0;x_m
))$ is the predicted (nominal) successor state. The quantity $V^\beta_N(
\tilde{z}^+)$ is the cost along the nominal evolution from $z_m$. Since
 $V^\beta_N(\cdot)$ is a Lyapunov function for the nominal system on 
$\overline{\mathcal{Z}}_r$ by Lemma \ref{lem:betalyap}, we have that
$V^\beta_N(\tilde{z}^+) \leq V^\beta_N(z_m)-\alpha_3(\abs{z_m})$. \\
We also have that $V^\beta_N(z) \leq \alpha_2(\abs{z})$ for $z \in 
\overline{\mathcal{Z}}_r$. Therefore, we can select $\gamma > 0$ such that 
$max_{\abs{z} \leq \gamma} V^\beta_N(z) \leq \overline{V}_\gamma < \overline{V}$.
For $z$ such that $\abs{z} > \gamma$, we have $\alpha_3(\abs{z}) \geq 
\alpha_3(\gamma) > 0$. Take $\gamma' = \min\{\overline{V}_\gamma,
\alpha_3(\gamma)\}$. We therefore have $V^\beta_N(\tilde{z}^+) \leq 
\overline{V}-\gamma'$ for all $z \in \overline{\mathcal{Z}}_r$.
\\FIXME: is it neccesary to find a uniform bound like this, or can we just 
argue if z=0 then V=0 so $V<\overline{V}$ if $z\neq0$ then $V<\overline{V}$?
\\
Recall that $\tilde{x}^+-x_m^+ = f(x_m,u(0;x_m)) - f(x,u(0;x_m)) - d + e^+$.
By the continuity of $V_N^\beta(\cdot)$, it follows that there exists
$\delta_1 > 0$ such that $V^\beta_N(x_m^+,\tilde{\mathbf{u}}) < \overline{V}$
for all $(z_m,e,d,e^+) \in \overline{\mathcal{S}}_\rho \times 
\delta_1\mathbb{B}^3$. From \eqref{eq:newimproved} it follows that 
$V^\beta_N(x^+,\mathbf{u}^+) \leq V^\beta_N(x_m^+,\tilde{\mathbf{u}}) <
\overline{V}$. Recall from \eqref{eq:vnbeta} that
$\beta V_f(\phi(N;x_m^+,\mathbf{u}^+)) \leq V^\beta_N(x^+,\mathbf{u}^+)$.
Since $\beta \geq \overline{V}/\alpha \geq 1$, we have that
$V_f(\phi(N;x_m^+,\mathbf{u}^+)) < \alpha$, which proves that
$\phi(N;x_m^+;\mathbf{u}^+) \in \text{int}(\mathbb{X}_f)$. From the 
continuity of $V_N^\beta(\cdot)$, we can choose $\rho > 0$ sufficiently
small so that $\overline{V}_N^\rho(z_m^+) \leq \overline{V}$. Taking
$\delta = \min\{\rho,\delta_1\}$ we have proved that $z_m^+ \in \overline{
\mathcal{S}}_\rho$ for all $(z_m,e,d,e^+) \in \overline{\mathcal{S}}_\rho 
\times \delta\mathbb{B}^3$. This implies that 
\begin{equation*}
x(k) \in \overline{\mathcal{C}}_\rho \subseteq \mathbb{X}_0 \text{for all}
k \in \mathbb{I}_{\geq 0}
\end{equation*}
and also that $x_m(k) \in \mathbb{X}_0$ for all $k \in \mathbb{I}_{\geq 0}$.
Hence, \eqref{eq:sras1} and \eqref{eq:sras1} hold with 
$\mathcal{X}_N = \mathbb{X}_0$.
Robust Stability. This section follows exactly the corresponding part of
Theorem \ref{th:mainsras} because we did not mention the feasibility 
restoration step therein. 
\end{proof}
When $\abs{d},\abs{e} \rightarrow 0$, it follows from the definition of 
$\overline{\mathcal{C}}_\rho$ that
$\overline{\mathcal{C}}_\rho \rightarrow \mathbb{X}_0$ and SRAS holds over a 
set approaching the admissible set of initial conditions. Combined with
Proposition \ref{pp:admissibleset}, in the limit of small disturbances and
large $\overline{V}$, the region of attraction for the disturbed system 
without state constraints converges to the closure of the restricted 
feasible set.

\section{Robust economic MPC}
\subsection{Economic MPC problem}
\label{sec:empc}
Economic MPC refers to the use of an economic stage cost in place of the
(typically quadratic) regulation stagecost traditionally used in MPC. The 
key difference between the economic stage cost and a typical regulation
stagecost is that the economic stage cost is almost never positive definite
with respect to the steady state regardless of whether the process is
optimally operated at steady state or not. As such, the range of $\ell(
\cdot)$ is no longer $\mathbb{R}_{\geq 0}$ as was part of Assumption 
\ref{as:continuity} and also fails Assumption \ref{as:pdstagecost}. In
the economic case, it is also not obvious what to use as a
terminal cost and terminal control law to satisfy Assumption 
\ref{as:terminalstability}. As a first step to addressing these issues,
we present a new set of assumptions that can be expected to hold for 
when considering general economic stagecosts. Note that because only
the form of $\ell(\cdot)$, $V_f(\cdot)$ and $\mathbb{X}_f$ change in 
the case of Economic MPC, we still refer to \eqref{eq:system} - 
\eqref{eq:closed} as the nominal system.
\begin{assumption}
\label{as:newcontinuity}
The model $f:\mathbb{R}^n \times \mathbb{R}^m \rightarrow \mathbb{R}^n$, 
stage cost $\ell: \mathbb{R}^n \times \mathbb{R}^m \rightarrow \mathbb{R}$ 
and terminal cost $V_f:\mathbb{R}^n \rightarrow \mathbb{R}$ are continuous.
\end{assumption}
\begin{assumption}
\label{as:newterminalstability}
There exists a terminal control law $\kappa_f:\mathbb{X}_f \rightarrow
\mathbb{U}$ such that
\begin{align}
%\label{eq:invariant}
f(x,\kappa_f(x)) &\in \mathbb{X}_f \qquad &\forall x \in \mathbb{X}_f\\
%\label{eq:terminalstability}
V_f(f(x,\kappa_f(x))) &\leq V_f(x)-\ell(x,\kappa_f(x))+\ell(x_s,u_s), 
&\forall x \in \mathbb{X}_f
\end{align}
\end{assumption}
We leave Assumption \ref{as:constraints} unchanged.
%\begin{assumption}
%\label{as:newconstraints1}
%The sets $\mathbb{U}$, $\mathbb{X}$ and $\mathbb{X}_f$ are compact. 
%The sets $\mathbb{X}$ and $\mathbb{X}_f$
%contain the origin in their interiors and $\mathbb{X}_f \subset
%\mathbb{X}$.
%\end{assumption}

%\begin{assumption}
%\label{as:newconstraints2}
%The set $\mathbb{U}$ is compact. The sets $\mathbb{X}$ and $\mathbb{X}_f$
%are closed, contain the origin in their interiors and $\mathbb{X}_f \subset
%\mathbb{X}$. The functions $\ell(\cdot)$ and $V_f(\cdot)$ are bounded below
%over their entire domains.
%\end{assumption}
%Assumptions \ref{as:newcontinuity}, \ref{as:newterminalstability} and 
%\ref{as:notpdstagecost} replace Assumptions \ref{as:continuity}, 
%\ref{as:terminalstability} and \ref{as:pdstagecost}. 
%We may use either of Assumptions \ref{as:newconstraints1}, 
%\ref{as:newconstraints2} to replace Assumption \ref{as:constraints}.
 
Assumptions \ref{as:newcontinuity}, %\ref{as:newconstraints1},
and \ref{as:newterminalstability}
replace Assumptions \ref{as:continuity}, 
%\ref{as:constraints}
and \ref{as:pdstagecost}. 
To replace the claim that $\ell(\cdot)$ is positive definite from 
Assumption \ref{as:pdstagecost}, we make the following definition and 
assumption.
\begin{definition}
\label{def:strictdissip}
The system \eqref{eq:system} is strictly dissipative with respect to the
supply rate %$s:\mathbb{X} \times \mathbb{U} \rightarrow \mathbb{R}$ 
$s:\mathbb{R}^n \times \mathbb{R}^m \rightarrow \mathbb{R}$ if
there exists a storage function %$\lambda:\mathbb{X} \rightarrow \mathbb{R}$
$\lambda:\mathbb{R}^n \rightarrow \mathbb{R}$ and a positive definite 
function %$\rho:\mathbb{X} \rightarrow \mathbb{R}_{\geq 0}$
$\rho:\mathbb{R}^n \rightarrow \mathbb{R}_{\geq 0}$ such that for all 
$(x,u) \in \mathbb{X} \times \mathbb{U}$
\begin{equation}
\lambda(f(x,u)) - \lambda(x) \leq -\rho(x-x_s)+s(x,u)
\end{equation}
\end{definition}
\begin{assumption}
\label{as:strictdissip}
The system \eqref{eq:system} is strictly dissipative with respect to the
supply rate $s(x,u) = \ell(x,u) - \ell(x_s,u_s)$.
\end{assumption}
FIXME: does strict dissipativity imply the solution of the steady-state
problem is unique? \\
Assumption \ref{as:strictdissip} is sufficient to establish that 
the process is suboptimally
operated off of steady state, that is, FIXME.
\begin{assumption}
\label{as:storagecontinuity}
There exists a storage function $\lambda(\cdot)$ satisfying Assumption
\ref{as:strictdissip} that is continuous.
% FIXME: what is this equivalent to? look it up
\end{assumption}
Finally, we require the following assumption.
\begin{assumption}
\label{as:adhoc}
One of the following conditions hold: FIXME
\begin{enumerate}
\item \label{as:adhoc1} $\mathbb{X}$ is compact.
\item \label{as:adhoc2} There exists a positive definite function 
$\rho(\cdot)$ satisfying Assumption \ref{as:strictdissip} that is of class 
$\mathcal{K}_\infty$ and there exists a terminal control law 
satisfying Assumption \ref{as:newterminalstability} that is stabilizing.
\end{enumerate}
\end{assumption}

\subsection{Rotated EMPC}
The key to proving the robust stability of suboptimal EMPC is to show that
the EMPC assumptions of section \ref{sec:empc} collectively imply the
assumptions of \ref{sec:mpc}. To this end, we define the rotated stagecost
, terminal cost and resulting objective function
\begin{align}
\label{eq:rotatedstagecost}
&L(x,u) = \ell(x,u)-\ell(x_s,u_s)+\lambda(x)-\lambda(f(x,u)) \\
\label{eq:rotatedterminalcost}
&\overline{V}_f = V_f(x)-V_f(x_s)+\lambda(x)-\lambda(x_s) \\
\label{eq:rotatedobjective}
&\overline{V}_N(x,\mathbf{u}) = \sum_{k=0}^{N-1}L(x(k),u(k))+\overline{V}_f(
x(N))
\end{align}
Note that only the objective function has changed from the original problem
; the feasible set \eqref{eq:admissibleinputs} remains unchanged.
\begin{proposition}
\label{pp:equivalent}
For all $\mathbf{u} \in \mathbb{R}^{mN}$, the 
rotated objective function $\overline{V}_N(\cdot)$ from 
\eqref{eq:rotatedobjective} and unrotated objective
function $V_N(\cdot)$ from \eqref{eq:objective} differ only by a constant 
(with respect to $\mathbf{u}$). In particular the (sub)optimal solution
sets of the respective problems are identical.
\end{proposition}
\begin{proof}
The proof is instructive and so we reproduce the proof from 
\citep{amrit:rawlings:angeli:2011}. Expanding the rotated objective 
function, we have
\begin{align*}
\overline{V}_N(x,\mathbf{u}) &= \sum_{k=0}^{N-1}L(x(k),u(k))+\overline{V}_f(
x(N)) \\
&= \sum_{k=0}^{N-1}\ell(x(k),u(k))-\ell(x_s,u_s)+\lambda(x(k))-
\lambda(x(k+1))\\
&\quad +V_f(x(N))-V_f(x_s)+\lambda(x(N)) - \lambda(x_s) \\
&= \sum_{k=0}^{N-1}\ell(x(k),u(k))+V_f(x(N)) \\
&\quad -N\ell(x_s,u_x)+\lambda(x)-\lambda(x(N))-V_f(x_s)+\lambda(x(N))-
\lambda(x_s) \\
&= V_N(x,\mathbf{u}) -N\ell(x_s,u_x)+\lambda(x)-V_f(x_s)-\lambda(x_s)
\end{align*}
Since $N\ell(x_s,u_x),\lambda(x),V_f(x_s)$ and $\lambda(x_s)$ are all 
constant with respect to $\mathbf{u}$, this completes the proof.
\end{proof}
With Proposition \ref{pp:equivalent}, we may determine stability for the
original system by examining the stability of the rotated system. Before
continuiing, we also reproduce a theorem for the economic performance
of EMPC from \citep{amrit:rawlings:angeli:2011}. 
\begin{definition}
The asymptotic average performance of a trajectory $(x(k),u(k))$, $k \in 
\mathbb{I}_{\geq 0}$ is defined as
\begin{equation*}
\lim_{T \rightarrow \infty} \ell_{av}(T) = \sum_{k=0}^T\frac{\ell(x(k),u(k))}{T+1}
\end{equation*}
If this limit does not exist, then the asymptotic average performance is
defined as the interval $[\liminf_{T \rightarrow \infty}\ell_{av},
\limsup_{T \rightarrow \infty}\ell_{av}]$ and statements of the form ``the 
asymptotic average performance is'' should be interpreted as ``every 
element of the asymptotic average performance is''.
\end{definition}
\begin{theorem}
If Assumptions \ref{as:newcontinuity}, \ref{as:constraints} and 
\ref{as:newterminalstability} hold, then the asymptotic average performance
of the system \eqref{eq:system} in closed-loop with the optimal EMPC 
controller FIXME is better than or equal to the performance of the
optimal steady state.
\end{theorem}
Notice that we do not need strict dissipativity (Assumption 
\ref{as:strictdissip}) to guarantee superior asymptotic average performance
relative to the optimal steady state. \\
\begin{lemma}
\label{lem:rotatedsame}
The pair $(\overline{V}_f(\cdot),L(\cdot))$ satisfies Assumption
\ref{as:terminalstability} if and only if the pair $(V_f(\cdot),
\ell(\cdot))$ satisfies Assumption \ref{as:newterminalstability} for the
same terminal region $\mathbb{X}_f$ and controller $\kappa(\cdot)$.
\end{lemma}
\begin{proof}
The proof is included in \citep{amrit:rawlings:angeli:2011}.
\end{proof}
Note that Lemma \ref{lem:rotatedsame} guarantees that we may use the
terminal controller of Assumption \ref{as:newterminalstability} to 
construct the warm start of the rotated system \eqref{eq:warm} without
knowledge of the storage function.

\begin{lemma}
\label{lem:underbound}
FIXME FIXME X not compact!

Given a positive definite function $\rho(x)$ defined on a compact set C
containing the origin, there exists a function $\gamma \in \mathcal{K}$
such that
\begin{equation*}
\gamma(\abs{x}) \leq \rho(x) \quad \forall x \in C
\end{equation*}
$\gamma$ can be chosen to be of class $\mathcal{K}_\infty$.
\end{lemma}


\begin{lemma}
\label{lem:assumptions}
The assumptions of Section \ref{sec:empc} (SET) collectively imply 
the ssumptions of Section \ref{sec:mpc} (SET). FIXME 
\end{lemma}
\begin{proof}
We examine each of the assumptions in Section \ref{sec:mpc} in turn. \\
Assumption \ref{as:continuity}. Since $\lambda(\cdot)$ is 
continuous by Assumption \ref{as:storagecontinuity}, we have from 
Assumption \ref{as:newcontinuity} that $L(\cdot)$,
$\overline{V}_f(\cdot)$ and
$\overline{V}_N(\cdot)$ are continuous. From their definitions, it is also
clear that $L(x_s,u_s) = 0$ and $\overline{V}_f(x_s) = 0$. From equation
\eqref{eq:rotatedstagecost} and Assumption \ref{as:strictdissip}, we have
\begin{align}
\label{eq:newposdef}
L(x,u) &= \ell(x,u)-\ell(x_s,u_s)+\lambda(x)-\lambda(f(x,u)) \\ \nonumber
&\geq \ell(x,u)-\ell(x_s,u_s)+\rho(x-x_s)-s(x,u) \\ \nonumber
&= \rho(x-x_s)
\end{align}
We now must prove that $\overline{V}_f$ is nonnegative. \\
Suppose Assumption \ref{as:adhoc}.\ref{as:adhoc2} holds. Let $x \in 
\mathbb{X}_f$ and consider the closed-loop system $x^+ = f(x,\kappa(x))$.
From Lemma \ref{lem:rotatedsame}, we have for all $k \in \mathbb{I}_{\geq 0}$
\begin{equation*}
\overline{V}_f(x(k+1;x)) - \overline{V}_f(x(k;x)) \leq -L(x(k;x),\kappa_f(
x(k;x)))
\end{equation*}
Summing up, we have
\begin{equation*}
\overline{V}_f(x) \geq \sum_{k=0}^{T-1} L(x(k;x),\kappa_f(x(k;x))) +
\overline{V}_f(x(T;x))
\end{equation*}
Since $\kappa$ is a stable control law by Assumption 
\ref{as:adhoc}.\ref{as:adhoc2}, we take the limit as $T \rightarrow \infty$
\begin{equation*}
\overline{V}_f(x) \geq \sum_{k=0}^{T-1} L(x(k;x),\kappa_f(x(k;x))) \geq 0
\end{equation*}
The proof of this fact when Assumption \ref{as:adhoc}.\ref{as:adhoc2} holds
is given in \citep{amrit:rawlings:angeli:2011}. This concludes the proof
of Assumption \ref{as:continuity}. \\
Assumption \ref{as:terminalstability}. This Assumption follows immediately
from Assumption \ref{as:newterminalstability} and Lemma 
\ref{lem:rotatedsame}. \\
Assumption \ref{as:pdstagecost}. Suppose Assumption 
\ref{as:adhoc}.\ref{as:adhoc1} holds. Then by Lemma \ref{lem:underbound}
we have immediately that $L(x,u) \geq \alpha$, $\alpha \in 
\mathcal{K}_\infty$ because $\mathbb{X}$ is compact. Now suppose Assumption
\ref{as:adhoc}.\ref{as:adhoc2}. Then we have $L(x,u) \geq \alpha$, 
$\alpha \in \mathcal{K}_\infty$ immediately because we can select $\rho(
\cdot)$ from \eqref{eq:newposdef} to be of class $\mathcal{K}_\infty$. \\
\end{proof}
\begin{lemma}
\label{lem:norotation}
Consider the conditions required for an admissible suboptimal input on the 
rotated system, from \eqref{eq:feasible} - \eqref{eq:terminal}
\begin{align*}
\mathbf{u}^+ &\in \mathcal{U}_N(x^+) \\
\overline{V}_N(x^+,\mathbf{u}^+) &\leq \overline{V}_N(x^+,\tilde{
\mathbf{u}}) \\
\overline{V}_N(x^+,\mathbf{u}^+) &\leq \overline{V}_f(x^+) \ \text{if} \ x^+ 
\in r\mathbb{B} \\
\end{align*}
These conditions hold if and only if the following conditions hold for the
nonrotated problem
\begin{align}
\label{eq:nrfeasible}
\mathbf{u}^+ &\in \mathcal{U}_N(x^+) \\
\label{eq:nrimproved}
V_N(x^+,\mathbf{u}^+) &\leq V_N(x^+,\tilde{\mathbf{u}}) \\
\label{eq:nrterminal}
V_N(x^+,\mathbf{u}^+) &\leq V_f(x^+) -N\ell(x_s,u_s) \ \text{if} \ x^+ 
\in r\mathbb{B} \\
\end{align}
\end{lemma}
\begin{proof}
We emphasize again that rotation of the control problem does not change the
feasible set, and therefore condition \eqref{eq:nrfeasible} is unchanged.
We may obtain condition \ref{eq:nrimproved} immediately from 
Proposition \ref{pp:equivalent} by noting that both objective functions
involve the same initial condition. For the condition \eqref{eq:nrterminal}
we use the last line of the proof of Proposition \ref{pp:equivalent} and 
\eqref{eq:rotatedterminalcost} to compute
\begin{align*}
0 &\leq \overline{V}_f(x^+) - \overline{V}_N(x^+,\mathbf{u}^+) \\
&= V_f(x^+)+\lambda(x^+)-V_f(x_s)-\lambda(x_s) \\
&\quad -V_N(x^+,\mathbf{u}^+)+N\ell(x_s,u_s)-\lambda(x^+)+\lambda(x_s)
+V_f(x_s) \\
&= V_f(x^+)-V_N(x^+,\mathbf{u}^+)+N\ell(x_s,u_s)
\end{align*}
The key point of this lemma is that we do \emph{not} need to know what
the storage function is to implement suboptimal EMPC.
\end{proof}
Before proceeding, we make one important comment. When we apply the methods
of the proof of Lemma \ref{lem:norotation} to check whether the nonrotated
cost drops when construction the warm start as in \eqref{eq:warm}, we find
\begin{align*}
0 &\leq \overline{V}_N(x,\mathbf{u}) - \overline{V}_N(x^+,\tilde{
\mathbf{u}}) \\
&= V_N(x,\mathbf{u})-N\ell(x_s,u_s)+\lambda(x)-\lambda(x_s)-V_f(x_s)\\
&\quad -V_N(x^+,\tilde{\mathbf{u}})+N\ell(x_s,u_s)-\lambda(x^+)+
\lambda(x_s)+V_f(x_s) \\
&= V_N(x,\mathbf{u})-V_N(x^+,\tilde{\mathbf{u}})+\lambda(x)-\lambda(x^+)
\end{align*}
Therefore, while we are still guaranteed that the cost drops for the rotated
system, the cost may \emph{increase} for the nonrotated system when the 
warm start is created, though
this does not affect stability. This is, in fact, a common feature in
EMPC \citep{angeli:amrit:rawlings:2012}.
FIXME: I think I need to think about this a little more. \\
With these lemmas, we may now apply the previous theorems to the EMPC 
closed-loop systems.
\begin{theorem}
Suppose Assumptions \ref{as:newcontinuity}, \ref{as:constraints},
\ref{as:newterminalstability}, \ref{as:strictdissip}, 
\ref{as:storagecontinuity} and \ref{as:adhoc} hold. Consider the 
closed-loop system $x^+ = f(x,\kappa_N(x))$ where $\kappa_N(x)$ selects
any input satisfying the conditions \eqref{eq:nrfeasible} - 
\eqref{eq:nrterminal} with the warm start defined in \eqref{eq:warm}. 
Then the closed-loop
system is asymptotically stable on arbitrarily large subsets of
$\mathcal{X}_N$.
\end{theorem}
\begin{proof}
The proof follows directly from Lemmas \ref{lem:assumptions} and 
\ref{lem:norotation} and Theorem \ref{th:nominalstability}.
\end{proof}
\begin{theorem}
Suppose Assumptions \ref{as:newcontinuity}, \ref{as:constraints},
\ref{as:newterminalstability}, \ref{as:strictdissip}, 
\ref{as:storagecontinuity}, \ref{as:adhoc} and 
\ref{as:vn0continuous} hold. Consider the 
closed-loop system $x^+ = f(x,\kappa_N(x))$ where $\kappa_N(x)$ selects
any input satisfying the conditions \eqref{eq:nrfeasible} - 
\eqref{eq:nrterminal} with the warm start defined in \eqref{eq:warm}. 
Then the closed-loop
system is SRAS on $\mathcal{C}_\rho$ where $\mathcal{C}_\rho$ is defined by
equation \eqref{eq:crho} for the \emph{rotated} system.
\end{theorem}
\begin{proof}
Note that neither the feasibility restoration step \eqref{eq:resto} nor
its related Assumption \ref{as:vn0continuous} refer to any rotated 
functions. We may therefore simply cite Lemmas \ref{lem:assumptions} and 
\ref{lem:norotation} and Theorem \ref{th:mainsras}.
\end{proof}
From this theorem, while we still do not need to know the storage function 
to implement suboptimal EMPC, we must be able to compute it to a priori
determine the set over which EMPC is robust. This is not a significant 
drawback, however, because nobody really knows how to compute 
$\mathcal{C}_\rho$ even for systems with a quadratic stagecost and terminal
cost and nobody in industry could possibly care less FIXME. \\
Finally, while it is possible to apply the techniques of Section
\ref{sec:nostateconstraints}, we must know the storage function explicitly
to be able to implement the suboptimal controller.
\begin{theorem}
Suppose Assumptions \ref{as:newcontinuity}, \ref{as:nostateconstraints},
\ref{as:newterminalstability}, \ref{as:strictdissip}, 
\ref{as:storagecontinuity} and \ref{as:adhoc}.\ref{as:adhoc2} hold. 
Consider the 
closed-loop system $x^+ = f(x,\kappa_N(x))$ where $\kappa_N(x)$ selects
any input satisfying the (rotated) conditions
\begin{align*}
&\mathbf{u}^+ \in \mathbb{U}^N \\
&\overline{V}^\beta_N(x^+,\mathbf{u}^+) \leq \overline{V}^\beta_N(x^+,\tilde{
\mathbf{u}}) \\
&\overline{V}^\beta_N(x^+,\mathbf{u}^+) \leq \beta \overline{V}_f(x^+) 
\quad \text{when} x^+ \in r\mathbb{B}
\end{align*}
with the warm start defined in \eqref{eq:warm} and $\overline{V}^\beta_N(
\cdot)$ defined by \eqref{eq:vnbeta}.
Then the closed-loop system is SRAS on $\overline{\mathcal{C}_\rho}$ 
where $\mathcal{C}_\rho$ is defined by
equation \eqref{eq:crho} for the rotated system. Furthermore, for the same
controller, but the closed-loop system $x^+ = f(x,\kappa_N(x+e))+d$, 
this new system is SRAS on the
set $\overline{\mathcal{C}}_\rho$ defined for the rotated system by 
\eqref{eq:crhonew}.
\end{theorem}

\subsection{Terminal region design and miscellaneous}
If the storage function is known, a variety of standard techniques can
be used to design $V_f(\cdot)$, $\kappa_f(\cdot)$ and $\mathbb{X}_f$ for
the rotated system. See \citep{amrit:rawlings:angeli:2011}, for example 
FIXME.
In the more common case that the storage function is not known, we may
still easily design them for the nonrotated system so that Assumptions
\ref{as:newcontinuity} and \ref{as:newterminalstability} are satisfied.
We recount here the assumption and procedure from 
\citep{amrit:rawlings:angeli:2011}. For clarity of exposition, we again
revert to deviation variables (i.e. $(x_s,u_x) = (0,0)$).
\begin{assumption}
\label{as:construction}
The functions $f(\cdot)$ and $\ell(\cdot)$ are twice continuously 
differentiable on $\mathbb{R}^n \times \mathbb{R}^m$ and the linearized
system $x^+ = Ax+Bu$ is stabilizable, where $A = f_x(0,0)$ and 
$B = f_u(0,0)$.
Furthermore, both $\mathbb{X}$ \emph{and} $\mathbb{U}$ contain the 
origin in their interiors.
%FIXME: This is clearly required (and was said during a proof in the paper,
%but not stated as part of the assumptions of the theorem in that section.
\end{assumption}
% FIXME: do we need a detectability assumption on $\overline{\ell}_{xx}$ and
% A?
The procedure is as follows
\begin{enumerate}
\item Select any matrix $K$ such that $A_k=A+BK$ is stable. Define 
$\overline{\ell}(x) = \ell(x,Kx)-\ell(0,0)$.
\item Select any compact set $\mathcal{C}$. Calculate the maximum 
eigenvalue $\lambda^\star$ of the hessian of $\ell_{xx}(x)$ on $\mathcal{C}$.
\footnote{$\lambda^\star$ exists because the eigenvalues of a matrix are continuous
functions of its elements, and the elements of this matrix are continuous
functions of x by Assumption \ref{as:construction}}
\item Select any $\alpha > -\lambda^\star$.
\item Compute $Q = (\lambda^\star+\alpha)I$ and $q = \overline{\ell}_x(0)$.
\item Compute P by solving the Lyapunov equation $A_k'PA_k-P=-Q$ and compute
$p=(I-A_k')^{-1}q$.
\item Define the terminal cost function to be
\begin{equation}
\label{eq:costformula}
V_f(x) = \frac{1}{2}x'Px+p'x
\end{equation}
\item Define the terminal set to be a level set of $V(x) = x'Px$
\begin{equation}
\label{eq:setformula}
\mathbb{X}_f = \text{lev}_bV
\end{equation}
where b must be chosen so that
\begin{align*}
V(f(x,Kx))-V(x) &\leq - \frac{1}{2}x'Qx \quad \forall x \in \text{lev}_bV
\quad &\text{(linearization error)} \\
\text{lev}_bV &\subset \mathcal{C} &\text{(stagecost eigenvalue bound)} \\
(x,Kx) &\in \mathbb{Z} \quad \forall x \in \text{lev}_bV
&\text{(input constraints)}
\end{align*}
\end{enumerate}
Note that this construction does not require knowledge of the storage 
function $\lambda$.
\begin{theorem}[Terminal penalty construction]
\citep{amrit:rawlings:angeli:2011}
\label{th:construction}
Suppose Assumption \ref{as:construction} holds. Define $\mathbb{X}_f$ and 
$V_f(\cdot)$ by \eqref{eq:setformula} and \eqref{eq:costformula}. Then
$\mathbb{X}_f$ is nonempty and contains the origin in its interior,
$V_f(\cdot)$ exists and together they satisfy Assumption 
\ref{as:newterminalstability} with control law $\kappa_f(x) = Kx$.
\end{theorem}
It is important to emphasize that while these results have been stated
for suboptimal economic model predictive control, they all apply to 
optimal economic model predictive control as well. The main conclusion of
this paper is that, despite its novel nature, economic model predictive 
control inherits the laudable stability and robustness properties of
optimal (noneconomic) model predictive control.

\subsection{Asymptotic average performance of suboptimal EMPC}
Before concluding, we briefly present a novel result on the (nominal)
asymptotic average performance of suboptimal EMPC.
\begin{theorem}
Suppose Assumptions \ref{as:newcontinuity}, \ref{as:constraints} and 
\ref{as:newterminalstability} hold.  Consider the 
closed-loop system $x^+ = f(x,\kappa_N(x))$ where $\kappa_N(x)$ selects
any input satisfying the conditions \eqref{eq:nrfeasible} - 
\eqref{eq:nrterminal} with the warm start defined in \eqref{eq:warm}. 
We have that the asymptotic average performance
of this closed-loop system is better than or equal to the performance of the
optimal steady state.
\end{theorem}
\begin{proof}
% FIXME: too similar
%\tilde{\mathbf{u}} = \{u(1;x),u(2;x),\dots,u(N-1;x),\kappa_f(x(N;x))\}
From equation \eqref{eq:warm} and Assumption \ref{as:newterminalstability}, 
we have
\begin{align*}
V_N(x^+,\tilde{\mathbf{u}}) &= V_N(x,\mathbf{u})-\ell(x,u(0;x))+
\ell(x(N;x),\kappa_f(x(N;x))) \\
&\quad -V_f(x(N;x))+V_f(f(x(N;x),\kappa_f(x(N;x)))) \\
&\leq V_N(x,\mathbf{u})-\ell(x,u(0;x))+\ell(x_s,u_s) \\
\end{align*}
From Lemma \ref{lem:norotation}, we have
\begin{equation*}
V_N(x^+,\mathbf{u}^+) \leq V_N(x,\mathbf{u})-\ell(x,u(0;x))+\ell(x_s,u_s)
\end{equation*}
%FIXME: do we need that X \times U is compact here? I don't think so.
From this equation, we may use the analysis of 
\citep{angeli:amrit:rawlings:2012}{Theorem 1} to complete the proof.
\end{proof}
Note that this result does not require the assumption of strict 
dissipativity and consequently does not require asymptotic stability of
the closed-loop system. In fact, the only way for EMPC to strictly 
outperform the optimal steady state in terms of asymptotic average
performance is if the closed-loop system is unstable.\\
Is it possible to prove this for thie disturbed system? That would be a
nice result. FIXME \\
Talk about how its not possible to compare transient closed-loop 
performances. FIXME \\
Put some kind of example here. FIXME \\
Write something about Assumption \ref{as:storagecontinuity}. FIXME \\
Talk about how ``tight'' of an assumption dissipativity is, and how
strict dissipativity compares to it in that regard. FIXME \\
Need to talk about how to and what happens when the infinite horizon
economic cost-to-go is selected as the terminal penalty. FIXME \\

\section{Conclusion}
% FIXME: too similar to other papers
words words words

\subsection{Junk}

FIXME: are there better methods for relaxing the terminal constraint? I 
mean by ``better'' as having a larger feasible set for $\beta = 1$ so that 
if the terminal cost is the infinite horizon cost to go, the MPC cost is
optimal. The above achieves this for a subset of the terminal region
only. The method of \citep{rawlings:mayne:2009}{p. 151-153} looks like a
better method in these terms but it probably wouldn't work for 
suboptimal MPC.\\
FIXME: given that the steady-state optimum is in the interiors of 
$\mathbb{X}$ and $\mathbb{U}$ and that the system is strictly dissipative,
(and stabilizability) shouldn't you be able to prove that the optimal 
controller is linear and therefore $\overline{\ell}$ is actually positive
definite from Rishi's paper? 

The steady-state problem is defined as
\begin{equation}
\label{eq:steadystate}
\ell(x_s,u_s) = 
 \min \Big \{ \ell(x,u) \mid (x,u) \in \mathbb{Z}, \,  x = f(x,u) \Big \}
\end{equation}
Assumptions \ref{as:continuity},\ref{as:constraints} guarantee the 
steady-state problem \eqref{eq:steadystate} has a solution. FIXME: do they?
\begin{assumption}[Unique optimal steady state] 
\label{as:unique}
The solution of the steady-state problem \eqref{eq:steadystate} is unique.
\end{assumption}

FIXME: remove $\lambda$ continuous?
V must be locally bounded, but can be discontinuous away from zero

\end{comment}



\section{Appendix}
%\begin{proposition}
%Any optimal solution $\mathbf{u}^0(x)$ of $\mathbb{P}_N(x)$ satisfies 
%conditions \eqref{eq:feasible}, \eqref{eq:improved} 
%for all $x \in \mathcal{X}_N$ and \eqref{eq:terminal} for all 
%$x \in \mathcal{X}_f$.
%\end{proposition}
%\begin{proof}
%Conditions \eqref{eq:feasible} and \eqref{eq:improved} hold because of the 
%optimality of $\mathbf{u}^0(x)$. We now consider the final claim.
%Let $x(0) = x \in \mathbb{X}_f$, $u(0) = 
%\kappa_f(x(0))$ and $x(1) = f(x(0),u(0))$. By Assumption 
%\ref{as:terminalstability}, we have $V_f(x(1)) + \ell(x(0),u(0)) \leq 
%V_f(x(0))$ and $x(1) \in \mathbb{X}_f$. Next, let $u(1) = \kappa_f(x(1))$ 
%and $x(2) = f(x(1),u(1))$. Again by Assumption \ref{as:terminalstability}, 
%we have $V_f(x(2)) + \ell(x(1),u(1)) \leq V_f(x(1))$ and $x(2) \in 
%\mathbb{X}_f$ which, together with the previous inequality, imply 
%$V_f(x(2)) + \ell(x(1),u(1)) + \ell(x(0),u(0)) \leq V_f(x(0))$. By induction 
%and the definition of $V_N(\cdot)$ we have $V_N(x,\mathbf{u}_f) \leq V_f(x)$
%for $\mathbf{u}_f = {u(0),u(1),\dots,u(N-1)}$. By the optimality of 
%$\mathbf{u}^0(x)$, we conclude that $V_N(x,\mathbf{u}^0(x)) \leq V_f(x)$.
%\end{proof}

\begin{lemma}
If the set $\mathcal{Z}$ is positively invariant for the difference 
inclusion $z^+ \in H(z)$, $H(0) = \{0\}$, $0 \in \mathcal{Z}$ 
and there exists a Lyapunov function $V$ on $\mathcal{Z}$, then the 
origin is asymptotically stable on $\mathcal{Z}$.
\end{lemma}
\begin{proof}
% (similar) Also, why is this proof so long while the exponential proof was literally 4 lines?
From \eqref{eq:upperlyap} we have $\alpha_2(\abs{z}) \geq V(z)$ and therefore 
$\abs{z} \geq \alpha_2^{-1}(V(z))$.
Substituting this relation into \eqref{eq:decreaselyap}, we have
\begin{align*}
\max_{z^+ \in H(z)} V(z^+) &\leq V(z) - \alpha_3(\abs{z}) \\
&\leq V(z) - \alpha_3(\alpha_2^{-1}(V(z))) \\
&\leq \sigma_1(V(z))
\end{align*}
where $\sigma_1(\cdot) = (\cdot) - \alpha_3 \circ \alpha_2^{-1} (\cdot)$. Using the properties of
$\mathcal{K}$ functions (see e.g. \citep{khalil:2002}), we conclude that $\sigma_1(\cdot)$ is continuous
on its domain $\mathbb{R}_{\geq 0}$, zero at zero and that $0 < \sigma_1(s) < s$ for $s > 0$. However, for 
$\sigma_1(\cdot)$ to be a $\mathcal{K}_\infty$ function, it must also be monotonically increasing and unbounded.
To satisfy these properties, we define
\begin{equation*}
\sigma_2(s) = \max_{s' \in [0,s]} \sigma_1(s'), \quad s \in \mathbb{R}_{\geq 0}
\end{equation*}
The set $[0,s]$ is compact and $\sigma_1(\cdot)$ is continuous, so the maximum exists. By optimality, 
$\sigma_2(\cdot)$ is nondecreasing. Suppose $\sigma_2(\cdot)$ is discontinuous at $c \in \mathbb{R}_{\geq 0}$, 
i.e., has a positive jump. Define
\begin{equation*}
a_1 = \lim_{s \nearrow c} \sigma_2(s), \quad a_2 = \lim_{s \searrow c} \sigma_2(s)
\end{equation*}
where the limits exist because $\sigma_2(\cdot)$ is nondecreasing \cite[p. 149-150]{bartle:sherbert:2000}.
Because $\sigma_1(\cdot)$ is continuous, we must have that $\sigma_1(c) \leq a_1 < a_2$ or the limit of 
$\sigma_2(\cdot)$ from below is violated. Since $\sigma_1(c) < a_2$ and $\sigma_1(\cdot)$ is continuous, we must 
have $\sigma_1(s) = a_2$ for some $s < c$ or the limit from above is violated. However, $\sigma_1(s) = a_2$ 
for some $s < c$ violates the limit from below, and we therefore conclude that $\sigma_2(\cdot)$ is continuous by
contradiction. We define
\begin{equation*}
\sigma(s) = \frac{1}{2}(s+\sigma_2(s)), \quad s \in \mathbb{R}_{\geq 0}
\end{equation*}
where now $\sigma(\cdot)$ is continuous, strictly increasing and unbounded and $\sigma(0) = 0$.
We conclude that $\sigma(\cdot) \in \mathcal{K}_\infty$ and note that $\sigma_1(s) < \sigma(s) < s$ for 
$s > 0$. Therefore
\begin{equation*}
\max_{z^+ \in H(z)} V(z^+) \leq \sigma(V(z))
\end{equation*}
Repeated use of the above and then \eqref{eq:upperlyap} gives
\begin{equation*}
\max_{z(k+1) \in H(z(k)),k=0,\dots,i-1} V(z(i)) \leq \sigma^i(\alpha_2(\abs{z}))
\end{equation*}
where $\sigma^i(\cdot)$ represents the composition of $\sigma(\cdot)$ with itself $i$ times. 
Using \eqref{eq:lowerlyap}, we have
\begin{align*}
\max_{z(k+1) \in H(z(k)),k=0,\dots,i-1} \abs{z(i)} %&\leq \alpha_1^{-1}(\sigma^i(\alpha_2(\abs{z}))) \\
&\leq \beta(\abs{z},i)
\end{align*}
where
\begin{equation*}
\beta(s,i) = \alpha_1^{-1}(\sigma^i(\alpha_2(s)))
\end{equation*}
To prove that $\beta(\cdot)$ is a $\mathcal{KL}$ function, we first note that for all $s \geq 0$, the sequence 
$w_i = \sigma^i(\alpha_2(s))$ is nonincreasing with $i$ and bounded below by zero. The sequence $w_i$ therefore 
converges, say, to $a$, as $i \rightarrow \infty$. By the definition of $\sigma^i$, we have that $w_i \rightarrow
a$ and $\sigma(w_i) \rightarrow a$ as $a \rightarrow \infty$. Since $\sigma(\cdot)$ is continuous, we also have 
that $\sigma(w_i) \rightarrow \sigma(a)$ as $i \rightarrow \infty$. As a result, $\sigma(a) = a$, which implies 
that $a = 0$. Therefore, for all $s \geq 0$, $\beta(s,i) \rightarrow 0$ as $i \rightarrow \infty$. From the
properies of $\mathcal{K}$ functions, we have that $\alpha_1^{-1}(\sigma^i(\alpha_2(s)))$ is a
$\mathcal{K}$ function for all $i \geq 0$. We conclude that $\beta(\cdot) \in \mathcal{KL}$ and the proof is
complete.
\end{proof}


\bibliographystyle{abbrvnat}
\bibliography{abbreviations,articles,books,unpub,proceedings}

\end{document}

