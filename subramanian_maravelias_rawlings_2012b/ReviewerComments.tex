\documentclass{article}
\usepackage{url}
\usepackage{almostfull}
\title{Response to Reviewer Comments}
\date{\today}
\newcommand{\response}[1]{\textbf{Response}: #1}
\begin{document}
\maketitle

\paragraph{Reviewer-1}

\begin{enumerate}
\item
The paper presents a rigorous formulation and analysis of economic MPC
for the problem of inventory management in supply chains. A novel cost
function is presented;  the approach enables the integration of the
scheduling and inventory control tasks as well.  A number of
illustrative examples are presented. I do not see issues of technical
correctness with the paper; however, the paper is hampered somewhat by
a lack of accessibility in the presentation that may be acceptable in
other journals, but I would hope not be the case in CACE.  This is
illustrated in Section 2, for example. 
A real problem will likely have 1) longer (and uncertain) lead times and 2) uncertain demand.
In these scenarios, I would perhaps not be so eager to have MPC with an economic objective 
function, as having guarantees of stability would not be sufficient. Somehow the economic 
objective function would have to allow a means to incorporate potentially significant robustness 
margins in the response, while still maintaining meaningfulness as a metric. Can the authors 
justify why would the proposed approach be better than a well-coordinated hierarchical 
approach, which would maintain a separation between the control and
scheduling layers?


A minor issue: some of the references need to be updated, particularly
those that involve the existence of journal versions of papers cited
from conferences.  This applies to References [2] and [13], for
example. 
 

\response{The reviewer raises two related but distinguishable points: (i) the suitability of the 
economic objective function, and (ii) the choice of the solution approach. Regarding the 
objective function, we believe that the selection of an “economic” metric is the natural one in 
the context of supply chain (SC), where managers are interested in minimizing costs and not 
necessarily maintaining certain levels of inventory (tracking problem). Now, given the adoption 
of an economic objective, our goal is to develop new results for EMPC in systems with 
integrators, something that has not been done in the past. In other words, we aim to expand 
the scope of these methods in new classes of problems. Of course, a number of other 
approaches could be used to address the same problems (hierarchical decomposition is one of 
them). However, no theoretical results are available for the closed-loop performance of these 
approaches, even when all parameters are known deterministically. Thus, although they may 
appear promising, there is no guarantee they will work; in fact, counter examples showing how 
these methods fail can be constructed.

We have made referencesin the text to papers detailing Model
predictive control. A complete introduction to MPC was out of the
scope of the paper. We have also made the corrections to the references.}
\end{enumerate}
\paragraph{REVIEWER-2}
\begin{enumerate}
\item Since the authors are dealing with inventory management in supply chains, which often have 
integer units (books, parts, etc.), they should at least discuss the limits of their continuous 
variable formulation for these problems. While the difference between 50.5 and 51 units/day will 
not radically change a solution, presumably 1.25 units/day would require 3 days of 1 unit and 1 
day of 2 units. 

\response{The reviewer raises two interesting points. The first concerns the dynamic model of 
the supply chain (section 3) and the second the production model at each manufacturing node.
Regarding the first, please note that our primary focus is the supply chain of process industries, 
where the major products are commodity chemicals and fluids (often shipped using pipelines, 
trucks, etc). Thus, we believe that the treatment of the variables in the dynamic model as 
continuous variables is reasonable. Note that even if products are produced in discrete 
amounts (e.g., due to batch size restrictions), the orders and shipments, and thus inventory 
levels, can be continuous variables. Regarding the second point, we note that the model can be 
easily modified to enforce production of discrete amounts be simply
setting $\underline{B}_i = \bar{B}_i$ in eq. (24); 
i.e. without definig variable $B_i$ as an integer. However, the scheduling model does have binary 
 $X_{(i,t)}$and $Z_{(i,i’,t)}$.}

\item While the authors discuss periodic supply chains, I would like to better understand how this 
relates to a typical Federal Express or UPS problem, where deliveries are made daily, with 
perhaps Saturday mornings but no Saturday afternoon or Sunday service.
\response{Restrictions on transportation and production can be easily handled through
constraints on the corresponding variables. For example, in a problem with a 14 day horizon 
divided into 336 1-hour periods, shipments between Saturday 12:00 pm and Monday 6:00 am 
can be forbidden by setting $S_{i,t} = 0 for t {132, …, 174}$ (assuming zero time delay). Similar 
restrictions can be placed on orders and production (e.g., setting scheduling variables to zero 
during weekends).}
\end{enumerate}
\end{document}
